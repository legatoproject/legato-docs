This section contains details about known platform constraints and advanced info about Yocto and Legato.





\hyperlink{platformConstraints}{Platform Constraints} ~\newline
 \hyperlink{yoctoMain}{Yocto Info} ~\newline






Copyright (C) Sierra Wireless Inc. Use of this work is subject to license. \hypertarget{platformConstraints}{}\section{Platform Constraints}\label{platformConstraints}
The A\+R755x, A\+R8652, and W\+P8548 platforms have the following limitations\+:

\hyperlink{platformConstraintsAudio}{Audio} ~\newline
 \hyperlink{platformConstraintsSecStorage}{Secure Storage}





Copyright (C) Sierra Wireless Inc. Use of this work is subject to license. \hypertarget{platformConstraintsAudio}{}\subsection{Audio}\label{platformConstraintsAudio}
Audio functions are constrained by the following platform limitations for A\+R755x, A\+R8652, and W\+P8548.

This table lists possible audio connections\+:

\begin{TabularC}{7}
\hline
\rowcolor{lightgray}{\bf Input stream }&\PBS\centering {\bf Spkr }&\PBS\centering {\bf U\+S\+B Tx }&\PBS\centering {\bf P\+C\+M Tx }&\PBS\centering {\bf I2\+S Tx }&\PBS\centering {\bf Mdm Tx }&\PBS\centering {\bf Rec  }\\\cline{1-7}
Mic &\PBS\centering &\PBS\centering &\PBS\centering &\PBS\centering &\PBS\centering X($\ast$) &\PBS\centering X($\ast$) \\\cline{1-7}
U\+S\+B Rx &\PBS\centering &\PBS\centering &\PBS\centering &\PBS\centering &\PBS\centering X &\PBS\centering \\\cline{1-7}
P\+C\+M Rx &\PBS\centering &\PBS\centering &\PBS\centering &\PBS\centering &\PBS\centering X &\PBS\centering \\\cline{1-7}
I2\+S Rx &\PBS\centering &\PBS\centering &\PBS\centering &\PBS\centering &\PBS\centering &\PBS\centering X \\\cline{1-7}
Mdm Rx &\PBS\centering X($\ast$) &\PBS\centering X &\PBS\centering X &\PBS\centering X &\PBS\centering &\PBS\centering X \\\cline{1-7}
Pb &\PBS\centering X($\ast$) &\PBS\centering X &\PBS\centering X &\PBS\centering X &\PBS\centering X &\PBS\centering \\\cline{1-7}
\end{TabularC}
($\ast$) A\+R platform only

Pb = file playback ~\newline
 Rec = file recording

Only 1 point to 1 point connection is possible except for Pb and Rec. Pb and Rec can be added to any active 1 point to 1 point connections or simply tied to 1 single stream.

\begin{TabularC}{9}
\hline
\rowcolor{lightgray}{\bf Function \textbackslash{} I\+F }&\PBS\centering {\bf I2\+S }&\PBS\centering {\bf P\+C\+M }&\PBS\centering {\bf U\+S\+B }&\PBS\centering {\bf Analog (Mic / Spkr) }&\PBS\centering {\bf Mdm Tx }&\PBS\centering {\bf Mdm Rx }&\PBS\centering {\bf Pb }&\PBS\centering {\bf Rec  }\\\cline{1-9}
Gain (Set / Get) &\PBS\centering &\PBS\centering &\PBS\centering &\PBS\centering X($\ast$) &\PBS\centering X &\PBS\centering X &\PBS\centering X &\PBS\centering \\\cline{1-9}
Mute / Unmute &\PBS\centering &\PBS\centering &\PBS\centering &\PBS\centering X($\ast$) &\PBS\centering X &\PBS\centering X &\PBS\centering X &\PBS\centering \\\cline{1-9}
N\+S &\PBS\centering &\PBS\centering &\PBS\centering &\PBS\centering &\PBS\centering X &\PBS\centering &\PBS\centering &\PBS\centering \\\cline{1-9}
E\+C &\PBS\centering &\PBS\centering &\PBS\centering &\PBS\centering &\PBS\centering X &\PBS\centering &\PBS\centering &\PBS\centering \\\cline{1-9}
I\+I\+R &\PBS\centering &\PBS\centering &\PBS\centering &\PBS\centering &\PBS\centering X &\PBS\centering X &\PBS\centering &\PBS\centering \\\cline{1-9}
F\+I\+R &\PBS\centering &\PBS\centering &\PBS\centering &\PBS\centering &\PBS\centering X &\PBS\centering X &\PBS\centering &\PBS\centering \\\cline{1-9}
A\+G\+C &\PBS\centering Only Tx &\PBS\centering Only Tx &\PBS\centering Only Tx &\PBS\centering Only Spkr($\ast$) &\PBS\centering X &\PBS\centering X &\PBS\centering &\PBS\centering \\\cline{1-9}
\end{TabularC}
($\ast$) A\+R platform only

N\+S = Noise Suppressor ~\newline
 E\+C = Echo Canceller ~\newline
 I\+I\+R = Infinite Impulse Response filter ~\newline
 F\+I\+R = Finite Impulse Response filter ~\newline
 A\+G\+C = Automatic Gain Control

{\bfseries Playback and recording}

File playback supports these file types\+:
\begin{DoxyItemize}
\item .wav
\item .amr
\item .awb
\end{DoxyItemize}

File recording is only supported for {\bfseries }.wav file types.

One single file playback and one single file recording can be activated simultaneously.

Continuous playback of one file doesn\textquotesingle{}t work well.

{\bfseries P\+C\+M insertion/extraction}

Only supports 16-\/bit sampling resolution on the A\+R7 platform.





Copyright (C) Sierra Wireless Inc. Use of this work is subject to license. \hypertarget{platformConstraintsSecStorage}{}\subsection{Secure Storage}\label{platformConstraintsSecStorage}
The A\+R755x, A\+R8652, and W\+P8548 platforms maximum space available for secure storage is 256 K\+B.





Copyright (C) Sierra Wireless Inc. Use of this work is subject to license. \hypertarget{yoctoMain}{}\section{Yocto Info}\label{yoctoMain}
This is advanced-\/level info about Legato and Yocto\+:

\hyperlink{yoctoLegato}{Yocto Legato Builds} ~\newline
 \hyperlink{yoctoInstallNotes}{Yocto Install Notes} ~\newline






Copyright (C) Sierra Wireless Inc. Use of this work is subject to license. \hypertarget{yoctoLegato}{}\subsection{Yocto Legato Builds}\label{yoctoLegato}
The Yocto Project is an open source collaboration project that provides templates, tools and methods to help you create custom Linux-\/based systems for embedded products regardless of the hardware architecture.

Legato is compatible with Yocto 1.\+6.\+1\hypertarget{yocto_legato_getstartedYoctoLinux_yoctoDirectories}{}\subsubsection{Yocto Linux Directories}\label{yocto_legato_getstartedYoctoLinux_yoctoDirectories}
Untar the file {\bfseries Legato-\/\+Dist-\/\+Source-\/mdm9x15-\/15.\+05.\+tar.\+bz2} to a directory\+: 
\begin{DoxyCode}
$ cd <some directory>
$ tar xvjf Legato-Linux-Dist-mdm9x15-15.05.tar.bz2
\end{DoxyCode}


These file/directories will be extracted\+:
\begin{DoxyItemize}
\item {\bfseries linux-\/yocto-\/3.\+4.\+git} -\/ Linux kernel -\/ 14.\+1.\+0.\+Beta.\+rc2
\item {\bfseries Legato-\/\+Yocto1\+\_\+6} -\/ Sierra Wireless specific layers
\item {\bfseries meta-\/swi} -\/ Hardware adaptations
\item {\bfseries meta-\/swi-\/extras} -\/ Build scripts and proprietary code
\end{DoxyItemize}

Yocto uses a layered model for its build system. In Legato, device-\/specific layers are {\ttfamily meta-\/swi-\/}\mbox{[}target\mbox{]} and {\ttfamily meta-\/swi-\/bin}.

Yocto 1.\+6.\+1 and poky environment documentation is available at \href{https://www.yoctoproject.org/documentation/archived?keys=&field_version_tid=45}{\tt Yocto docs}\hypertarget{yocto_legato_getstartedYoctoLinux_prebuiltYoctoImages}{}\subsubsection{Pre-\/built Yocto Image}\label{yocto_legato_getstartedYoctoLinux_prebuiltYoctoImages}
The prebuilt directory contains a kernel and root file system that can be built using the command below. It\textquotesingle{}s the same as what\textquotesingle{}s flashed on the device, and can be used to return to a known state.


\begin{DoxyItemize}
\item {\bfseries kernel} -\/ kernel image
\item {\bfseries rootfs} -\/ smallish root file system ready for basic development.
\end{DoxyItemize}

Instructions on how to flash these images to the device are given in the file R\+E\+A\+D\+M\+E.\+bsp.\hypertarget{yocto_legato_getstartedYoctoLinux_rebuildYoctoImages}{}\subsubsection{Rebuild Yocto Image}\label{yocto_legato_getstartedYoctoLinux_rebuildYoctoImages}
Instructions for rebuilding the image are contained in the file Legato-\/\+Yocto1\+\_\+6/meta-\/swi-\/extras/\+R\+E\+A\+D\+M\+E.\+build

Running the build script with no arguments will print out the help message. There is also a file called {\ttfamily stdbuild.\+sh} to build the images with common options. This can take a long time the first time you run it. Once the build is complete, you\textquotesingle{}ll find new images in the directory\+: 
\begin{DoxyCode}
../build/tmp/deploy/images 
\end{DoxyCode}


Links are created at the end of the build to point to the latest kernel (kernel) and root file system (rootfs) \begin{DoxyWarning}{Warning}
You can’t rebuild Yocto images on Ubuntu 13.\+10.
\end{DoxyWarning}
\hypertarget{yocto_legato_getstartedYoctoLinux_flashYoctoImagesLin}{}\subsubsection{Linux Flash Yocto}\label{yocto_legato_getstartedYoctoLinux_flashYoctoImagesLin}
You can flash the Yocto images on Linux;, the device must be in {\itshape fastboot} mode. From the shell prompt on the device run\+: 
\begin{DoxyCode}
root@swi-mdm9x15:~# sys\_reboot bootloader 
\end{DoxyCode}


After a few seconds, the device will enumerate as an Android Bootloader Device. You can test this on the development P\+C with\+:


\begin{DoxyCode}
 dave@devpc$ fastboot devices
MDM9615
\end{DoxyCode}


Then erase and rewrite the kernel and root filesystem partitions using the fastboot command (the partitions are called kernel and rootfs)\+:


\begin{DoxyCode}
dave@devpc $ fastboot erase kernel
...
dave@devpc $ fastboot erase rootfs
...
dave@devpc $ fastboot flash kernel kernel
...
dave@devpc $ fastboot flash rootfs rootfs
...
dave@devpc $ fastboot reboot
rebooting...
finished. total time: 0.001s
\end{DoxyCode}


\begin{DoxyNote}{Note}
You have to use the micro-\/\+U\+S\+B connection for fastboot
\end{DoxyNote}
\hypertarget{yocto_legato_getstartedYoctoLinux_custYoctoImages}{}\subsubsection{Custom Yocto Image}\label{yocto_legato_getstartedYoctoLinux_custYoctoImages}
Because the Legato image is a Yocto-\/compliant B\+S\+P, there are many options for customizing.

Here\textquotesingle{}s the easy way to add some packages to the rootfs\+:


\begin{DoxyCode}
meta-swi-extras/meta-swi-bin/recipes/images/9615-cdp-sierra-image.inc 
\end{DoxyCode}


For more complex customizations, refer to the Yocto documentation.





Copyright (C) Sierra Wireless Inc. Use of this work is subject to license. \hypertarget{yoctoInstallNotes}{}\subsection{Yocto Install Notes}\label{yoctoInstallNotes}
Here are some helpful tips for Yocto Linux working with Sierra Wireless devices.\hypertarget{yocto_install_notes_getstartedInstallNotes_yoctoNoPwd}{}\subsubsection{Log on Without Password}\label{yocto_install_notes_getstartedInstallNotes_yoctoNoPwd}
To log onto the target through the serial interface, log on as {\bfseries root} with no password. This allows manual changes to other interfaces (e.\+g., U\+S\+B if you want to run C\+D\+C-\/\+E\+C\+M).\hypertarget{yocto_install_notes_getstartedInstallNotes_yoctoCDC}{}\subsubsection{C\+D\+C-\/\+E\+C\+M for I\+P vs Ethernet}\label{yocto_install_notes_getstartedInstallNotes_yoctoCDC}
Using C\+D\+C-\/\+E\+C\+M for I\+P connections will cause the M\+A\+C address to change every reboot because the kernel allocates software-\/defined M\+A\+C addresses to interfaces. This may cause new connections to be detected by your host every time you reboot the device. See \hyperlink{getstartedConfigIP}{Configure I\+P Address} and \hyperlink{toolsTarget_setNet}{set\+Net}\hypertarget{yocto_install_notes_getstartedInstallNotes_yoctoPrefIP}{}\subsubsection{Setup Preferred I\+P}\label{yocto_install_notes_getstartedInstallNotes_yoctoPrefIP}
Either boot with an Ethernet cable plugged in and let the device obtain an I\+P address using D\+H\+C\+P, or enable C\+D\+C-\/\+E\+C\+M and the micro-\/\+U\+S\+B cable. You can then determine the I\+P address using {\ttfamily ifconfig} on the console or provide a fixed I\+P from your D\+H\+C\+P server. Then use {\ttfamily ssh} and {\ttfamily scp} to access the device and transfer files.\hypertarget{yocto_install_notes_getstartedInstallNotes_yoctoDefUSB}{}\subsubsection{Change Default U\+S\+B Classes}\label{yocto_install_notes_getstartedInstallNotes_yoctoDefUSB}
During the boot sequence, startup scripts run the file {\ttfamily /etc/legato/usbsetup} that enumerates the U\+S\+B types listed in {\ttfamily /etc/legato/usbmode}.

You can easily override default types by creating your own {\ttfamily usbmode} file in {\ttfamily /mnt/flash/startup/usb}. The new file will take precedence over the old one. By default, the device will enumerate an E\+C\+M port, an A\+C\+M port for A\+T commands and U\+S\+B audio.\hypertarget{yocto_install_notes_getstartedInstallNotes_yoctoStaticIP}{}\subsubsection{Setup Target Static I\+P}\label{yocto_install_notes_getstartedInstallNotes_yoctoStaticIP}
You can configure your target and development P\+C so it doesn\textquotesingle{}t prompt for a password, run\+: {\bfseries bin/configtargetssh} 

It\textquotesingle{}ll look like this\+: \begin{DoxyVerb}dave@devbox:~/legato$ bin/configtargetssh 192.168.1.2
Generating new key pair... --->>> LEAVE THE PASSPHRASE EMPTY <<<---
Generating public/private rsa key pair.
Enter passphrase (empty for no passphrase):
Enter same passphrase again:
Your identification has been saved in /home/dave/.ssh/id_rsa.legatoTarget.
Your public key has been saved in /home/dave/.ssh/id_rsa.legatoTarget.pub.
The key fingerprint is:
af:c5:e4:8c:7d:53:b4:a4:72:c8:67:f3:88:9c:3f:67 dave's key for Legato
target devices.
The key's randomart image is:
+--[ RSA 2048]----+
|                 |
|                 |
|              o  |
|         . . + . |
|        S = * o  |
|         X B =   |
|        . @ + .  |
|         o o..E  |
|        .   .+   |
+-----------------+
/home/dave/.ssh/known_hosts updated.
Original contents retained as /home/dave/.ssh/known_hosts.old
Connecting to the target...  (ENTER TARGET'S ROOT PASSWORD WHENEVER
PROMPTED FOR A PASSWORD)
(press ENTER if your target doesn't have a root password)
The authenticity of host '192.168.1.2 (192.168.1.2)' can't be established.
RSA key fingerprint is f2:2f:66:a5:14:20:bd:46:8f:a2:02:b3:99:6f:72:24.
Are you sure you want to continue connecting (yes/no)? yes
Warning: Permanently added '192.168.1.2' (RSA) to the list of known hosts.
root@192.168.1.2's password:
dave@devpc:~/legato$
\end{DoxyVerb}
\hypertarget{yocto_install_notes_getstartedInstallNotes_yoctoReadWrite}{}\subsubsection{Read/\+Write rootfs}\label{yocto_install_notes_getstartedInstallNotes_yoctoReadWrite}
It\textquotesingle{}s convenient right now to enable r/w to rootfs, but it will be moving to a read-\/only model soon. If you change the rootfs (e.\+g., in /etc, /lib, /bin), you may cripple your device and have to re-\/install your rootfs.\hypertarget{yocto_install_notes_getstartedInstallNotes_yoctoNoDown}{}\subsubsection{Disable New Package Download}\label{yocto_install_notes_getstartedInstallNotes_yoctoNoDown}
The Yocto build defaults to disable new package downloads.

Enable downloading new packages over the Internet, modify\+:

{\ttfamily $<$build-\/dir$>$/conf/local}.conf value {\ttfamily B\+B\+\_\+\+N\+O\+\_\+\+N\+E\+T\+W\+O\+R\+K} to \char`\"{}0\char`\"{} . The file is only present after a build has run once.\hypertarget{yocto_install_notes_yoctoBSPTestReport}{}\subsubsection{Setup using B\+S\+P\+Test\+Report.\+xls}\label{yocto_install_notes_yoctoBSPTestReport}
The {\ttfamily B\+S\+P\+Test\+Report.\+xls} spreadsheet is used for instructions and reports to help set up different B\+S\+P features to test. Start with the Contents sheet.\hypertarget{yocto_install_notes_getstartedInstallNotes_yoctoNFSRoot}{}\subsubsection{Use nfs root}\label{yocto_install_notes_getstartedInstallNotes_yoctoNFSRoot}
The file {\bfseries meta-\/swi/meta-\/swi-\/mdm9x15/conf/machine/qcom-\/mdm9615.\+conf} contains examples if you need to include a lot of extra packages in your rootfs or if you build the debug version. See R\+E\+A\+D\+M\+E.\+bsp for details. Remember to untar the generated rootfs to an exported file system. The file pointed to by the rootfs link contains the most recent root file system. There’s more info available at \href{https://www.kernel.org/doc/Documentation/filesystems/nfs/nfsroot.txt}{\tt kernel docs}. 



Copyright (C) Sierra Wireless Inc. Use of this work is subject to license. 