This section contains detailed info about Legato\textquotesingle{}s Definition Files. The {\ttfamily def} files define reusable external interfaces and internal content for system, application, and component files that are used as input to build tools.

The Definition Files\textquotesingle{} \hyperlink{defFilesOverview}{Overview} has high-\/level info.



 \hyperlink{interfaceDefLang}{Interface Definition Language} ~\newline
 \hyperlink{defFilesAdef}{Application Definition .adef} ~\newline
 \hyperlink{defFilesCdef}{Component Definition .cdef} ~\newline
 \hyperlink{defFilesSdef}{System Definition .sdef}





Copyright (C) Sierra Wireless Inc. Use of this work is subject to license. \hypertarget{defFilesOverview}{}\section{Overview}\label{defFilesOverview}
Legato definition files are used as input to the \hyperlink{buildTools}{Build} tools {\ttfamily mksys}, {\ttfamily mkapp}, {\ttfamily mkcomp}, and {\ttfamily mkexe} (known collectively as the mk tools). The definition files, together with \hyperlink{interfaceDefLang}{Interface Definition Language}, make it easier to develop component-\/based software, automatically generating a lot of boiler-\/plate code.

Interfaces are defined using {\ttfamily .api} files. Implementations of those interfaces are constructed as reusable components. Components can be combined into apps. Apps can be installed and run on target devices and integrated into systems of inter-\/communicating applications deployed together to target devices in a single step.

All def files use a \hyperlink{defFilesFormat}{Common File Format}.\hypertarget{def_files_overview_defFilesOverview_adef}{}\subsection{.\+adef Files}\label{def_files_overview_defFilesOverview_adef}
Application definition {\ttfamily }.adef files are used to specify the external interfaces and internal content of applications that can be built, installed, and run on target devices.

{\ttfamily .adef} files can also be used to override some settings of components without having to change those components themselves, thereby making the components more reusable.

Each application has a {\ttfamily .adef} file that describes\+:
\begin{DoxyItemize}
\item which executables should be built from which components
\item additional files from the build system to be included in the app
\item processes to be started (by running what executables with what command-\/line arguments and environment variables) when the app starts
\item if the app should automatically start when the target device boots
\item files (or other file system objects) from the target root file system to be available to the app at runtime
\item limits on the app at runtime (e.\+g., cpu limits, memory limits, etc.)
\item overrides for memory pool sizes and configuration settings for components in the app (future)
\item I\+P\+C bindings between components within the app
\item I\+P\+C interfaces will be made visible to other apps
\end{DoxyItemize}\hypertarget{def_files_overview_defFilesOverview_cdef}{}\subsection{.\+cdef Files}\label{def_files_overview_defFilesOverview_cdef}
Component definition {\ttfamily }.cdef files are used to specify the external interfaces and internal content of reusable software components.

Each component has a single {\ttfamily Component.\+cdef} file that defines\+:
\begin{DoxyItemize}
\item source code files used to build the component
\item files from the build system to be included in any app that includes the component
\item files on the target to be accessible to any app that includes the component
\item I\+P\+C interfaces the component implements
\item other components needed by this component
\item memory pool sizes
\item configuration settings for the component (future)
\end{DoxyItemize}\hypertarget{def_files_overview_defFilesOverview_sdef}{}\subsection{.\+sdef Files}\label{def_files_overview_defFilesOverview_sdef}
System definition {\ttfamily }.sdef files are used to interconnect applications with each other and with the target\textquotesingle{}s runtime environment (e.\+g., hardware devices).

.sdef files can also override some app settings.

An {\ttfamily .sdef} file defines a system of one or more applications that can be deployed to target devices in the field. Each {\ttfamily .sdef} file lists\+:
\begin{DoxyItemize}
\item apps are to be installed on the device
\item I\+P\+C connections permitted between apps
\item overrides for limits, configuration settings, and I\+P\+C bindings within apps
\end{DoxyItemize}\hypertarget{def_files_overview_defFilesOverview_searchPaths}{}\subsection{Build Tool Search Paths}\label{def_files_overview_defFilesOverview_searchPaths}
Two are used\+:
\begin{DoxyItemize}
\item {\bfseries source} search path -\/ lists file system directories where the build tools look for source code files and components.
\item {\bfseries interface} search path -\/ lists file system directories where the build tools look for interface definition files (.h files and .api files).
\end{DoxyItemize}

The default search path “.\+” is the current working directory where the build tool ran.

Search paths can be changed through mk tools command-\/line options\+:
\begin{DoxyItemize}
\item \textquotesingle{}-\/s\textquotesingle{} sets the source search path
\item ‘-\/i’ sets the interface search path
\end{DoxyItemize}


\begin{DoxyCode}
$ mksys packageTracker.sdef -i ~/work/interfaces -s ~/work/components
\end{DoxyCode}
\hypertarget{def_files_overview_defFilesOverview_overriding}{}\subsection{Precedence and Overriding}\label{def_files_overview_defFilesOverview_overriding}
Settings in a {\ttfamily .sdef} file override settings from {\itshape both} the {\ttfamily .adef} and {\ttfamily Component.\+cdef}, while the {\ttfamily .adef} overrides settings in the {\ttfamily Component.\+cdef}. This increases reusability by allowing an integrator to override a component\textquotesingle{}s or app\textquotesingle{}s settings without having to change that component or app.

Most configuration settings can be overridden on the target device at runtime (although, some won\textquotesingle{}t take effect until the app is restarted).\hypertarget{def_files_overview_defFilesOverview_unsandboxedApps}{}\subsection{Unsandboxed App Limits}\label{def_files_overview_defFilesOverview_unsandboxedApps}
Although it\textquotesingle{}s better for apps to be sandboxed, apps aren\textquotesingle{}t limited to running inside a sandbox. We call these {\itshape unsandboxed} apps.

Even though unsanboxed apps aren\textquotesingle{}t constrained the same way sandboxed apps are constrained, the Supervisor will still enforce \hyperlink{def_files_adef_defFilesAdef_cpuShare}{C\+P\+U Share} and \hyperlink{def_files_adef_defFilesAdef_processMaxFileBytes}{Max File Bytes} limits.





Copyright (C) Sierra Wireless Inc. Use of this work is subject to license. \hypertarget{defFilesFormat}{}\subsection{Common File Format}\label{defFilesFormat}
All of the definition files have the same format. They\textquotesingle{}re divided into sections. Each section type has a set of permitted content, which could be simple, numerical or text values, named items, or other sections (subsections).\hypertarget{def_files_format_defFileFormatSections}{}\subsubsection{Sections}\label{def_files_format_defFileFormatSections}
A section always starts with a section name followed by a colon (\textquotesingle{}\+:\textquotesingle{}).


\begin{DoxyCode}
faultAction: restart

maxFileSystemBytes: 200K
\end{DoxyCode}


Sections that can have multiple content items always have curly braces around their content.


\begin{DoxyCode}
sources:
\{
    helloWorld.c
    myClass.cpp
\}
\end{DoxyCode}


All sections are optional; they can be empty or omitted.

Sections can be in any order, multiple occurrences are permitted, and some sections have sub-\/sections.


\begin{DoxyCode}
requires:
\{
    file:
    \{
        /dev/ttyS0  /dev/uart
    \}

    file:
    \{
        /tmp/foo  /tmp/
    \}

    dir:
    \{
    \}
\}
\end{DoxyCode}
\hypertarget{def_files_format_defFileFormatNamedItems}{}\subsubsection{Named Items}\label{def_files_format_defFileFormatNamedItems}
A named item always begins with the item name followed by an equals sign (\textquotesingle{}=\textquotesingle{}).


\begin{DoxyCode}
myPool = 45
\end{DoxyCode}


If a named item can have multiple content items, then it must have curly braces around its content.


\begin{DoxyCode}
myExe = \{ myComponent otherComponent \}
\end{DoxyCode}
\hypertarget{def_files_format_defFileFormatNumberValues}{}\subsubsection{Numerical Values}\label{def_files_format_defFileFormatNumberValues}
Integers are required in some sections.


\begin{DoxyCode}
maxFileDescriptors: 100
\end{DoxyCode}


Numbers can be expressed in multiples of 1024 by adding the \textquotesingle{}K\textquotesingle{} suffix.


\begin{DoxyCode}
maxFileSystemBytes: 120K
\end{DoxyCode}
\hypertarget{def_files_format_defFileFormatTextValues}{}\subsubsection{Text Values}\label{def_files_format_defFileFormatTextValues}
Some sections contain text-\/based values. The format of these varies depending on the type of section.\hypertarget{def_files_format_defFileFormatUsingEnvVarsIn}{}\subsubsection{Environment Variables}\label{def_files_format_defFileFormatUsingEnvVarsIn}
It\textquotesingle{}s possible to use environment variables inside of {\ttfamily .sdef}, {\ttfamily .adef}, and {\ttfamily Component.\+cdef} files.

This works the same way as in shell scripts, by prefixing the environment variable name with a dollar sign (\$)\+:


\begin{DoxyCode}
requires:
\{
    api:
    \{
        $BUILD\_ROOT/interfaces/httpdCtrl.api
    \}
\}
\end{DoxyCode}


If necessary, the variable name can also be enclosed in curly braces.


\begin{DoxyCode}
requires:
\{
    api:
    \{
        $\{PRODUCT\_FAMILY\}\_interfaces/httpdCtrl.api
    \}
\}
\end{DoxyCode}


Some useful environment variables\+:


\begin{DoxyItemize}
\item {\ttfamily L\+E\+G\+A\+T\+O\+\_\+\+R\+O\+O\+T} = path to where the Legato framework sources are located
\item {\ttfamily T\+A\+R\+G\+E\+T} = build target (e.\+g., ar7, wp7, localhost)
\item {\ttfamily L\+E\+G\+A\+T\+O\+\_\+\+B\+U\+I\+L\+D} = shorthand for {\ttfamily \$\+L\+E\+G\+A\+T\+O\+\_\+\+R\+O\+O\+T/build/\$\+T\+A\+R\+G\+E\+T}
\end{DoxyItemize}\hypertarget{def_files_format_defFileFormatWhitespace}{}\subsubsection{Whitespace}\label{def_files_format_defFileFormatWhitespace}
Tabs, spaces, new-\/lines, carriage-\/returns and comments act as token separators only, and are otherwise all ignored. Consequently, choice of indentation and spacing style is relatively unrestricted.\hypertarget{def_files_format_defFileFormatComments}{}\subsubsection{Comments}\label{def_files_format_defFileFormatComments}
Comments can be included using 
\begin{DoxyCode}
\textcolor{comment}{// C++-style one-line comments.}

\textcolor{comment}{/* or}
\textcolor{comment}{   C-style multi-line comments. */}
\end{DoxyCode}
\hypertarget{def_files_format_defFileFormatSample}{}\subsubsection{Sample}\label{def_files_format_defFileFormatSample}

\begin{DoxyCode}
\textcolor{comment}{/* Component.cdef file for a hypothetical web server component that provides a control API}
\textcolor{comment}{ * (defined in httpdCtrl.api) that allows other apps to start and stop the web server via IPC.}
\textcolor{comment}{ */}

bundles:
\{
    \textcolor{comment}{// Include all the web pages from the build host in the application, and have them appear}
    \textcolor{comment}{// to the application under /var/www.}
    dir:
    \{
        htdocs   /var/www
    \}
\}

provides :     \textcolor{comment}{// Note that space is allowed between the section name and the colon.}
\{
    api:    \{   httpdCtrl.api
            \}
\}

sources: \{httpd.c\}
\end{DoxyCode}






Copyright (C) Sierra Wireless Inc. Use of this work is subject to license. \hypertarget{interfaceDefLang}{}\section{Interface Definition Language}\label{interfaceDefLang}
This topic contains detailed info about how A\+P\+Is can use Legato\textquotesingle{}s interface definition language (I\+D\+L). Legato\textquotesingle{}s I\+D\+L helps apps be written in multiple, different programming languages.





\hyperlink{interfaceDefLangSyntax}{Syntax} ~\newline
 \hyperlink{interfaceDefLangC}{C Language Support} ~\newline






Related info

\hyperlink{buildToolsifgen}{ifgen} ~\newline
 \hyperlink{defFiles}{Definition Files} ~\newline
 \hyperlink{basicAppsIPC}{Use I\+P\+C}\hypertarget{interface_def_lang_interfaceDefLang_overview}{}\subsection{Overview}\label{interface_def_lang_interfaceDefLang_overview}
Legato components can provide A\+P\+Is for other components to use. It can be done conventionally by writing a C header file to define the interface. But a C header file can only be used by components also written in C, and functions defined in a C header file can only be implemented by C code. C compilers won\textquotesingle{}t generate I\+P\+C code, so unless you write your own, your A\+P\+I implementation and its user are forced to run inside the same process. This can severely limit the reusability of components and can force using a programming language not ideally suited to a particular problem domain or developer skillset. It also leaves inter-\/process communication (I\+P\+C) to be implemented manually, which can be time-\/consuming and fraught with bugs and security issues.

To simplify things, Legato has an I\+D\+L, similar to C, that helps define A\+P\+Is so they can be used in multiple, different programming languages.

These I\+D\+L files are called {\bfseries  A\+P\+I definition } ({\ttfamily .api}) files.

They\textquotesingle{}re processed by the \hyperlink{buildToolsifgen}{ifgen} tool that generates function definitions and I\+P\+C code in an implementation language chosen by the component writer. Most of the time, developers don\textquotesingle{}t need to know much about {\ttfamily ifgen}, because it\textquotesingle{}s automatically run by other build tools, as needed.

An A\+P\+I client\+:
\begin{DoxyItemize}
\item import the A\+P\+I into their component (add the {\ttfamily .api} file to the {\ttfamily api\+:} subsection of the {\ttfamily requires\+:} section of the component\textquotesingle{}s {\ttfamily Component.\+cdef} file)
\item include/import the generated code into their program (e.\+g., in C\+: {\ttfamily \#include \char`\"{}interfaces.\+h\char`\"{}})
\item call the functions in the A\+P\+I
\end{DoxyItemize}

This automatically will do I\+P\+C connection opening/closing, message buffer allocation/release, message passing, synchronization between client and server threads/processes, and sandbox security access control.

An A\+P\+I server\+:
\begin{DoxyItemize}
\item export the A\+P\+I from their component (add the {\ttfamily .api} file to the {\ttfamily api\+:} subsection of the {\ttfamily provides\+:} section of the component\textquotesingle{}s {\ttfamily Component.\+cdef} file)
\item include/import the generated code into their program (e.\+g., in C\+: {\ttfamily \#include \char`\"{}interfaces.\+h\char`\"{}})
\item implement the functions in the A\+P\+I
\end{DoxyItemize}

The server\textquotesingle{}s functions are called automatically when the client calls the auto-\/generated client-\/side versions of those functions.





Copyright (C) Sierra Wireless Inc. Use of this work is subject to license. \hypertarget{interfaceDefLangSyntax}{}\subsection{Syntax}\label{interfaceDefLangSyntax}
The A\+P\+I file supports defining functions, events, handlers (callback functions) and user-\/defined types.

\hyperlink{interface_def_lang_syntax_interfaceDefLangSyntax_function}{Functions} are similar to C functions. They can take input and output parameters, and can return a result.

A \hyperlink{interface_def_lang_syntax_interfaceDefLangSyntax_handler}{handler} is a callback function, which can be passed as a parameter to a function and can be registered to be called when \hyperlink{interface_def_lang_syntax_interfaceDefLangSyntax_event}{events} occur.

The A\+P\+I file currently supports a limited number of pre-\/defined types. There is also support for a number of different user-\/defined types.\hypertarget{interface_def_lang_syntax_interfaceDefLangSyntax_types}{}\subsubsection{Type Support}\label{interface_def_lang_syntax_interfaceDefLangSyntax_types}
The A\+P\+I file currently supports a limited number of pre-\/defined types. These are\+:

\begin{DoxyVerb}uint8
uint16
uint32
uint64

int8
int16
int32
int64

string
file
handler
\end{DoxyVerb}


The unsigned and signed integer types are self-\/explanatory. See \hyperlink{interface_def_lang_syntax_interfaceDefLangSyntax_function}{Specifying a Function} for details on the {\ttfamily string} type. The {\ttfamily file} type is used to pass an open file descriptor (fd) as a parameter. This is used for passing an fd between a client and server. The {\ttfamily handler} type is used for passing a handler as a parameter.

The following user-\/defined types are supported\+:


\begin{DoxyItemize}
\item D\+E\+F\+I\+N\+E
\item E\+N\+U\+M
\item B\+I\+T\+M\+A\+P
\item R\+E\+F\+E\+R\+E\+N\+C\+E
\end{DoxyItemize}

Type definitions can also be shared between A\+P\+I files with U\+S\+E\+T\+Y\+P\+E\+S.

Also, all C types defined in the {\ttfamily \hyperlink{legato_8h}{legato.\+h}} file are available. The most commonly used of these is the \hyperlink{le__basics_8h_a1cca095ed6ebab24b57a636382a6c86c}{le\+\_\+result\+\_\+t} type. Support for directly using these C types is a temporary measure and will be removed in a future release.\hypertarget{interface_def_lang_syntax_interfaceDefLangSyntax_typesDefine}{}\paragraph{D\+E\+F\+I\+N\+E}\label{interface_def_lang_syntax_interfaceDefLangSyntax_typesDefine}
A D\+E\+F\+I\+N\+E is specified as\+:

\begin{DoxyVerb}DEFINE <name> = <value>;
\end{DoxyVerb}


The {\ttfamily value} can be a string or an expression evaluated to a numeric value (when the definition is read).\hypertarget{interface_def_lang_syntax_interfaceDefLangSyntax_typesEnum}{}\paragraph{E\+N\+U\+M}\label{interface_def_lang_syntax_interfaceDefLangSyntax_typesEnum}
An E\+N\+U\+M is specified as\+:

\begin{DoxyVerb}ENUM <name>
{
    [<elementList>]
};
\end{DoxyVerb}


The {\ttfamily element\+List} is a comma separated list of elements. The elements should all be uppper-\/case. Element values assigned are internally generated and can\textquotesingle{}t be explicitly given.\hypertarget{interface_def_lang_syntax_interfaceDefLangSyntax_typesBitmask}{}\paragraph{B\+I\+T\+M\+A\+S\+K}\label{interface_def_lang_syntax_interfaceDefLangSyntax_typesBitmask}
A B\+I\+T\+M\+A\+S\+K is a special type of E\+N\+U\+M. It is specified the same way as an E\+N\+U\+M, but the internally generated element values are bit positions, i.\+e. 0x1, 0x2, 0x4, etc.\hypertarget{interface_def_lang_syntax_interfaceDefLangSyntax_typesReference}{}\paragraph{R\+E\+F\+E\+R\+E\+N\+C\+E}\label{interface_def_lang_syntax_interfaceDefLangSyntax_typesReference}
A R\+E\+F\+E\+R\+E\+N\+C\+E is specified as\+:

\begin{DoxyVerb}REFERENCE <name>;
\end{DoxyVerb}


The R\+E\+F\+E\+R\+E\+N\+C\+E is used to define a reference to an object. The object reference is mapped to an opaque reference in C and an object instance or similar in other languages.\hypertarget{interface_def_lang_syntax_interfaceDefLangSyntax_typesUsetypes}{}\paragraph{U\+S\+E\+T\+Y\+P\+E\+S}\label{interface_def_lang_syntax_interfaceDefLangSyntax_typesUsetypes}
You can share type definitions between .api files with U\+S\+E\+T\+Y\+P\+E\+S. This is commonly referred to as importing, although only the type definitions are imported or used. Any code related definitions in a .api file, e.\+g. F\+U\+N\+C\+T\+I\+O\+N, are ignored. These U\+S\+E\+T\+Y\+P\+E\+S can even be nested.

As an example of usage, suppose the files defs.\+api, common.\+api and example.\+api are defined as follows\+:

defs.\+api \begin{DoxyVerb}DEFINE FIVE = 5;
\end{DoxyVerb}


common.\+api \begin{DoxyVerb}USETYPES defs.api;

DEFINE TEN = defs.FIVE + 5;
\end{DoxyVerb}


example.\+api \begin{DoxyVerb}USETYPES common.api;

DEFINE twenty = common.TEN + defs.FIVE + 5;
\end{DoxyVerb}


This example illustrates that nesting causes an implicit U\+S\+E\+T\+Y\+P\+E\+S. Thus, any definitions from defs.\+api, can be used in example.\+api, in the same way as if it had explicitly imported defs.\+api.\hypertarget{interface_def_lang_syntax_interfaceDefLangSyntax_function}{}\subsubsection{Specifying a Function}\label{interface_def_lang_syntax_interfaceDefLangSyntax_function}
A function is specified as\+:

\begin{DoxyVerb}FUNCTION [<returnType>] <name>
(
    [<parameterList>]
);
\end{DoxyVerb}


The {\ttfamily parameter\+List} can contain one or more parameters separated by commas, or can be empty if there are no parameters. These parameters types are supported\+:


\begin{DoxyCode}
<type> <name> [ ( \textcolor{stringliteral}{"IN"} | \textcolor{stringliteral}{"OUT"} ) ] 
\end{DoxyCode}

\begin{DoxyItemize}
\item scalar type
\item defaults to I\+N if a direction is not specified
\end{DoxyItemize}


\begin{DoxyCode}
<type> <name> \textcolor{stringliteral}{"["} [ <minSize> \textcolor{stringliteral}{".."} ] <maxSize> \textcolor{stringliteral}{"]"} \textcolor{stringliteral}{"IN"} 
\end{DoxyCode}

\begin{DoxyItemize}
\item an I\+N array
\item {\ttfamily max\+Size} specifies the maximum number of elements allowed for the array
\item optional {\ttfamily min\+Size} specifies the minimum number of elements required for the array
\end{DoxyItemize}


\begin{DoxyCode}
<type> <name> \textcolor{stringliteral}{"["} <minSize> \textcolor{stringliteral}{"]"} \textcolor{stringliteral}{"OUT"} 
\end{DoxyCode}

\begin{DoxyItemize}
\item an O\+U\+T array
\item array should be large enough to store {\ttfamily min\+Size} elements; if supported by the function implemention, a shorter O\+U\+T array can be used.
\end{DoxyItemize}


\begin{DoxyCode}
\textcolor{stringliteral}{"string"} <name> \textcolor{stringliteral}{"["} [ <minSize> \textcolor{stringliteral}{".."} ] <maxSize> \textcolor{stringliteral}{"]"} \textcolor{stringliteral}{"IN"} 
\end{DoxyCode}

\begin{DoxyItemize}
\item an I\+N string
\item {\ttfamily max\+Size} specifies the maximum string length allowed,
\item optional {\ttfamily min\+Size} specifies the minimum string length required
\item string length is given as number of characters, excluding any terminating characters
\end{DoxyItemize}


\begin{DoxyCode}
\textcolor{stringliteral}{"string"} <name> \textcolor{stringliteral}{"["} <minSize> \textcolor{stringliteral}{"]"} \textcolor{stringliteral}{"OUT"} 
\end{DoxyCode}

\begin{DoxyItemize}
\item an O\+U\+T string
\item string should be large enough to store {\ttfamily min\+Size} characters; if supported by the function implemention, a shorter O\+U\+T string can be used.
\item string length is given as number of characters, excluding any terminating characters
\end{DoxyItemize}

{\ttfamily  \char`\"{}handler\char`\"{} $<$name$>$ "}
\begin{DoxyItemize}
\item a handler (callback) function.
\item see \hyperlink{interface_def_lang_syntax_interfaceDefLangSyntax_handler}{Specifying a Handler} for information on how to declare a handler.
\end{DoxyItemize}

The {\ttfamily return\+Type} is optional, and if specified, can be any type that\textquotesingle{}s not an array, string, or handler.\hypertarget{interface_def_lang_syntax_interfaceDefLangSyntax_event}{}\subsubsection{Specifying an Event}\label{interface_def_lang_syntax_interfaceDefLangSyntax_event}
Do this to specify an event\+:

\begin{DoxyVerb}EVENT <eventType>
(
    <parameterList>
);
\end{DoxyVerb}


The {\ttfamily parameter\+List} can contain one or more parameters separated by commas. It can contain anything that\textquotesingle{}s valid for a function, but it must contain one handler parameter. The parameters are used when registering a handler for the specified event.

See \hyperlink{interface_def_lang_c_interfaceDefLangC_event}{Events in C} for details on the C code generated from the above event definition.\hypertarget{interface_def_lang_syntax_interfaceDefLangSyntax_handler}{}\subsubsection{Specifying a Handler}\label{interface_def_lang_syntax_interfaceDefLangSyntax_handler}
Do this to specify a handler\+:

\begin{DoxyVerb}HANDLER <handlerType>
(
    [<parameterList>]
);
\end{DoxyVerb}


The {\ttfamily parameter\+List} can contain one or more parameters separated by commas, or can be empty if there are no parameters. It can only contain scalar types or string types, as described above for \hyperlink{interface_def_lang_syntax_interfaceDefLangSyntax_function}{Specifying a Function}. All the parameters should be I\+N parameters.

See \hyperlink{interface_def_lang_c_interfaceDefLangC_handler}{Handlers in C} for details on the C code generated from the above handler definition.\hypertarget{interface_def_lang_syntax_interfaceDefLangSyntax_comments}{}\subsubsection{Comments}\label{interface_def_lang_syntax_interfaceDefLangSyntax_comments}
The A\+P\+I file supports both C and C++ comment styles. Comments that use the doxygen formats \begin{DoxyVerb}/** \end{DoxyVerb}
 to start a multi-\/line comment or\begin{DoxyVerb}///< \end{DoxyVerb}
 to start a one line comment receive special processing. Multi-\/line comments at the start of the A\+P\+I file will be copied directly to the start of the appropriate generated files.

Comments given in the function definition will be copied to the appropriate generated files under the following conditions\+:
\begin{DoxyItemize}
\item Multi-\/line comments must start with\begin{DoxyVerb}/** \end{DoxyVerb}

\item Single-\/line comments must start with\begin{DoxyVerb}///< \end{DoxyVerb}

\item In a block of single-\/line comments, each line must start with\begin{DoxyVerb}///< \end{DoxyVerb}
 rather than just the first line. This is different from typical doxygen usage.
\item If the function definition is preceded by a multi-\/line comment then this comment will be copied to the appropriate generated files.
\item If any parameter is followed by a multi-\/line comment or one or more single line comments, then all these comments will be copied to the appropriate generated files.
\end{DoxyItemize}

If an event or handler definition is preceded by a multi-\/line comment, then this comment will be copied to the appropriate generated files, under the same conditions as function definitions.

Any comments provided after an element in an E\+N\+U\+M or B\+I\+T\+M\+A\+S\+K, will be copied to the appropriate generated files, under the same conditions as function parameter comments.\hypertarget{interface_def_lang_syntax_interfaceDefLangSyntax_sample}{}\subsubsection{Sample A\+P\+I}\label{interface_def_lang_syntax_interfaceDefLangSyntax_sample}
Here\textquotesingle{}s the {\ttfamily defn.\+api} file containing just type defintions


\begin{DoxyVerbInclude}
/**
 * Example of nested .api file
 */

DEFINE SIX = 6;
\end{DoxyVerbInclude}


Here\textquotesingle{}s the {\ttfamily common.\+api} file containing just type defintions, and using the types defined in {\ttfamily defn.\+api} 


\begin{DoxyVerbInclude}
/**
 * Common definitions potentially used across multiple .api files
 */

USETYPES defn;


/**
 * Definition example
 */
DEFINE FOUR = 4;

/**
 * Example of using previously DEFINEd symbol within an imported file.
 */
DEFINE TEN = FOUR + defn.SIX;

/**
 * Reference example
 */
REFERENCE OpaqueReference;

/**
 * ENUM example
 */
ENUM EnumExample
{
    ZERO,     ///< first enum
    ONE,      ///< second enum
    TWO,      ///< third enum
    THREE     ///< fourth enum
};

/**
 * BITMASK example
 */
BITMASK BitMaskExample
{
    BIT0,     ///< first
    BIT1,     ///< second
    BIT2,     ///< third
};


\end{DoxyVerbInclude}


Here\textquotesingle{}s the {\ttfamily example.\+api} file containing various definitions, and using the types defined in {\ttfamily defn.\+api} and {\ttfamily common.\+api} 


\begin{DoxyVerbInclude}
/**
 * Example API file
 */

// The .api suffix is optional
USETYPES defn;
USETYPES common.api;


DEFINE TEN = common.FOUR + defn.SIX;
DEFINE TWENTY = TEN + common.TEN;
DEFINE SOME_STRING = "some string";


/**
 * Handler definition
 */
HANDLER TestAHandler
(
    int32 x   ///< First parameter for the handler
              ///< Second comment line
);


/**
 * This event provides an example of an EVENT definition
 */
EVENT TestA
(
    uint32 data,          ///< Used when registering the handler i.e. it is
                          ///< passed into the generated ADD_HANDLER function.
    handler TestAHandler
);


/**
 * Function takes all the possible kinds of parameters, but returns nothing
 */
FUNCTION AllParameters
(
    common.EnumExample a,  ///< first one-line comment
                           ///< second one-line comment
    uint32 b OUT,
    uint32 data[common.TEN] IN,

    uint32 output[TEN] OUT,   ///< some more comments here
    ///< and some comments here as well

    string label [common.TEN..20] IN,
    string response [TWENTY] OUT
    ///< comments on final parameter, first line
    ///< and more comments
);


/**
 * Test file descriptors as IN and OUT parameters
 */
FUNCTION FileTest
(
    file dataFile IN,   ///< file descriptor as IN parameter
    file dataOut OUT    ///< file descriptor as OUT parameter
);

/**
 * Function that takes a handler parameter
 */
FUNCTION int32 UseCallback
(
    uint32 someParm IN,
    handler TestAHandler
);

\end{DoxyVerbInclude}






Copyright (C) Sierra Wireless Inc. Use of this work is subject to license. \hypertarget{interfaceDefLangC}{}\subsection{C Language Support}\label{interfaceDefLangC}
\hypertarget{interface_def_lang_c_interfaceDefLangC_filesGenerated}{}\subsubsection{Files Generated}\label{interface_def_lang_c_interfaceDefLangC_filesGenerated}
For the C language, five files are generated by \hyperlink{buildToolsifgen}{ifgen} from the component interface file\+:


\begin{DoxyItemize}
\item {\bfseries client interface header file (interface.\+h)} -\/ C definitions for the interface. This file should be included by any C source files that want to use the interface.
\item {\bfseries server header file (server.\+h) } -\/ C definitions for the server interface. There is some duplication between this file and the interface file, but it also contains definitions that are not part of the public client interface. This file should be included by any C source files that want to implement the interface.
\item {\bfseries local header file (local.\+h)} -\/ local header file provides common definitions for the client and server implementations. This file should only be included by the client and server implementation files.
\item {\bfseries client implementation file (client.\+c) } -\/ implements all of the interface functions. These functions handle the details of sending messages to the server, and processing the responses.
\item {\bfseries server implementation file (server.\+c)} -\/ implements handlers for all the interface functions. These handlers receive the message from the client side, call the corresponding real implementation of the function, and generate any responses back to the client side.
\end{DoxyItemize}

\begin{DoxyNote}{Note}
Client and server implementation files are provided to support client/server I\+P\+C implementations.
\end{DoxyNote}
\hypertarget{interface_def_lang_c_interfaceDefLangC_handlerParm}{}\subsubsection{Handler Parameters in C}\label{interface_def_lang_c_interfaceDefLangC_handlerParm}
A handler parameter can be used in both events and regular functions. This is how a handler parameter for a function in an interface file is mapped to actual C definitions (based on the \hyperlink{interface_def_lang_c_interfaceDefLangC_sampleAPI}{A\+P\+I File Sample Output}).

\begin{DoxyVerb}FUNCTION int32 UseCallback
(
    uint32 someParm IN,
    handler TestAHandler
);
\end{DoxyVerb}


This results in the function definition\+:

\begin{DoxyVerb}int32_t UseCallback
(
    uint32_t someParm,
    TestAHandlerFunc_t handlerPtr,
    void* contextPtr
);
\end{DoxyVerb}


A {\ttfamily context\+Ptr} parameter has been automatically added to the definition of the function. This {\ttfamily context\+Ptr} will be passed back to the specified handler function when it is invoked. If the {\ttfamily context\+Ptr} is not needed, then N\+U\+L\+L can be used.\hypertarget{interface_def_lang_c_interfaceDefLangC_event}{}\subsubsection{Events in C}\label{interface_def_lang_c_interfaceDefLangC_event}
This is how an event in an interface file is mapped to actual C definitions (based on the \hyperlink{interface_def_lang_c_interfaceDefLangC_sampleAPI}{A\+P\+I File Sample Output}).

\begin{DoxyVerb}EVENT TestA
(
    uint32 data,
    handler TestAHandler
);
\end{DoxyVerb}


This results in one type definition and two function definitions\+:

\begin{DoxyVerb}typedef struct TestAHandler* TestAHandlerRef_t;

TestAHandlerRef_t AddTestAHandler
(
    uint32_t data,
    TestAHandlerFunc_t handlerPtr,
    void* contextPtr
);

void RemoveTestAHandler
(
    TestAHandlerRef_t addHandlerRef
);
\end{DoxyVerb}


The parameters from the event definition are used in the A\+D\+D\+\_\+\+H\+A\+N\+D\+L\+E\+R function. This is used to register the given handler for events. The R\+E\+M\+O\+V\+E\+\_\+\+H\+A\+N\+D\+L\+E\+R function does not take any additional parameters, other than the reference returned by the A\+D\+D\+\_\+\+H\+A\+N\+D\+L\+E\+R function.

A {\ttfamily context\+Ptr} parameter has been automatically added to the definition of the A\+D\+D\+\_\+\+H\+A\+N\+D\+L\+E\+R function. The {\ttfamily context\+Ptr} passed to the A\+D\+D\+\_\+\+H\+A\+N\+D\+L\+E\+R function, will be passed back to the registered handler function. If the {\ttfamily context\+Ptr} is not needed, then N\+U\+L\+L can be used.\hypertarget{interface_def_lang_c_interfaceDefLangC_handler}{}\subsubsection{Handlers in C}\label{interface_def_lang_c_interfaceDefLangC_handler}
This is how a handler in an interface file is mapped to actual C definitions (based on the \hyperlink{interface_def_lang_c_interfaceDefLangC_sampleAPI}{A\+P\+I File Sample Output})\+:

\begin{DoxyVerb}HANDLER TestAHandler
(
    int32 x
);
\end{DoxyVerb}


This results in a type definition for the handler function pointer\+:

\begin{DoxyVerb}typedef void (*TestAHandlerFunc_t)
(
    int32_t x,
    void* contextPtr
);
\end{DoxyVerb}


A {\ttfamily context\+Ptr} parameter has been automatically added to the definition of the handler function pointer. The {\ttfamily context\+Ptr} passed to a regular function or A\+D\+D\+\_\+\+H\+A\+N\+D\+L\+E\+R function (see section \hyperlink{interface_def_lang_c_interfaceDefLangC_handlerParm}{Handler Parameters in C} or \hyperlink{interface_def_lang_c_interfaceDefLangC_event}{Events in C}, respectively) will be passed back to the handler function. If the {\ttfamily context\+Ptr} is not needed, then N\+U\+L\+L can be used.\hypertarget{interface_def_lang_c_interfaceDefLangC_client}{}\subsubsection{Client-\/specific Functions}\label{interface_def_lang_c_interfaceDefLangC_client}
These are client-\/specific functions\+:

\begin{DoxyVerb}void ConnectService
(
    void
);

void DisconnectService
(
    void
);
\end{DoxyVerb}


To use a service, a client must connect to the server using {\ttfamily Connect\+Service()}. This connection is only created for the current thread. Other threads must also call Connect\+Service() to use the service.

For the main thread, {\ttfamily Connect\+Service()} is usually automatically called when the client app is initialized. Disable this by using the .cdef provides \hyperlink{def_files_cdef_defFilesCdef_providesApiManualStart}{\mbox{[}manual-\/start\mbox{]} Option}.

If a client app uses multiple services, multiple {\ttfamily Connect\+Service()} functions need to be called; each function uses an appropriate prefix to distinguish these clients. If multiple client apps are used, each app must be initialized separately, using the appropriate Connect\+Service() function.

The Disconnect\+Service() function closes a connection to the server. It only closes the current thread connection; it must be called separately for each thread using a service. If a thread wants to use the service again, it must call Connect\+Service() to re-\/connect. When the app exits, all connections for all threads are automatically closed. A thread only needs to use Disconnect\+Service() when it wants to disconnect from a service while the app is still running (e.\+g., no longer needs the service so it can conserve resources).\hypertarget{interface_def_lang_c_interfaceDefLangC_server}{}\subsubsection{Server-\/specific Functions}\label{interface_def_lang_c_interfaceDefLangC_server}
These are server-\/specific functions\+:

\begin{DoxyVerb}le_msg_ServiceRef_t GetServiceRef
(
    void
);

le_msg_SessionRef_t GetClientSessionRef
(
    void
);

void AdvertiseService
(
    void
);
\end{DoxyVerb}


To provide a service, the server must advertise the service to any interested clients using {\ttfamily Advertise\+Service()}. This is usually automatically called during the initialization of the server daemon. This can be disabled using the .cdef provides \hyperlink{def_files_cdef_defFilesCdef_providesApiManualStart}{\mbox{[}manual-\/start\mbox{]} Option}.

If a server provides multiple services, multiple {\ttfamily Advertise\+Service()} functions need to be called; each function will have an appropriate prefix to distinguish each service.

The {\ttfamily Get\+Service\+Ref()} function is used to get the server session reference for the current service. It\textquotesingle{}s required if the server uses any of the server-\/specific \hyperlink{c_messaging}{Low-\/\+Level Messaging A\+P\+I} functions for this service.

For example, \hyperlink{le__messaging_8h_a426dfbae396599d80e52902165368907}{le\+\_\+msg\+\_\+\+Add\+Service\+Close\+Handler} can be used by the server to register a close handler whenever a client closes its connection. This may be needed to cleanup client specific data maintained by the server.

{\ttfamily Get\+Client\+Session\+Ref()} function is used to get the client session reference for the current service. This client session is only valid while executing the server-\/side function that implements an interface function. Once this server-\/side function returns, the client session can no longer be retrieved. {\ttfamily Get\+Client\+Session\+Ref()} is needed if the server wants to call any of the client-\/specific \hyperlink{c_messaging}{Low-\/\+Level Messaging A\+P\+I} functions for this service.

For example, \hyperlink{le__messaging_8h_a8c04f9cad0a768b4922a9987df84b65f}{le\+\_\+msg\+\_\+\+Get\+Client\+User\+Id()} can be used by the server to determine the User\+Id of the client using the service, which allows the server to perform any necessary User\+Id based authentication.\hypertarget{interface_def_lang_c_interfaceDefLangC_asyncServer}{}\subsubsection{Asynchronous Server}\label{interface_def_lang_c_interfaceDefLangC_asyncServer}
There are two alternatives to implement the server-\/side functionality.

The default case is where each server-\/side function has the same interface as the client-\/side function. The server-\/side function takes the I\+N parameters, and returns the O\+U\+T parameters and function result when the function exits.

In the async-\/server case, the server-\/side function doesn\textquotesingle{}t necessarily return the O\+U\+T parameters and function result when it exits. Instead, there\textquotesingle{}s a separate {\ttfamily Respond} function for each server-\/side function. The O\+U\+T parameters and function result are returned by passing these values to the {\ttfamily Respond} function. The {\ttfamily Respond} function can be called at any time, normally after the server-\/side function has exited.

Regardless of how the server-\/side functions are implemented, the client-\/side function waits until the O\+U\+T parameters and function result are returned.

The async-\/server functionality is not enabled by default. Enable it by using the .cdef provides \hyperlink{def_files_cdef_defFilesCdef_providesApiAsync}{\mbox{[}async\mbox{]} Option}.\hypertarget{interface_def_lang_c_interfaceDefLangC_sampleAPI}{}\subsubsection{A\+P\+I File Sample Output}\label{interface_def_lang_c_interfaceDefLangC_sampleAPI}
Here\textquotesingle{}s the generated client interface header file for the defn.\+api file from \hyperlink{interface_def_lang_c_interfaceDefLangC_sampleAPI}{A\+P\+I File Sample Output}


\begin{DoxyVerbInclude}
/*
 * ====================== WARNING ======================
 *
 * THE CONTENTS OF THIS FILE HAVE BEEN AUTO-GENERATED.
 * DO NOT MODIFY IN ANY WAY.
 *
 * ====================== WARNING ======================
 */


#ifndef DEFN_H_INCLUDE_GUARD
#define DEFN_H_INCLUDE_GUARD


#include "legato.h"


//--------------------------------------------------------------------------------------------------
/**
 * Example of nested .api file
 */
//--------------------------------------------------------------------------------------------------
#define DEFN_SIX 6


#endif // DEFN_H_INCLUDE_GUARD

\end{DoxyVerbInclude}


Here\textquotesingle{}s the generated client interface header file for the common.\+api file from \hyperlink{interface_def_lang_c_interfaceDefLangC_sampleAPI}{A\+P\+I File Sample Output}


\begin{DoxyVerbInclude}
/*
 * ====================== WARNING ======================
 *
 * THE CONTENTS OF THIS FILE HAVE BEEN AUTO-GENERATED.
 * DO NOT MODIFY IN ANY WAY.
 *
 * ====================== WARNING ======================
 */

/**
 * Common definitions potentially used across multiple .api files
 */

#ifndef COMMON_H_INCLUDE_GUARD
#define COMMON_H_INCLUDE_GUARD


#include "legato.h"

// Interface specific includes
#include "defn_interface.h"



//--------------------------------------------------------------------------------------------------
/**
 * Definition example
 */
//--------------------------------------------------------------------------------------------------
#define COMMON_FOUR 4


//--------------------------------------------------------------------------------------------------
/**
 * Example of using previously DEFINEd symbol within an imported file.
 */
//--------------------------------------------------------------------------------------------------
#define COMMON_TEN 10


//--------------------------------------------------------------------------------------------------
/**
 * Reference example
 */
//--------------------------------------------------------------------------------------------------
typedef struct common_OpaqueReference* common_OpaqueReferenceRef_t;


//--------------------------------------------------------------------------------------------------
/**
 * ENUM example
 */
//--------------------------------------------------------------------------------------------------
typedef enum
{
    COMMON_ZERO,
        ///< first enum

    COMMON_ONE,
        ///< second enum

    COMMON_TWO,
        ///< third enum

    COMMON_THREE
        ///< fourth enum
}
common_EnumExample_t;


//--------------------------------------------------------------------------------------------------
/**
 * BITMASK example
 */
//--------------------------------------------------------------------------------------------------
typedef enum
{
    COMMON_BIT0 = 0x1,
        ///< first

    COMMON_BIT1 = 0x2,
        ///< second

    COMMON_BIT2 = 0x4
        ///< third
}
common_BitMaskExample_t;


#endif // COMMON_H_INCLUDE_GUARD

\end{DoxyVerbInclude}


Here\textquotesingle{}s the generated client interface header file for the example.\+api file from \hyperlink{interface_def_lang_c_interfaceDefLangC_sampleAPI}{A\+P\+I File Sample Output}


\begin{DoxyVerbInclude}
/*
 * ====================== WARNING ======================
 *
 * THE CONTENTS OF THIS FILE HAVE BEEN AUTO-GENERATED.
 * DO NOT MODIFY IN ANY WAY.
 *
 * ====================== WARNING ======================
 */

/**
 * Example API file
 */

#ifndef EXAMPLE_H_INCLUDE_GUARD
#define EXAMPLE_H_INCLUDE_GUARD


#include "legato.h"

// Interface specific includes
#include "defn_interface.h"
#include "common_interface.h"


//--------------------------------------------------------------------------------------------------
/**
 * Connect the client to the service
 */
//--------------------------------------------------------------------------------------------------
void example_ConnectService
(
    void
);

//--------------------------------------------------------------------------------------------------
/**
 * Disconnect the client from the service
 */
//--------------------------------------------------------------------------------------------------
void example_DisconnectService
(
    void
);


//--------------------------------------------------------------------------------------------------

//--------------------------------------------------------------------------------------------------
#define EXAMPLE_TEN 10


//--------------------------------------------------------------------------------------------------

//--------------------------------------------------------------------------------------------------
#define EXAMPLE_TWENTY 20


//--------------------------------------------------------------------------------------------------

//--------------------------------------------------------------------------------------------------
#define EXAMPLE_SOME_STRING "some string"


//--------------------------------------------------------------------------------------------------
/**
 * Reference type used by Add/Remove functions for EVENT 'example_TestA'
 */
//--------------------------------------------------------------------------------------------------
typedef struct example_TestAHandler* example_TestAHandlerRef_t;


//--------------------------------------------------------------------------------------------------
/**
 * Handler definition
 *
 * @param x
 *        First parameter for the handler
 *        Second comment line
 * @param contextPtr
 */
//--------------------------------------------------------------------------------------------------
typedef void (*example_TestAHandlerFunc_t)
(
    int32_t x,
    void* contextPtr
);

//--------------------------------------------------------------------------------------------------
/**
 * Add handler function for EVENT 'example_TestA'
 *
 * This event provides an example of an EVENT definition
 */
//--------------------------------------------------------------------------------------------------
example_TestAHandlerRef_t example_AddTestAHandler
(
    uint32_t data,
        ///< [IN]
        ///< Used when registering the handler i.e. it is
        ///< passed into the generated ADD_HANDLER function.

    example_TestAHandlerFunc_t handlerPtr,
        ///< [IN]

    void* contextPtr
        ///< [IN]
);

//--------------------------------------------------------------------------------------------------
/**
 * Remove handler function for EVENT 'example_TestA'
 */
//--------------------------------------------------------------------------------------------------
void example_RemoveTestAHandler
(
    example_TestAHandlerRef_t addHandlerRef
        ///< [IN]
);

//--------------------------------------------------------------------------------------------------
/**
 * Function takes all the possible kinds of parameters, but returns nothing
 */
//--------------------------------------------------------------------------------------------------
void example_AllParameters
(
    common_EnumExample_t a,
        ///< [IN]
        ///< first one-line comment
        ///< second one-line comment

    uint32_t* bPtr,
        ///< [OUT]

    const uint32_t* dataPtr,
        ///< [IN]

    size_t dataNumElements,
        ///< [IN]

    uint32_t* outputPtr,
        ///< [OUT]
        ///< some more comments here
        ///< and some comments here as well

    size_t* outputNumElementsPtr,
        ///< [INOUT]

    const char* label,
        ///< [IN]

    char* response,
        ///< [OUT]
        ///< comments on final parameter, first line
        ///< and more comments

    size_t responseNumElements
        ///< [IN]
);

//--------------------------------------------------------------------------------------------------
/**
 * Test file descriptors as IN and OUT parameters
 */
//--------------------------------------------------------------------------------------------------
void example_FileTest
(
    int dataFile,
        ///< [IN]
        ///< file descriptor as IN parameter

    int* dataOutPtr
        ///< [OUT]
        ///< file descriptor as OUT parameter
);

//--------------------------------------------------------------------------------------------------
/**
 * Function that takes a handler parameter
 */
//--------------------------------------------------------------------------------------------------
int32_t example_UseCallback
(
    uint32_t someParm,
        ///< [IN]

    example_TestAHandlerFunc_t handlerPtr,
        ///< [IN]

    void* contextPtr
        ///< [IN]
);


#endif // EXAMPLE_H_INCLUDE_GUARD

\end{DoxyVerbInclude}






Copyright (C) Sierra Wireless Inc. Use of this work is subject to license. \hypertarget{defFilesAdef}{}\section{Application Definition .adef}\label{defFilesAdef}
{\ttfamily .adef} files can contain these sections\+:\hypertarget{def_files_adef_defFilesAdef_bindings}{}\subsection{Bindings}\label{def_files_adef_defFilesAdef_bindings}
Bindings allow client-\/side I\+P\+C A\+P\+I interfaces (listed in the {\itshape requires} sections of Component.\+cdef files) to be bound to server-\/side interfaces (listed in the {\itshape provides} sections of Component.\+cdef files).

This gives direct control over how I\+P\+C interfaces are interconnected so reusable components and/or apps can be bound (wired) together to form a working system.

Bindings can be between two interfaces inside the same app (internal binding), or can be between a client-\/side interface in an app and a server-\/side interface provided by another app (or non-\/app user).

Interface instances inside the app use the executable name, component name, and the service instance name separated by a period (‘.\+’).

The client interface always appears first with an arrow ({\ttfamily  -\/$>$ }) to separate the client interface from the server interface.

Here\textquotesingle{}s a code sample binding {\ttfamily client\+Exe.\+client\+Component.\+client\+Interface} to {\ttfamily server\+Exe.\+server\+Component.\+server\+Interface}\+:

\begin{DoxyVerb}bindings:
{
    clientExe.clientComponent.clientInterface -> serverExe.serverComponent.serverInterface
}
\end{DoxyVerb}


External services provided by other apps use the app name and the service name, separated by a period (\textquotesingle{}.\textquotesingle{})\+:

\begin{DoxyVerb}bindings:
{
    clientExe.clientComponent.clientInterface -> otherApp.serverInterface
}
\end{DoxyVerb}


To bind to a service provided by a non-\/app user, the app name is replaced by a user name inside angle brackets\+:

\begin{DoxyVerb}bindings:
{
    clientExe.clientComponent.clientInterface -> <root>.serverInterface
}
\end{DoxyVerb}


If client-\/side code was built using the {\ttfamily ifgen} tool or \hyperlink{buildToolsmkcomp}{mkcomp} or \hyperlink{buildToolsmkexe}{mkexe} with requires or bundles sections, the tool reading your .adef file won\textquotesingle{}t know about the client-\/side interface. Use a special work-\/around to bind those interfaces\+:

\begin{DoxyVerb}bindings:
{
    *.le_data -> dataConnectionService.le_data
}
\end{DoxyVerb}


This would bind any unknown client-\/side {\ttfamily le\+\_\+data} interfaces in the current app to the {\ttfamily le\+\_\+data} server-\/side interface on the {\ttfamily data\+Connection\+Service} app.\hypertarget{def_files_adef_defFilesAdef_bundles}{}\subsection{Bundles}\label{def_files_adef_defFilesAdef_bundles}
Lists additional files or directories to be copied from the build host into the app so they’re available to the app at runtime (e.\+g., audio files, web pages, executable scripts or programs built using some external build system).

\begin{DoxyVerb}bundles:
{
    file:
    {
        // Include the web server executable (built using some other build tool) in the app's /bin.
        [x] 3rdParty/webServer/bin/wwwServ  /bin/

        // Put the company logo into the app's /var/www/ for read-only access by the web server.
        images/abcCorpLogo.jpg  /var/www/

        // Make the appropriate welcome page for the product appear at /var/www/index.html.
        webContent/$PRODUCT_ID/welcome.html  /var/www/index.html

        // Create a file to record persistent custom audio messages into.
        [w] audio/defaultMessage.wav  /usr/share/sounds/customMessage.wav
    }

    dir:
    {
        // Recursively bundle the directory containing all the audio files into the app.
        // It will appear to the app read-only under /usr/share/sounds/.
        audio   /usr/share/sounds
    }
}
\end{DoxyVerb}


Three things need to be specified for each file or directory\+:
\begin{DoxyItemize}
\item access permissions
\item build system path
\item target path
\end{DoxyItemize}

{\bfseries Access permissions} -\/ any combination of one or more of the following letters, enclosed in square brackets\+:
\begin{DoxyItemize}
\item r = readable
\item w = writeable
\item x = executable
\end{DoxyItemize}

If permissions values are not specified, then read-\/only (\mbox{[}r\mbox{]}) is the default.

Directories always have executable permission set so they can be traversed. Setting the {\ttfamily \mbox{[}x\mbox{]}} permission in the {\ttfamily dir\+:} subsection causes the files under the directory to be made executable.

Setting {\ttfamily \mbox{[}w\mbox{]}} in the {\ttfamily dir\+:} subsection causes all files under that directory to be writeable, but the directory itself will not be writeable.

\begin{DoxyNote}{Note}
Directories in the persistent (flash) file system are never made writeable because the on-\/target flash file system does not support usage quotas (yet).
\end{DoxyNote}
{\bfseries Build system path} -\/ file system path on the build P\+C where the file is located at build time.

The path can be relative to the directory where the {\ttfamily }.adef file is located.

\begin{DoxyNote}{Note}
Environment variables can be used inside these paths.
\end{DoxyNote}
{\bfseries Target path} -\/ file system path on the target where the file will appear at runtime.

It\textquotesingle{}s an absolute path inside the app\textquotesingle{}s sandbox file system.

If the path ends with \textquotesingle{}/\textquotesingle{}, it means the directory path where the source object (file or directory) will be copied. The destination object will have the same name as the source object.

If the path doesn\textquotesingle{}t end in a \textquotesingle{}/\textquotesingle{}, it\textquotesingle{}s a full destination object path. The destination object could have a different name than the source object.

\begin{DoxyNote}{Note}
If the app is running unsandboxed, the bundled files and directories can be found in their installation location under {\ttfamily /opt/legato/apps/xxxx}, where xxxx is replaced by the app name.
\end{DoxyNote}
{\bfseries Quoting Paths}

File paths can be enclosed in quotation marks (either single \textquotesingle{} or double "). This is required when the file path contains spaces or comment start sequences \begin{DoxyVerb}"//" or  "/*"
\end{DoxyVerb}


{\bfseries File Ownership and Set-\/\+U\+I\+D Bits}

When the app is installed on a target\+: ~\newline

\begin{DoxyItemize}
\item the owner and group are set to {\ttfamily root} on all files in the app.
\item the {\ttfamily setuid} bit is cleared on everything in the app.
\end{DoxyItemize}

{\bfseries Overriding .cdef}

If a file is bundled by a component\textquotesingle{}s .cdef file and another file is bundled to the same target path by the .adef, then the file specified by the .adef will be bundled into the app.

If a directory is bundled by a component\textquotesingle{}s .cdef file and another directory is bundled to the same target path by the .adef, then the two directories will be merged into the bundle, where conflicts are resolved by omitting the file from the directory specified by the .cdef.\hypertarget{def_files_adef_defFilesAdef_cpuShare}{}\subsection{C\+P\+U Share}\label{def_files_adef_defFilesAdef_cpuShare}
Specifies the relative cpu share for an app.

Cpu share calculates the cpu percentage for a process relative to all other processes in the system. New cgroups and processes default value of {\bfseries 1024} if not otherwise configured. The actual percentage of the cpu allocated to a process is calculated like this\+:

(share value of process) / (sum of shares from all processes contending for the cpu)

All processes within a cgroup share the available cpu percentage share for that cgroup like this\+:


\begin{DoxyItemize}
\item cgroup\+A is configured with the default share value, 1024.
\item cgroup\+B is configured with 512 as its share value.
\item cgroup\+C is configured with 2048 as its share value.
\item cgroup\+A has one process running.
\item cgroup\+B has two processes running.
\item cgroup\+C has one process running.
\end{DoxyItemize}

This assumes all processes in cgroup\+A, cgroup\+B and cgroup\+C are running and not blocked waiting for an I/\+O or timer event, and another system process is also running.

Sum of all shares (including the one system process) is 1024 + 512 + 2048 + 1024 = 4608

The process in cgroup\+A will get 1024/4608 = 22\% of the cpu. The two processes in cgroup\+B will share 512/4608 = 11\% of the cpu, each process getting 5.\+5\%. The process in cgroup\+C will get 2048/4608 = 44\% of the cpu. The system process will get 1024/4608 = 22\% of the cpu.

\begin{DoxyVerb}cpuShare: 512
\end{DoxyVerb}


Although apps aren\textquotesingle{}t limited to running inside a sandbox (i.\+e., unsandboxed apps), the Supervisor still enforces limits. Ensure {\ttfamily cpu\+Share} values are set high enough for unsanboxed apps. Also see \hyperlink{def_files_adef_defFilesAdef_processMaxFileBytes}{Max File Bytes}.\hypertarget{def_files_adef_defFilesAdef_executables}{}\subsection{Executables}\label{def_files_adef_defFilesAdef_executables}
Lists executables to be constructed and moved to the {\ttfamily bin} directory inside the app.

An executable’s content is specified as a list of components.

\begin{DoxyVerb}executables:
{
    myExe = ( myComponent otherComponent )
}
\end{DoxyVerb}


Each component included in an executable will be built and linked into the executable and its {\ttfamily C\+O\+M\+P\+O\+N\+E\+N\+T\+\_\+\+I\+N\+I\+T} function will be run at process start-\/up. The component’s runtime files (shared libraries, source code, etc.) will be packaged inside the app to be installed on the target.

The mechanisms used to construct executables and components depend on the type of content and the target device.

C and C++ files will be compiled and linked using the appropriate compiler tool chain based on the target.

(In the future) Java code will be compiled to Java bytecode, and interpreted code, such as Python code will be simply copied into the app.

\begin{DoxyNote}{Note}
There may be incompatibilities between components that prevent them from being included in the same executable (e.\+g., Java component and C components in the same executable together). The build tools will advise if there\textquotesingle{}s a problem.
\end{DoxyNote}
\hypertarget{def_files_adef_defFilesAdef_groups}{}\subsection{Groups}\label{def_files_adef_defFilesAdef_groups}
Add an app\textquotesingle{}s user to groups on the target system.

\begin{DoxyVerb}groups:
{
    www-data
    modem
}
\end{DoxyVerb}
\hypertarget{def_files_adef_defFilesAdef_maxFileSystemBytes}{}\subsection{Max File System Bytes}\label{def_files_adef_defFilesAdef_maxFileSystemBytes}
Specifies the maximum amount of R\+A\+M that can be consumed by an app\textquotesingle{}s temporary (volatile) file system at runtime.

Default is {\bfseries 128\+K} 

\begin{DoxyVerb}maxFileSystemBytes: 120K
\end{DoxyVerb}


\begin{DoxyNote}{Note}
The file system size will also be limited by the {\ttfamily max\+Memory\+Bytes} limit.
\end{DoxyNote}
\hypertarget{def_files_adef_defFilesAdef_maxMemoryBytes}{}\subsection{Max Memory Bytes}\label{def_files_adef_defFilesAdef_maxMemoryBytes}
Specifies the maximum amount of memory (in bytes) that all processes in an app can share. Default is {\bfseries 40960\+K} (40\+M\+B).

\begin{DoxyVerb}maxMemoryBytes: 1000K
\end{DoxyVerb}


\begin{DoxyNote}{Note}
Will be rounded to the nearest memory page boundary.
\end{DoxyNote}
\hypertarget{def_files_adef_defFilesAdef_maxMQueueBytes}{}\subsection{Max M\+Queue Bytes}\label{def_files_adef_defFilesAdef_maxMQueueBytes}
Specifies the maximum number of bytes that can be allocated for P\+O\+S\+I\+X M\+Queues. Default is {\bfseries 512}.

\begin{DoxyVerb}maxMQueueBytes: 16K
\end{DoxyVerb}
\hypertarget{def_files_adef_defFilesAdef_maxQueuedSignals}{}\subsection{Max Queued Signals}\label{def_files_adef_defFilesAdef_maxQueuedSignals}
Specifies the maximum number of signals that can be waiting for delivery to processes in the app.

This limit will only be enforced when using {\ttfamily sigqueue()} to send a signal. Signals sent using {\ttfamily kill()} are limited to at most one of each type of signal anyway.

Default is {\bfseries 100}.

\begin{DoxyVerb}maxQueuedSignals: 200
\end{DoxyVerb}
\hypertarget{def_files_adef_defFilesAdef_maxThreads}{}\subsection{Max Threads}\label{def_files_adef_defFilesAdef_maxThreads}
Specifies the maximum number of threads allowed to run at one time\+: an integer number.

\begin{DoxyNote}{Note}
A single-\/threaded process (a running program that doesn\textquotesingle{}t start any threads other than the one running the {\ttfamily main()} function) counts as one thread.
\end{DoxyNote}
If {\ttfamily fork()} calls or {\ttfamily pthread\+\_\+create()} calls start failing with error code {\ttfamily E\+A\+G\+A\+I\+N} (seen in strace output as {\ttfamily clone()} system calls), then you are probably running into this limit.

Default is {\bfseries 20}.

\begin{DoxyVerb}maxThreads: 4
\end{DoxyVerb}
\hypertarget{def_files_adef_defFilesAdef_maxSecureStorageBytes}{}\subsection{Max Secure Storage Bytes}\label{def_files_adef_defFilesAdef_maxSecureStorageBytes}
Specifies the maximum number of bytes that can be stored in Secure Storage.

Default is {\bfseries 8\+K}.

\begin{DoxyVerb}maxSecureStorageBytes: 4K
\end{DoxyVerb}
\hypertarget{def_files_adef_defFilesAdef_pools}{}\subsection{Pools}\label{def_files_adef_defFilesAdef_pools}
\begin{DoxyWarning}{Warning}
This feature not yet implemented.
\end{DoxyWarning}
Sets the number of memory pool blocks in a \hyperlink{c_memory}{memory pool} in a given component in a given process.

This overrides any setting for the same pool in the \hyperlink{def_files_cdef_defFilesCdef_pools}{pools\+: section of the Component.cdef file}.


\begin{DoxyCode}
pools:
\{
    myProc.myComponent.myPool = 100
\}
\end{DoxyCode}
\hypertarget{def_files_adef_defFilesAdef_process}{}\subsection{Processes}\label{def_files_adef_defFilesAdef_process}
A {\ttfamily processes} section specifies processes to run when the app is started including environment variables, command-\/line arguments, limits, and fault handling actions.

The {\ttfamily processes} section is divided into subsections.

If different processes have different variables, they must be in separate {\ttfamily processes\+:} sections.

\begin{DoxyVerb}processes:
{
    // Start up these processes when the app starts
    run:
    {
        myProc1 = ( myExe --foo -b 43 )
        myProc2 = ( myExe –bar --b 92 )
        ( myExe2 "Hello, world." )  // If no proc name is specified, uses the exe name by default.
    }

    // Env var settings (name = value) for all processes in this section.
    envVars:
    {
        LE_LOG_LEVEL = DEBUG
    }

    priority: medium    // Starting (and maximum) scheduling priority.
                        // Process can only lower its priority from here.

    maxCoreDumpFileBytes: 100K  // Maximum size of core dump files.
    maxFileBytes: 50K           // Files are not allowed to grow bigger than this.
    maxLockedMemoryBytes: 32K   // Can't mlock() more than this many bytes.
    maxFileDescriptors: 20      // Can't have more than this number of FDs open at the same time.
}

processes:
{
    run:
    {
        ( realTimeExe )
    }

    priority: rt10   // Allow real-time scheduling (max priority 10) for processes in this section.

    /*-- Exception handling policy for processes in this section. --*/
    faultAction: restart   // Restart the process if it fails.
}
\end{DoxyVerb}
\hypertarget{def_files_adef_defFilesAdef_processRun}{}\subsubsection{Run}\label{def_files_adef_defFilesAdef_processRun}
Names a process to be started by the Supervisor when the app is started. Also specifies executable and command-\/line arguments.


\begin{DoxyCode}
run:
\{
    myProc1 = ( myExe --foo -b 43 )
\}
\end{DoxyCode}


Process name and command-\/line arguments are optional.


\begin{DoxyCode}
run:
\{
    ( myexe )
\}
\end{DoxyCode}


If the process name is not specified, then the process will be given the same name as the executable it\textquotesingle{}s running (e.\+g. {\ttfamily myexe})

A given executable can be launched multiple times. 
\begin{DoxyCode}
run:
\{
    process1 = ( myexe )
    process2 = ( myexe )
\}
\end{DoxyCode}


Command-\/line arguments passed to the process when started can appear after the executable name. 
\begin{DoxyCode}
run:
\{
    ( myexe --foo )
\}
\end{DoxyCode}



\begin{DoxyCode}
run:
\{
    ( myexe --bar  \textcolor{comment}{// Note that the command-line can be broken into multiple lines for readability.}
            --toto ) \textcolor{comment}{// And it can be commented too.}
\}
\end{DoxyCode}


Executable names can be the ones listed in the app’s {\ttfamily executables\+:} section. They can also be the names of files that are bundled into the app with {\ttfamily \mbox{[}x\mbox{]}} (executable) permission.

Quotation marks (either single {\bfseries \textquotesingle{}} or double {\bfseries "}) can be used if white-\/space (spaces, tabs, {\ttfamily //}, etc.) is needed inside a command-\/line argument, or if an empty argument is needed (\char`\"{}\char`\"{}).\hypertarget{def_files_adef_defFilesAdef_processEnvVars}{}\subsubsection{Env Vars}\label{def_files_adef_defFilesAdef_processEnvVars}
Environment variables appear as name = value pairs. First is the environment variable name and second is the variable value.

Enclose the value in quotation marks (either single \textquotesingle{} or double ") if white-\/space is required\+:


\begin{DoxyCode}
envVars:
\{
    LE\_LOG\_LEVEL = DEBUG
    MESSAGE = \textcolor{stringliteral}{"The quick brown fox jumped over the lazy dog."}
\}
\end{DoxyCode}
\hypertarget{def_files_adef_defFilesAdef_processFaultAction}{}\subsubsection{Fault Action}\label{def_files_adef_defFilesAdef_processFaultAction}
Specifies the action the Supervisor should take when the process terminates with a non-\/zero exit code or because of an un-\/caught signal (e.\+g., S\+I\+G\+S\+E\+G\+V, S\+I\+G\+B\+U\+S, S\+I\+G\+K\+I\+L\+L).

Possible values are\+:


\begin{DoxyItemize}
\item {\ttfamily ignore} -\/ Supervisor just logs a warning message; no further action taken.
\item {\ttfamily restart} -\/ log a critical message and restart the process.
\item {\ttfamily restart\+App} -\/ log a critical message and restart the entire app.
\item {\ttfamily stop\+App} -\/ log a critical message and terminate the entire app (send all processes the S\+I\+G\+T\+E\+R\+M signal, followed shortly by S\+I\+G\+K\+I\+L\+L).
\item {\ttfamily reboot} -\/ log an emergency message and reboot the system.
\end{DoxyItemize}

Default is {\bfseries ignore}.


\begin{DoxyCode}
faultAction: restart
\end{DoxyCode}
\hypertarget{def_files_adef_defFilesAdef_processPriority}{}\subsubsection{Priority}\label{def_files_adef_defFilesAdef_processPriority}
Specifies the starting (and maximum) scheduling priority. A running app process can only lower its priority from this point. Once it has lowered its priority, it can\textquotesingle{}t raise it again (e.\+g., if the process starts at medium priority and reduces to low priority, it can\textquotesingle{}t go back to medium priority). The default is {\bfseries medium}.

Values\+:
\begin{DoxyItemize}
\item {\bfseries idle} -\/ for low priority processes that get C\+P\+U time only if no other processes are waiting.
\item {\bfseries low}, {\bfseries medium}, {\bfseries high} -\/ intended for normal processes that contend for the C\+P\+U. Processes with these priorities don\textquotesingle{}t preempt each other, but their priorities affect how they\textquotesingle{}re inserted into the scheduling queue (high to low).
\item {\bfseries rt1} to {\bfseries rt32} -\/ intended for (soft) realtime processes. A higher realtime priority will preempt a lower realtime priority (ie. rt2 would preempt rt1). Processes with any realtime priority will preempt processes with high, medium, low and idle priorities.
\end{DoxyItemize}

\begin{DoxyWarning}{Warning}
Processes with realtime priorities preempt the Legato framework processes. Ensure that your design lets realtime processes relinquish the C\+P\+U appropriately.
\end{DoxyWarning}

\begin{DoxyCode}
priority: medium
\end{DoxyCode}
\hypertarget{def_files_adef_defFilesAdef_processMaxCoreDumpFileBytes}{}\subsubsection{Max Core Dump File Bytes}\label{def_files_adef_defFilesAdef_processMaxCoreDumpFileBytes}
Specifies the maximum size (in bytes) of core dump files that can be generated by processes.

Default is {\bfseries 8\+K} 

\begin{DoxyNote}{Note}
Core dump file size is limited by \hyperlink{def_files_adef_defFilesAdef_processMaxFileBytes}{Max File Bytes}. If the the core file is generated in the sandbox\textquotesingle{}s temporary runtime file system, it \textquotesingle{}ll also be limited by \hyperlink{def_files_adef_defFilesAdef_maxFileSystemBytes}{Max File System Bytes}.
\end{DoxyNote}

\begin{DoxyCode}
maxCoreDumpFileBytes: 100K
maxFileBytes: 100K
maxFileSystemBytes: 200K
\end{DoxyCode}
\hypertarget{def_files_adef_defFilesAdef_processMaxFileBytes}{}\subsubsection{Max File Bytes}\label{def_files_adef_defFilesAdef_processMaxFileBytes}
Specifies the maximum size processes can make files. The {\itshape K} suffix permits specifying in kilobytes (multiples of 1024 bytes).

Default is {\bfseries 88\+K} 


\begin{DoxyCode}
maxFileBytes: 200K
\end{DoxyCode}


Exceeding this limit results in a {\ttfamily S\+I\+G\+X\+F\+S\+Z} signal sent to the process. By default, this kills the process, but it can be blocked or caught to receive an error with {\ttfamily errno} set to {\ttfamily E\+F\+B\+I\+G}.

Although apps aren\textquotesingle{}t limited to running inside a sandbox (i.\+e., unsandboxed apps), the Supervisor still enforces limits. Ensure {\ttfamily max\+File\+Bytes} values are set high enough for unsanboxed apps. Also see \hyperlink{def_files_adef_defFilesAdef_cpuShare}{C\+P\+U Share}.

\begin{DoxyNote}{Note}
If the file is in the sandbox\textquotesingle{}s temporary runtime file system, the file size will also be limited by \hyperlink{def_files_adef_defFilesAdef_maxFileSystemBytes}{Max File System Bytes} and \hyperlink{def_files_adef_defFilesAdef_maxMemoryBytes}{Max Memory Bytes}. ~\newline

\end{DoxyNote}
\hypertarget{def_files_adef_defFilesAdef_processMaxFileDescriptors}{}\subsubsection{Max File Descriptors}\label{def_files_adef_defFilesAdef_processMaxFileDescriptors}
Specifies the maximum number of file descriptors a process can have open at one time.

Default is {\bfseries 256} 


\begin{DoxyCode}
maxFileDescriptors: 100
\end{DoxyCode}
\hypertarget{def_files_adef_defFilesAdef_processMaxLockedMemoryBytes}{}\subsubsection{Max Locked Memory Bytes}\label{def_files_adef_defFilesAdef_processMaxLockedMemoryBytes}
Specifies the maximum bytes of memory the process can lock into physical R\+A\+M (e.\+g., using {\ttfamily mlock()} ).

Can\textquotesingle{}t be higher than \hyperlink{def_files_adef_defFilesAdef_maxMemoryBytes}{Max Memory Bytes}.

Default is {\bfseries 8\+K} 


\begin{DoxyCode}
maxLockedMemoryBytes: 100K
\end{DoxyCode}


\begin{DoxyNote}{Note}
Also limits the maximum number of bytes of shared memory the app can lock into memory using {\ttfamily shmctl()}.
\end{DoxyNote}
\hypertarget{def_files_adef_defFilesAdef_processWatchdogAction}{}\subsubsection{Watchdog Action}\label{def_files_adef_defFilesAdef_processWatchdogAction}
Specifies the action the Supervisor should take when a process subscribed to the watchdog service fails to kick the watchdog before it expires. Possible values are the same as in \hyperlink{def_files_adef_defFilesAdef_processFaultAction}{Fault Action} as well as\+:

{\ttfamily stop} -\/ Supervisor terminates the process if it\textquotesingle{}s still running. If a watchdog action has not been supplied, the default action will restart the process.\hypertarget{def_files_adef_defFilesAdef_processWatchdogTimeout}{}\subsubsection{Watchdog Timeout}\label{def_files_adef_defFilesAdef_processWatchdogTimeout}
Specifies the timeout length (in milliseconds) for watchdogs called by processes in the enclosing processes section.

Once a process has called {\ttfamily \hyperlink{le__wdog__interface_8h_a43d5abc5c44309942efbe7b9c25d811f}{le\+\_\+wdog\+\_\+\+Kick()}}, it\textquotesingle{}s registered with the software watchdog service. If it then fails to call {\ttfamily \hyperlink{le__wdog__interface_8h_a43d5abc5c44309942efbe7b9c25d811f}{le\+\_\+wdog\+\_\+\+Kick()}} within the given timeout, the Supervisor is notified will take the action specified in \hyperlink{def_files_adef_defFilesAdef_processWatchdogAction}{Watchdog Action}

\begin{DoxyVerb}watchdogTimeout: 10000 // Must call le_wdog_Kick() at least every 10 seconds.
\end{DoxyVerb}


A special value \char`\"{}never\char`\"{} is permitted for this section. If used, the watchdog will never time out.

\begin{DoxyVerb}watchdogTimeout: never  // Disable watchdog for these processes.
\end{DoxyVerb}
\hypertarget{def_files_adef_defFilesAdef_extern}{}\subsection{Extern}\label{def_files_adef_defFilesAdef_extern}
Declares that an I\+P\+C A\+P\+I interface is available for binding via \hyperlink{def_files_sdef_defFilesSdef_bindings}{.sdef file\textquotesingle{}s bindings section} or \hyperlink{def_files_adef_defFilesAdef_bindings}{.adef file\textquotesingle{}s bindings section}.

\begin{DoxyVerb}extern:
{
    exeName.componentName.interfaceName
}
\end{DoxyVerb}


Internal interface instances are identified by the executable, component, and interface names. The different interface instance identifier parts are separated by a period (‘.\+’).

Outside the app, the interface will be identified using the app name and the interface name ({\ttfamily app\+Name.\+interface\+Name}), thereby hiding from other apps the details of what executable and component within the app implements the interface.

To have other apps see an interface with a different name, add {\ttfamily  different\+Name = } in front of the interface specification\+:

\begin{DoxyVerb}extern:
{
    differentName = exeName.componentName.interfaceName
}
\end{DoxyVerb}


The interface will then appear outside the app as {\ttfamily app\+Name.\+different\+Name}.\hypertarget{def_files_adef_defFilesAdef_externServer}{}\subsubsection{Server-\/\+Side Example}\label{def_files_adef_defFilesAdef_externServer}
This code sample declares the {\ttfamily temperature} interface on the {\ttfamily thermistor} component in the {\ttfamily sensor} executable should be made into a bindable external server-\/side interface (service) called {\ttfamily space\+Temp\+:} 

\begin{DoxyVerb}extern:
{
    spaceTemp = sensor.thermistor.temperature
}
\end{DoxyVerb}


If the app was called {\itshape thermometer}, then other apps could bind their client-\/side interfaces to {\ttfamily thermometer.\+space\+Temp} to receive temperature readings from this sensor app.\hypertarget{def_files_adef_defFilesAdef_externClient}{}\subsubsection{Client-\/\+Side Example}\label{def_files_adef_defFilesAdef_externClient}
This code sample declares that the {\ttfamily temperature} interface on the {\ttfamily temp\+Input} component in the {\ttfamily controller} executable should be made into a bindable external interface\+:

\begin{DoxyVerb}extern:
{
    controller.tempInput.temperature
}
\end{DoxyVerb}


If this app were called thermostat, then to outsiders (e.\+g., in a {\ttfamily }.sdef file or another app\textquotesingle{}s {\ttfamily }.adef file), the interface would be called {\ttfamily thermostat.\+temperature}.\hypertarget{def_files_adef_defFilesAdef_requires}{}\subsection{Requires}\label{def_files_adef_defFilesAdef_requires}
The {\ttfamily requires\+:} section specifies things the app needs from its runtime environment.

It can contain various subsections.\hypertarget{def_files_adef_defFilesAdef_requiresConfigTree}{}\subsubsection{Config Tree}\label{def_files_adef_defFilesAdef_requiresConfigTree}
Declares the app requires access to a specified configuration tree. Each app has its own configuration tree named after the app. There\textquotesingle{}s also a system configuration tree that contains privileged system data.

By default, an app only has read access to its own configuration tree.

Apps can be granted read access or both read and write access to trees using an optional access permission specifier\+:


\begin{DoxyItemize}
\item {\ttfamily \mbox{[}r\mbox{]}} -\/ grant read-\/only access
\item {\ttfamily \mbox{[}w\mbox{]}} -\/ grant read and write access
\end{DoxyItemize}

If access is granted to a tree, but the access mode (\mbox{[}r\mbox{]} or \mbox{[}w\mbox{]}) is not specified, only read permission will be granted.

A special tree name \char`\"{}.\char`\"{} can be used to refer to the app\textquotesingle{}s own configuration data tree.

\begin{DoxyVerb}requires:
{
    configTree:
    {
        [w] .       // I need write access to my configuration data.
        otherApp    // I need read access to another app's configuration data.
    }
}
\end{DoxyVerb}


\begin{DoxyWarning}{Warning}
Because config data can be saved to flash storage, granting write access to a config tree can make it possible for the app to wear out your flash device. Granting an untrusted app write access to another app\textquotesingle{}s config settings creates a security hole, because it makes it possible for the untrusted app to interfere with the other app\textquotesingle{}s operation. It\textquotesingle{}s especially dangerous to grant write access to the system tree, because it\textquotesingle{}s possible to completely compromise the target device. Even read access may be dangerous if any kind of security key, including P\+I\+N and P\+U\+K codes, are kept in the system tree. Don\textquotesingle{}t give apps direct access to the {\ttfamily system tree}.
\end{DoxyWarning}
\hypertarget{def_files_adef_defFilesAdef_requiresDir}{}\subsubsection{Dir}\label{def_files_adef_defFilesAdef_requiresDir}
Used to declare directories located on the target outside of the app that are to be made accessible to the app.

The location inside the app\textquotesingle{}s sandbox where the directory will appear is also specified.

Things listed here are expected to be found on the target at runtime. They are not copied into the app at build time; they are made accessible to the app inside of its sandbox at runtime.

Each entry consists of two file system paths\+:


\begin{DoxyItemize}
\item The {\bfseries first} path is the path to the directory {\bfseries outside} of the app. This must be an absolute path (beginning with \char`\"{}/\char`\"{}) and can never end in a \char`\"{}/\char`\"{}.
\item The {\bfseries second} path is the absolute path {\bfseries inside} the app’s sandbox where the directory will appear at runtime.
\end{DoxyItemize}

Paths can be enclosed in quotation marks (either single \textquotesingle{} or double "). This is required when it contains spaces or character sequences that would start comments.

If the destination path ends in a \char`\"{}/\char`\"{}, the name from the source is appended to it. Otherwise, only the destination path is used. That means {\ttfamily /foo/bar} /baz will appear as {\ttfamily /baz} inside the sandbox, and {\ttfamily /foo/bar} /baz/ will appear as {\ttfamily /baz/bar} inside the sandbox.

\begin{DoxyVerb}requires:
{
    dir:
    {
        // I need access to /proc for debugging.
        /proc   /

        // For now, I want access to all executables and libraries in /bin and /lib.
        // Later I'll remove this and replace with just the files I really need in the field.
        // Also, I don't want to hide the stuff that the tools automatically bundle into my app's
        // /bin and /lib for me, so I'll make the root file system's /bin and /lib accessible as
        // my app's /usr/bin and /usr/lib.
        /bin    /usr/bin
        /lib    /usr/lib
    }
}
\end{DoxyVerb}


\begin{DoxyNote}{Note}
Even though the directory appears in the app\textquotesingle{}s sandbox doesn\textquotesingle{}t mean the app can access it. The directory permissions settings must also allow it. File permissions (both D\+A\+C and M\+A\+C) and ownership (group and user) on the original files in the target system remain in effect inside the sandbox.
\end{DoxyNote}
\begin{DoxyWarning}{Warning}
Any time anything is made accessible from inside an app sandbox, the security risks must be considered carefully. Ask yourself if access to the object can be exploited by the app (or a hacker who has broken into the app) to access sensitive information or launch a denial-\/of-\/service attack on other apps within the target device or other devices connected to the target device?
\end{DoxyWarning}
\begin{DoxyNote}{Note}
It\textquotesingle{}s not possible to put anything inside of a directory that was mapped into the app from outside of the sandbox. If you {\itshape require} {\ttfamily /bin} to appear at {\ttfamily /usr/bin}, you can\textquotesingle{}t then {\itshape bundle} a file into {\ttfamily /usr/bin} or {\itshape require} something to appear in {\ttfamily /usr/bin}; that would have an effect on the contents of the /bin directory outside of the app.
\end{DoxyNote}
\hypertarget{def_files_adef_defFilesAdef_requiresFile}{}\subsubsection{File}\label{def_files_adef_defFilesAdef_requiresFile}
Declares\+:
\begin{DoxyItemize}
\item specific files located on the target outside of the app, but made accessible to the app.
\item location inside the app\textquotesingle{}s sandbox where the file will appear.
\end{DoxyItemize}

Things listed in {\ttfamily requires} are expected to be found on the target at runtime. They\textquotesingle{}re not copied into the app at build time; they are made accessible to the app inside of its sandbox at runtime.

Each entry consists of two file system paths\+:


\begin{DoxyItemize}
\item path to the object in the file system outside of the app, which must be an absolute path (beginning with ‘/’).
\item absolute file system path inside the app’s sandbox where the object will appear at runtime.
\end{DoxyItemize}

A file path can be enclosed in quotation marks (either single \textquotesingle{} or double "). This is required when it contains spaces or character sequences that would start comments.

The first path can\textquotesingle{}t end in a \textquotesingle{}/\textquotesingle{}.

If the second path ends in a \textquotesingle{}/\textquotesingle{}, then it\textquotesingle{}s specifying the directory where the object appears, and the object has the same name inside the sandbox as it has outside the sandbox.

\begin{DoxyVerb}requires:
{
    file:
    {
        // I get character stream input from outside via a named pipe (read-only)
        /var/run/someNamedPipe  /var/run/

        // I need to be able to play back audio files installed in /usr/local/share/audio.
        "/usr/local/share/audio/error message.wav" /usr/share/audio/
        '/usr/local/share/audio/success message.wav' /usr/share/audio/
    }
}
\end{DoxyVerb}


\begin{DoxyNote}{Note}
Even though the file system object appears in the app\textquotesingle{}s sandbox it still needs permissions settings on the file. File permissions (both D\+A\+C and M\+A\+C) and ownership (group and user) on the original file in the target system remain in effect inside the sandbox.
\end{DoxyNote}
It\textquotesingle{}s also possible to give the object a different names inside and outside of the sandbox by adding a name to the end of the second path.

\begin{DoxyVerb}requires:
{
    file:
    {
        // Program uses /var/run/someNamedPipe which it calls /var/run/externalPipe.
        /var/run/someNamedPipe  /var/run/externalPipe
    }
}
\end{DoxyVerb}


\begin{DoxyWarning}{Warning}
When something is accessible from inside an app sandbox, there are potential security risks (e.\+g., access to the object could be exploited by the app, or hacker, to access sensitive information or launch a denial-\/of-\/service attack on other apps within the target device or other devices connected to the target device).
\end{DoxyWarning}
\hypertarget{def_files_adef_defFilesAdef_requiresDevice}{}\subsubsection{Device}\label{def_files_adef_defFilesAdef_requiresDevice}
Declares\+:
\begin{DoxyItemize}
\item device files that reside on the target outside of the app, but made accessible to the app.
\item location inside the app\textquotesingle{}s sandbox where the file will appear.
\item access permissions the app is given to the device file.
\end{DoxyItemize}

Things listed in {\ttfamily requires} are expected to be found on the target at runtime. They\textquotesingle{}re not copied into the app at build time; they are made accessible to the app inside of its sandbox at runtime.

Each entry consists of two file system paths and a set of optional access permissions\+:


\begin{DoxyItemize}
\item access permissions, readable (\mbox{[}r\mbox{]}) and/or writeable (\mbox{[}w\mbox{]}). Executable is not allowed on device files. If permission values are not specified, then read-\/only (\mbox{[}r\mbox{]}) is the default.
\item path to the object in the file system outside of the app, which must be an absolute path (beginning with ‘/’). This must be a path to a valid character or block device file.
\item absolute file system path inside the app’s sandbox where the object will appear at runtime.
\end{DoxyItemize}

A file path can be enclosed in quotation marks (either single \textquotesingle{} or double "). This is required when it contains spaces or character sequences that would start comments.

The first path can\textquotesingle{}t end in a \textquotesingle{}/\textquotesingle{}.

If the second path ends in a \textquotesingle{}/\textquotesingle{}, then it\textquotesingle{}s specifying the directory where the object appears, and the object has the same name inside the sandbox as it has outside the sandbox.

\begin{DoxyVerb}requires:
{
    device:
    {
        // I get read-only access to the SPI port.
        [r]     /dev/sierra_spi   /dev/sierra_spi

        // I get read-only access to the NMEA port.
                /dev/nmea         /dev/nmea

        // I get read and write access to the I2C port.
        [rw]    /dev/sierra_i2c   /dev/
    }
}
\end{DoxyVerb}


Note that if a hot-\/plug device is unplugged and plugged back in, the app must be restarted before it can access the device.

It\textquotesingle{}s also possible to give the object a different names inside and outside of the sandbox by adding a name to the end of the second path.

\begin{DoxyVerb}requires:
{
    device:
    {
        /dev/ttyS0  /dev/port1     // Program uses /dev/port1, but UART0 is called /dev/ttyS0.
    }
}
\end{DoxyVerb}


\begin{DoxyWarning}{Warning}
When something is accessible from inside an app sandbox, there are potential security risks (e.\+g., access to the object could be exploited by the app, or hacker, to access sensitive information or launch a denial-\/of-\/service attack on other apps within the target device or other devices connected to the target device).

This section is experimental. Future releases of may not support this section.
\end{DoxyWarning}
\hypertarget{def_files_adef_defFilesAdef_sandboxed}{}\subsection{Sandboxed}\label{def_files_adef_defFilesAdef_sandboxed}
Specifies if the app will be launched inside a sandbox.

Permitted content in this section is\+:


\begin{DoxyItemize}
\item {\bfseries true} -\/ app will be run inside of a sandbox.
\item {\bfseries false} -\/ app won\textquotesingle{}t be run in a sandbox.
\end{DoxyItemize}

The default is {\bfseries true}.

If an app is unsandboxed (app is not inside of a sandbox), it can see the target device\textquotesingle{}s real root file system. A sandboxed app can\textquotesingle{}t see the target\textquotesingle{}s real root file system; a sandboxed app has a separate root file system, which it can\textquotesingle{}t leave.

Each app has its own user and primary group I\+Ds. The app\textquotesingle{}s user name and primary group name are both {\ttfamily appxxxx"}, where the {\ttfamily xxxx} is the name of the app.

User and/or group will be automatically created if missing for the specified app, but only users and groups of sandboxed apps will automatically be deleted when those apps are uninstalled.

\begin{DoxyVerb}sandboxed: false
\end{DoxyVerb}
\hypertarget{def_files_adef_defFilesAdef_start}{}\subsection{Start}\label{def_files_adef_defFilesAdef_start}
Specifies if the app should start automatically at start-\/up\+:
\begin{DoxyItemize}
\item {\bfseries auto} starts automatically by the Supervisor.
\item {\bfseries manual"} starts through manual prompt to the Supervisor. Default is {\bfseries auto}.
\end{DoxyItemize}

\begin{DoxyVerb}start: auto
\end{DoxyVerb}
\hypertarget{def_files_adef_defFilesAdef_version}{}\subsection{Version}\label{def_files_adef_defFilesAdef_version}
Optional field that specifies a string to use as the app\textquotesingle{}s version string.

\begin{DoxyVerb}version: 0.3a
\end{DoxyVerb}


\begin{DoxyNote}{Note}

\begin{DoxyItemize}
\item The {\bfseries {\ttfamily --append-\/to-\/version}} option to {\ttfamily mkapp} can be used to add to the app\textquotesingle{}s version string.
\item {\bfseries {\ttfamily app foo version}} can be run on-\/target to get the version string of the app called \char`\"{}foo\char`\"{}.
\end{DoxyItemize}
\end{DoxyNote}
\hypertarget{def_files_adef_defFilesAdef_watchdogAction}{}\subsection{Watchdog Action}\label{def_files_adef_defFilesAdef_watchdogAction}
The {\ttfamily watchdog\+Action} section sets the recovery action to take if a process in this app expires its watchdog. This value will be used if there is no value set in the processes section for a given process. See \hyperlink{def_files_adef_defFilesAdef_processWatchdogAction}{Watchdog Action} in the processes section.\hypertarget{def_files_adef_defFilesAdef_watchdogTimeout}{}\subsection{Watchdog Timeout}\label{def_files_adef_defFilesAdef_watchdogTimeout}
The {\ttfamily watchdog\+Timeout} section sets the default timeout in milliseconds for processes in this app. Will be used if there\textquotesingle{}s no value set in the processes section for a given process. See \hyperlink{def_files_adef_defFilesAdef_processWatchdogTimeout}{Watchdog Timeout} in the processes section.





Copyright (C) Sierra Wireless Inc. Use of this work is subject to license. \hypertarget{defFilesCdef}{}\section{Component Definition .cdef}\label{defFilesCdef}
{\ttfamily Component.\+cdef} files can contain these sections\+:\hypertarget{def_files_cdef_defFilesCdef_assets}{}\subsection{Assets}\label{def_files_cdef_defFilesCdef_assets}
Used to describe collections of information (fields) that can be exchanged with cloud services like Air\+Vantage.

The assets section can hold multiple assets, each asset is given a name.

An asset can be\+:


\begin{DoxyItemize}
\item {\ttfamily variables} values that are readable by the server; writable by the client.
\item {\ttfamily settings} values that are writable by the server; readable by the client.
\item {\ttfamily commands} that are custom commands sent from the server and executable by the client.
\end{DoxyItemize}

The variable and setting fields can have an associated data type and optionally a default value. The current data types supported are {\ttfamily  bool, int, float, and string}.

In this code sample, {\ttfamily hello\+World} app has one setting (the message to be logged), one variable (records how many times a message has been logged). The app also exposes two commands\+: one to log the currently configured message and one to log the default message.

The {\ttfamily assets} definition looks like this\+:

\begin{DoxyVerb}assets:
{
    message =
    {
        settings:
        {
            string greeterMessage = "Hello world."   // When requested log this message.
        }

        variables:
        {
            int greetCount     // How many times has a greeting been logged?
        }

        commands:
        {
            greetNow           // Log the message recorded in greeterMessage.
            standardGreeting   // Use "Hello world," instead of what is set in greeterMessage.
        }
    }
}
\end{DoxyVerb}


Once you\textquotesingle{}ve defined your assets, it\textquotesingle{}s up to the app to instantiate instances of them at run-\/time using the \hyperlink{c_le_avdata}{Air\+Vantage Data} A\+P\+I. Also see \hyperlink{howToAVData}{Manage Air\+Vantage Data}.\hypertarget{def_files_cdef_defFilesCdef_bundles}{}\subsection{Bundles}\label{def_files_cdef_defFilesCdef_bundles}
Lists additional files or directories to be copied from the build host into the app so they’re available to the app at runtime (e.\+g., audio files, web pages, executable scripts or programs built using some external build system).

\begin{DoxyVerb}bundles:
{
    file:
    {
        // Include the web server executable (built using some other build tool) in the app's /bin.
        [x] 3rdParty/webServer/bin/wwwServ  /bin/

        // Put the company logo into the app's /var/www/ for read-only access by the web server.
        images/abcCorpLogo.jpg  /var/www/

        // Make the appropriate welcome page for the product appear at /var/www/index.html.
        webContent/$PRODUCT_ID/welcome.html  /var/www/index.html

        // Create a file to record persistent custom audio messages into.
        [w] audio/defaultMessage.wav  /usr/share/sounds/customMessage.wav
    }

    dir:
    {
        // Recursively bundle the directory containing all the audio files into the app.
        // It will appear to the app read-only under /usr/share/sounds/.
        audio   /usr/share/sounds
    }
}
\end{DoxyVerb}


Three things need to be specified for each file or directory\+:
\begin{DoxyItemize}
\item access permissions
\item build system path
\item target path
\end{DoxyItemize}

{\bfseries Access permissions} -\/ any combination of one or more of the following letters, enclosed in square brackets\+:
\begin{DoxyItemize}
\item r = readable
\item w = writeable
\item x = executable
\end{DoxyItemize}

If permissions values are not specified, then read-\/only (\mbox{[}r\mbox{]}) is the default.

Directories always have executable permission set so they can be traversed. Setting the {\ttfamily \mbox{[}x\mbox{]}} permission in the {\ttfamily dir\+:} subsection causes the files under the directory to be made executable.

Setting {\ttfamily \mbox{[}w\mbox{]}} in the {\ttfamily dir\+:} subsection causes all files under that directory to be writeable, but the directory itself will not be writeable.

\begin{DoxyNote}{Note}
Directories in the persistent (flash) file system are never made writeable because the on-\/target flash file system does not support usage quotas (yet).
\end{DoxyNote}
{\bfseries Build system path} -\/ file system path on the build P\+C where the file is located at build time.

The path can be relative to the directory where the {\ttfamily }.adef file is located.

\begin{DoxyNote}{Note}
Environment variables can be used inside these paths.
\end{DoxyNote}
{\bfseries Target path} -\/ file system path on the target where the file will appear at runtime.

It\textquotesingle{}s an absolute path inside the app\textquotesingle{}s sandbox file system.

If the path ends with \textquotesingle{}/\textquotesingle{}, it means the directory path where the source object (file or directory) will be copied. The destination object will have the same name as the source object.

If the path doesn\textquotesingle{}t end in a \textquotesingle{}/\textquotesingle{}, it\textquotesingle{}s a full destination object path. The destination object could have a different name than the source object.

\begin{DoxyNote}{Note}
If the app is running unsandboxed, the bundled files and directories can be found in their installation location under {\ttfamily /opt/legato/apps/xxxx}, where xxxx is replaced by the app name.
\end{DoxyNote}
{\bfseries Quoting Paths}

File paths can be enclosed in quotation marks (either single \textquotesingle{} or double "). This is required when the file path contains spaces or comment start sequences \begin{DoxyVerb}"//" or  "/*"
\end{DoxyVerb}


{\bfseries File Ownership and Set-\/\+U\+I\+D Bits}

When the app is installed on a target\+: ~\newline

\begin{DoxyItemize}
\item the owner and group are set to {\ttfamily root} on all files in the app.
\item the {\ttfamily setuid} bit is cleared on everything in the app.
\end{DoxyItemize}\hypertarget{def_files_cdef_defFilesCdef_cFlags}{}\subsection{C Flags}\label{def_files_cdef_defFilesCdef_cFlags}
Provides a way to specify command-\/line arguments to pass to the compiler when compiling C source code files.

Flags are separated by whitespace.


\begin{DoxyCode}
cflags:
\{
    -g -O0
    -DDEBUG=1
\}
\end{DoxyCode}
\hypertarget{def_files_cdef_defFilesCdef_cxxFlags}{}\subsection{C++ Flags}\label{def_files_cdef_defFilesCdef_cxxFlags}
Provides a way to specify command-\/line arguments to pass to the compiler when compiling C++ source code files.

Flags are separated by whitespace.


\begin{DoxyCode}
cxxflags:
\{
    -std=c++0x
    -g -O0
\}
\end{DoxyCode}
\hypertarget{def_files_cdef_defFilesCdef_ldFlags}{}\subsection{Linker Flags}\label{def_files_cdef_defFilesCdef_ldFlags}
Provides a way to specify command-\/line arguments to pass to the compiler when linking C/\+C++ object (.o) files together into a component shared library (.so) file.

Flags are separated by whitespace.


\begin{DoxyCode}
ldflags:
\{
    -Lfoo/bar
\}
\end{DoxyCode}
\hypertarget{def_files_cdef_defFilesCdef_pools}{}\subsection{Pools}\label{def_files_cdef_defFilesCdef_pools}
\begin{DoxyWarning}{Warning}
This feature not yet implemented.
\end{DoxyWarning}
Specifies the number of memory pool blocks that each \hyperlink{c_memory}{memory pool} should contain.


\begin{DoxyCode}
pools:
\{
    myPool = 45
\}
\end{DoxyCode}
\hypertarget{def_files_cdef_defFilesCdef_provides}{}\subsection{Provides}\label{def_files_cdef_defFilesCdef_provides}
Lists things this component provides (exports) to other software either inside or outside of the app.

The only subsection supported today is the {\ttfamily api} subsection.\hypertarget{def_files_cdef_defFilesCdef_providesApi}{}\subsubsection{A\+P\+I}\label{def_files_cdef_defFilesCdef_providesApi}
Lists I\+P\+C services provided by this component to other components.

Contents use the same syntax as the {\ttfamily requires\+:} \hyperlink{def_files_cdef_defFilesCdef_requiresApi}{A\+P\+I} section, except the options are different.

Here\textquotesingle{}s a code sample where {\ttfamily  greet.\+api } defines a function called {\ttfamily Send()} where the C source code for the component (in {\ttfamily greet\+Server.\+c}) is implement a function called {\ttfamily greet\+\_\+\+Send()}.


\begin{DoxyCode}
provides:
\{
    api:
    \{
        greet.api   \textcolor{comment}{// We offer the Greet API to others so they can say “hello” to the world.}
        heat = digitalOutput.api
        cool = digitalOutput.api
    \}
\}

sources:
\{
    greetServer.c
    tempControl.c
\}
\end{DoxyCode}


The component must implement the A\+P\+I functions being provided.

In C, the source code must {\ttfamily \#include “interfaces.\+h”} to get the auto-\/generated function prototype definitions and type definitions. The function and type names defined in the {\ttfamily }.api files are prefixed with the interface name and an underscore (similar to required A\+P\+Is).\hypertarget{def_files_cdef_defFilesCdef_providesApiManualStart}{}\paragraph{\mbox{[}manual-\/start\mbox{]} Option}\label{def_files_cdef_defFilesCdef_providesApiManualStart}
To reduce the initialization code a component writer needs to write, the build tools automatically try to advertise the service when the executable is run. Sometimes this is not the preferred behaviour.

The {\bfseries {\ttfamily }\mbox{[}manual-\/start}\mbox{]} option tells the build tools {\bfseries not} to automatically advertise this A\+P\+I with the Service Directory when the process starts. If {\ttfamily }\mbox{[}manual-\/start\mbox{]} option is used, the component can control when it wants to start offering the service to others by calling the {\ttfamily xxxx\+\_\+\+Advertise\+Service()} function explicitly in the component source code when it\textquotesingle{}s ready.


\begin{DoxyCode}
provides:
\{
    api:
    \{
        foo.api [manual-start]
    \}
\}
\end{DoxyCode}
\hypertarget{def_files_cdef_defFilesCdef_providesApiAsync}{}\paragraph{\mbox{[}async\mbox{]} Option}\label{def_files_cdef_defFilesCdef_providesApiAsync}
The server of a service can also implement the functions as if they were called directly by the client (even though the client may be running inside another process). When the client calls an A\+P\+I function, the server\textquotesingle{}s A\+P\+I function gets called, and when the server returns from the function, the function returns in the client process.

Sometimes the server needs to hold onto the client request and do other things (like handing requests from other clients in the meantime) before sending a response back. This is called asynchronous mode, and is enabled using the {\ttfamily  \mbox{[}async\mbox{]} } keyword on the end of the {\ttfamily api} section entry\+:


\begin{DoxyCode}
provides:
\{
    api:
    \{
        bar.api [async]
    \}
\}
\end{DoxyCode}


When asynchronous mode is enabled for a server-\/side interface, the generated code changes as follows\+:
\begin{DoxyItemize}
\item {\ttfamily command\+Ref} parameter is added to the beginning of all the A\+P\+I functions\textquotesingle{} parameter lists.
\item return value is removed from every A\+P\+I function.
\item {\ttfamily Respond()} function is generated for every A\+P\+I function.
\end{DoxyItemize}

In async mode, the server responds to the client\textquotesingle{}s call to A\+P\+I function {\ttfamily F()} by calling the associated {\ttfamily F\+Respond()} function.

The {\ttfamily Respond} functions all take the {\ttfamily command\+Ref} as their first parameter. If an A\+P\+I function has a return value, that return value is sent to the client through the second parameter of the {\ttfamily Respond} function. Any output parameters defined in the A\+P\+I function are also passed as parameters to the {\ttfamily Respond} function.

See \hyperlink{interfaceDefLang}{Interface Definition Language} for more information, or try it and have a look at the generated header files.\hypertarget{def_files_cdef_defFilesCdef_requires}{}\subsection{Requires}\label{def_files_cdef_defFilesCdef_requires}
The {\ttfamily requires\+:} section specifies things the component needs from its runtime environment.

It can contain various subsections.\hypertarget{def_files_cdef_defFilesCdef_requiresApi}{}\subsubsection{A\+P\+I}\label{def_files_cdef_defFilesCdef_requiresApi}
Lists I\+P\+C A\+P\+Is used by this component.

Here\textquotesingle{}s a code sample of a component using the Configuration Data A\+P\+I (defined in le\+\_\+cfg.\+api) to read its configuration data\+:


\begin{DoxyCode}
requires:
\{
    api:
    \{
        le\_cfg.api
    \}
\}
\end{DoxyCode}


This creates a client-\/side I\+P\+C interface called {\ttfamily le\+\_\+cfg} on this component, and it makes the functions and data types defined inside {\ttfamily le\+\_\+cfg.\+api} available for use in the component\textquotesingle{}s program code.

The name of the {\ttfamily }.api file (minus the {\ttfamily }.api extension) is the name of the interface, and in C code, the names of functions and data types defined in the {\ttfamily }.api file are prefixed with the name of the interface with an underscore separator.


\begin{DoxyCode}
requires:
\{
    api:
    \{
        print.api \textcolor{comment}{// WriteLine() from the API will appear in my C code as "print\_WriteLine()".}
    \}
\}
\end{DoxyCode}


To rename the interface, an interface name followed by an equals sign (\textquotesingle{}=\textquotesingle{}) can be added in front of the {\ttfamily }.api file path.


\begin{DoxyCode}
requires:
\{
    api:
    \{
        hello = greet.api \textcolor{comment}{// Send() from the API will appear as "hello\_Send()" in my code.}
    \}
\}
\end{DoxyCode}


Multiple instances of the same A\+P\+I listed in the {\ttfamily api\+:} section must have unique instance names, and appear as separate functions with different prefixes.


\begin{DoxyCode}
requires:
\{
    api:
    \{
        heat = digitalOutput.api   \textcolor{comment}{// Used to turn on and off the heater.}
        cool = digitalOutput.api   \textcolor{comment}{// Used to turn on and off the cooling (A/C).}
    \}
\}
\end{DoxyCode}


If {\ttfamily  digital\+Output.\+api } defines two functions {\ttfamily On()} {\ttfamily and} Off(), the component’s source code would have four functions available to it\+: {\ttfamily heat\+\_\+\+On()}, {\ttfamily heat\+\_\+\+Off()}, {\ttfamily cool\+\_\+\+On()}, and {\ttfamily cool\+\_\+\+Off()}.

C/\+C++ source code must {\ttfamily  \#include “interfaces.\+h”} to use the auto-\/generated function definitions. The build tools will automatically generate a version of {\ttfamily interfaces.\+h} customized for your component that includes all declarations for all the interfaces the component uses.

The build tools search for the interface definition (.api) file based on the interface search path.\hypertarget{def_files_cdef_defFilesCdef_requiresApiOptions}{}\paragraph{Options}\label{def_files_cdef_defFilesCdef_requiresApiOptions}
To reduce the amount of initialization code a component needs to write, the build tools automatically generate the client-\/side I\+P\+C code for that A\+P\+I, and automatically try to connect to the server when the executable is run. There are a couple of options that can be used to suppress this behaviour.

The {\bfseries {\ttfamily }\mbox{[}types-\/only}\mbox{]} option tells the build tools the client only wants to use type definitions from the A\+P\+I. This means the client-\/side I\+P\+C code will not be generated for this A\+P\+I, but the types defined in the A\+P\+I will still be available to the component (through {\ttfamily interfaces.\+h} in C/\+C++).

The {\bfseries {\ttfamily }\mbox{[}manual-\/start}\mbox{]} option tells the build tools not to automatically connect to this A\+P\+I\textquotesingle{}s server when the process starts. This means the component can control when it wants to connect to the server by calling the {\ttfamily xxxx\+\_\+\+Connect\+Service()} function explicitly in the component source code.


\begin{DoxyCode}
requires:
\{
    api:
    \{
        foo.api [types-only]    \textcolor{comment}{// Only need typedefs from here.  Don't need IPC code generated.}
        bar.api [manual-start]  \textcolor{comment}{// I'll start this when I'm ready by calling bar\_ConnectService().}
    \}
\}
\end{DoxyCode}
\hypertarget{def_files_cdef_defFilesCdef_requiresFile}{}\subsubsection{File}\label{def_files_cdef_defFilesCdef_requiresFile}
Declares\+:
\begin{DoxyItemize}
\item specific files that reside on the target outside of the app, but made accessible to the app.
\item location inside the app\textquotesingle{}s sandbox where the file will appear.
\end{DoxyItemize}

Things listed in {\ttfamily requires} are expected to be found on the target at runtime. They\textquotesingle{}re not copied into the app at build time; they are made accessible to the app inside of its sandbox at runtime.

Each entry consists of two file system paths\+:


\begin{DoxyItemize}
\item path to the object in the file system outside of the app, which must be an absolute path (beginning with ‘/’).
\item absolute file system path inside the app’s sandbox where the object will appear at runtime.
\end{DoxyItemize}

A file path can be enclosed in quotation marks (either single \textquotesingle{} or double "). This is required when it contains spaces or character sequences that would start comments.

The first path can\textquotesingle{}t end in a \textquotesingle{}/\textquotesingle{}.

If the second path ends in a \textquotesingle{}/\textquotesingle{}, then it\textquotesingle{}s specifying the directory where the object appears, and the object has the same name inside the sandbox as it has outside the sandbox.

\begin{DoxyVerb}requires:
{
    file:
    {
        // I get character stream input from outside via a named pipe (read-only)
        /var/run/someNamedPipe  /var/run/

        // I need to be able to play back audio files installed in /usr/local/share/audio.
        "/usr/local/share/audio/error message.wav" /usr/share/audio/
        '/usr/local/share/audio/success message.wav' /usr/share/audio/
    }
}
\end{DoxyVerb}


\begin{DoxyNote}{Note}
Even though the file system object appears in the app\textquotesingle{}s sandbox it still needs permissions settings on the file. File permissions (both D\+A\+C and M\+A\+C) and ownership (group and user) on the original file in the target system remain in effect inside the sandbox.
\end{DoxyNote}
It\textquotesingle{}s also possible to give the object a different names inside and outside of the sandbox by adding a name to the end of the second path.

\begin{DoxyVerb}requires:
{
    file:
    {
        // Program uses /var/run/someNamedPipe which it calls /var/run/externalPipe.
        /var/run/someNamedPipe  /var/run/externalPipe
    }
}
\end{DoxyVerb}


\begin{DoxyWarning}{Warning}
When something is accessible from inside an app sandbox, there are potential security risks (e.\+g., access to the object could be exploited by the app, or hacker, to access sensitive information or launch a denial-\/of-\/service attack on other apps within the target device or other devices connected to the target device).
\end{DoxyWarning}
\hypertarget{def_files_cdef_defFilesCdef_requiresDevice}{}\subsubsection{Device}\label{def_files_cdef_defFilesCdef_requiresDevice}
Declares\+:
\begin{DoxyItemize}
\item device files that reside on the target outside of the app, but made accessible to the app.
\item location inside the app\textquotesingle{}s sandbox where the file will appear.
\item access permissions the app is given to the device file.
\end{DoxyItemize}

Things listed in {\ttfamily requires} are expected to be found on the target at runtime. They\textquotesingle{}re not copied into the app at build time; they are made accessible to the app inside of its sandbox at runtime.

Each entry consists of two file system paths and a set of optional access permissions\+:


\begin{DoxyItemize}
\item access permissions, readable (\mbox{[}r\mbox{]}) and/or writeable (\mbox{[}w\mbox{]}). Executable is not allowed on device files. If permission values are not specified, then read-\/only (\mbox{[}r\mbox{]}) is the default.
\item path to the object in the file system outside of the app, which must be an absolute path (beginning with ‘/’). This must be a path to a valid character or block device file.
\item absolute file system path inside the app’s sandbox where the object will appear at runtime.
\end{DoxyItemize}

A file path can be enclosed in quotation marks (either single \textquotesingle{} or double "). This is required when it contains spaces or character sequences that would start comments.

The first path can\textquotesingle{}t end in a \textquotesingle{}/\textquotesingle{}.

If the second path ends in a \textquotesingle{}/\textquotesingle{}, then it\textquotesingle{}s specifying the directory where the object appears, and the object has the same name inside the sandbox as it has outside the sandbox.

\begin{DoxyVerb}requires:
{
    device:
    {
        // I get read-only access to the SPI port.
        [r]     /dev/sierra_spi   /dev/sierra_spi

        // I get read-only access to the NMEA port.
                /dev/nmea         /dev/nmea

        // I get read and write access to the I2C port.
        [rw]    /dev/sierra_i2c   /dev/
    }
}
\end{DoxyVerb}


Note that if a hot-\/plug device is unplugged and plugged back in, the app must be restarted before it can access the device.

It\textquotesingle{}s also possible to give the object a different names inside and outside of the sandbox by adding a name to the end of the second path.

\begin{DoxyVerb}requires:
{
    device:
    {
        /dev/ttyS0  /dev/port1     // Program uses /dev/port1, but UART0 is called /dev/ttyS0.
    }
}
\end{DoxyVerb}


\begin{DoxyWarning}{Warning}
When something is accessible from inside an app sandbox, there are potential security risks (e.\+g., access to the object could be exploited by the app, or hacker, to access sensitive information or launch a denial-\/of-\/service attack on other apps within the target device or other devices connected to the target device).

This section is experimental. Future releases of may not support this section.
\end{DoxyWarning}
\hypertarget{def_files_cdef_defFilesCdef_requiresDir}{}\subsubsection{Dir}\label{def_files_cdef_defFilesCdef_requiresDir}
Specifies directories on target device to make accessible to the app.

The location inside the app\textquotesingle{}s sandbox at which the directory will appear is also specified.

Things listed here are expected to be found on the target at runtime. They are not copied into the app at build time; they are made accessible to the app inside of its sandbox at runtime.

Each entry consists of two file system paths\+:


\begin{DoxyItemize}
\item The {\bfseries first} path is the path to the directory {\bfseries outside} of the app. This must be an absolute path (beginning with ‘/’) and can never end in a \textquotesingle{}/\textquotesingle{}.
\item The {\bfseries second} path is the absolute path {\bfseries inside} the app’s sandbox where the directory will appear at runtime.
\end{DoxyItemize}

Paths can be enclosed in quotation marks (either single \textquotesingle{} or double "). This is required when it contains spaces or character sequences that would start comments.

If the second path ends in a \textquotesingle{}/\textquotesingle{}, then it\textquotesingle{}s specifying the directory into which the object will appear, and the object will have the same name inside the sandbox as it has outside the sandbox.

\begin{DoxyVerb}requires:
{
    dir:
    {
        // I need access to /proc for debugging.
        /proc   /

        // For now, I want access to all executables and libraries in /bin and /lib.
        // Later I'll remove this and replace with just the files I really need in the field.
        // Also, I don't want to hide the stuff that the tools automatically bundle into my app's
        // /bin and /lib for me, so I'll make the root file system's /bin and /lib accessible as
        // my app's /usr/bin and /usr/lib.
        /bin    /usr/bin
        /lib    /usr/lib
    }
}
\end{DoxyVerb}


\begin{DoxyNote}{Note}
Although the directory appears in the app\textquotesingle{}s sandbox, it doesn\textquotesingle{}t mean the app can access it. The directory permissions settings must also allow it. File permissions (both D\+A\+C and M\+A\+C) and ownership (group and user) on the original files in the target system remain in effect inside the sandbox.
\end{DoxyNote}
\begin{DoxyWarning}{Warning}
Any time anything is accessible from inside an app sandbox, the security risks must be considered carefully. Ask yourself if access to the object can be exploited by the app (or a hacker who has broken into the app) to access sensitive information or launch a denial-\/of-\/service attack on other apps within the target device or other devices connected to the target device?
\end{DoxyWarning}
\begin{DoxyNote}{Note}
It\textquotesingle{}s not possible to put anything inside of a directory that was mapped into the app from outside of the sandbox. If you {\itshape require} {\ttfamily /bin} to appear at {\ttfamily /usr/bin}, you can\textquotesingle{}t then {\itshape bundle} a file into {\ttfamily /usr/bin} or {\itshape require} something to appear in {\ttfamily /usr/bin}; that would have an effect on the contents of the /bin directory outside of the app.
\end{DoxyNote}
\hypertarget{def_files_cdef_defFilesCdef_requiresLib}{}\subsubsection{Lib}\label{def_files_cdef_defFilesCdef_requiresLib}
The {\ttfamily lib\+:} subsection of the {\ttfamily requires\+:} section is used to add a required library to a Component.

A required library is a library file that must exist in the target file system (outside the app\textquotesingle{}s sandbox), and is needed by a component at runtime.

The required library must be linked with the component that\textquotesingle{}s part of the the executable. The library name is specified without the leading \char`\"{}lib\char`\"{} or the trailing \char`\"{}.\+so\char`\"{}.


\begin{DoxyCode}
requires:
\{
    lib:
    \{
        foo    \textcolor{comment}{// I need access to libfoo.so}
    \}
\}
\end{DoxyCode}


This will result in {\ttfamily -\/lfoo} being passed to the linker when linking any executables that include this component. You will need a copy of this library where the linker can find it on the build host.

Also, on the target device at runtime, the dynamic linker will look for the library so it must be made available inside the app sandbox, somewhere in the dynamic linker\textquotesingle{}s library search path. This can be done using the {\ttfamily files\+:} or {\ttfamily dirs\+:} subsection of the {\ttfamily requires\+:} section of either the {\ttfamily .cdef} or {\ttfamily .adef} file.


\begin{DoxyCode}
requires:
\{
    file:
    \{
        \textcolor{comment}{// Make the "foo" library available inside the app sandbox (in the app's /lib directory).}
        /usr/lib/libfoo.so.3     /lib/
        /usr/lib/libfoo.so.3.1.1 /lib/
    \}
\}
\end{DoxyCode}
\hypertarget{def_files_cdef_defFilesCdef_requiresComponent}{}\subsubsection{Component}\label{def_files_cdef_defFilesCdef_requiresComponent}
Declares this component depends on another component.


\begin{DoxyCode}
requires:
\{
    component:
    \{
        foo
        bar
    \}
\}
\end{DoxyCode}


Any app that uses a component will also use any other components that component requires, and any components they require, etc.

Specifying a dependency on another component ensures that calls to component initialization functions ( {\ttfamily C\+O\+M\+P\+O\+N\+E\+N\+T\+\_\+\+I\+N\+I\+T} in C/\+C++ components ) are sorted in the correct order. If component A depends on component B, then component B will be initialized first.

Dependency loops are not allowed\+: component C can\textquotesingle{}t depend on another component that (either directly or indirectly) depends on component C. The build tools detect dependency loops and report any error.\hypertarget{def_files_cdef_defFilesCdef_sources}{}\subsection{Sources}\label{def_files_cdef_defFilesCdef_sources}
Contains a list of source code files.

If C or C++ code, one source file must implement a {\ttfamily C\+O\+M\+P\+O\+N\+E\+N\+T\+\_\+\+I\+N\+I\+T} function. The framework will automatically call that function at start-\/up.


\begin{DoxyCode}
sources:
\{
    foo.c
    bar.c
    init.c      \textcolor{comment}{// This one implements the COMPONENT\_INIT}
\}
\end{DoxyCode}






Copyright (C) Sierra Wireless Inc. Use of this work is subject to license. \hypertarget{c_le_avdata}{}\subsection{Air\+Vantage Data}\label{c_le_avdata}
\hyperlink{le__avdata__interface_8h}{A\+P\+I Reference}





This A\+P\+I provides a data service to allow apps to manage app-\/specific data on the Air\+Vantage server.

Data is setup as {\bfseries assets} --- a collection of fields that can be managed by the Air\+Vantage server.

An asset field is a single piece of information that can be managed by the Air Vantage server. There can be multiple fields in an asset.

A field can be\+:
\begin{DoxyItemize}
\item {\ttfamily variable} allows an app to set the value, and can be read from the A\+V server.
\item {\ttfamily setting} allows the A\+V server to set the value, and can be read by an app.
\item {\ttfamily command} allows the A\+V server to invoke a command or function in the app.
\end{DoxyItemize}

Currently, only variable and setting fields are supported; command fields will be supported in the future.

All fields have names. Variable and setting fields also have a type. The possible field types, along with the default value are\+:
\begin{DoxyItemize}
\item string (empty string)
\item integer (0)
\item float (0.\+0)
\item boolean (false)
\item binary (zero-\/length block).
\end{DoxyItemize}

Default values can be overwritten in the asset definition. Currently, only string, integer and boolean fields are supported; float and binary fields will be supported in the future.\hypertarget{c_le_avdata_le_avdata_instance}{}\subsubsection{Asset Data Instances}\label{c_le_avdata_le_avdata_instance}
An app that needs to send data to the Air\+Vantage server must first create an asset instance using le\+\_\+avc\+\_\+\+Create() with the name of the asset definition. Once an asset instance is created, the app can begin accessing the instance\textquotesingle{}s fields.

Multiple instances of the same asset can be created, as well as multiple instances of different assets.

Asset definitions are specified in the \hyperlink{def_files_cdef_defFilesCdef_assets}{Assets} section of the app\textquotesingle{}s {\ttfamily cdef} file.\hypertarget{c_le_avdata_le_avdata_field}{}\subsubsection{Field Values and Activity}\label{c_le_avdata_le_avdata_field}
Set functions are available to set variable field values. Get functions are available to get settings fields\textquotesingle{} values.

An app can register a handler so that it can be called when activity occurs on a field. This is optional for variable and setting fields, but is required for command fields.
\begin{DoxyItemize}
\item {\ttfamily variable} called when the field is read by the A\+V server. The app can then call the appropriate set function to provide a new value for the field.
\item {\ttfamily setting} called when the field has been updated by the A\+V server with a new value. The app can use the appropriate get function to retrieve the new value.
\item {\ttfamily command} called when the A\+V server wants to invoke a command. The app should perform an appropriate action or call a function to execute this command. Currently, registering a handler is only supported for setting fields; variable and command fields will be added in the future.
\end{DoxyItemize}

Leaving it optional to register handlers for variable and setting fields allows an app to decide how it wants to access variable and setting fields. It can decide to only do something in response to the A\+V server, or it can work independently of the A\+V server updating variables when it has a new value, and reading settings only when it needs the value.





Copyright (C) Sierra Wireless Inc. Use of this work is subject to license. \hypertarget{defFilesSdef}{}\section{System Definition .sdef}\label{defFilesSdef}
{\ttfamily .sdef} files can contain these sections\+:\hypertarget{def_files_sdef_defFilesSdef_App}{}\subsection{App}\label{def_files_sdef_defFilesSdef_App}
An {\ttfamily apps\+:} section declares one or more apps to be deployed to the target system.


\begin{DoxyCode}
apps:
\{
    webserver
\}
\end{DoxyCode}


This looks for an app definition file called {\ttfamily webserver.\+adef} and includes it in the system.

The {\ttfamily apps\+:} section can override limits and other app settings.

Here\textquotesingle{}s a code sample to deploy a web server limiting its share of the C\+P\+U under heavy load to 500 (see \hyperlink{def_files_adef_defFilesAdef_cpuShare}{C\+P\+U Share})\+:


\begin{DoxyCode}
apps:
\{
    webServer
    \{
        cpuShare: 500
    \}
\}
\end{DoxyCode}


Any of the following subsections can be used in an {\ttfamily }.sdef {\ttfamily apps\+:} section, and will override the .adef setting for all processes in that app\+:

\hyperlink{def_files_adef_defFilesAdef_cpuShare}{C\+P\+U Share} ~\newline
 \hyperlink{def_files_adef_defFilesAdef_processFaultAction}{Fault Action} ~\newline
 \hyperlink{def_files_adef_defFilesAdef_groups}{Groups} ~\newline
 \hyperlink{def_files_adef_defFilesAdef_processMaxCoreDumpFileBytes}{Max Core Dump File Bytes} ~\newline
 \hyperlink{def_files_adef_defFilesAdef_processMaxFileBytes}{Max File Bytes} ~\newline
 \hyperlink{def_files_adef_defFilesAdef_processMaxFileDescriptors}{Max File Descriptors} ~\newline
 \hyperlink{def_files_adef_defFilesAdef_maxFileSystemBytes}{Max File System Bytes} ~\newline
 \hyperlink{def_files_adef_defFilesAdef_processMaxLockedMemoryBytes}{Max Locked Memory Bytes} ~\newline
 \hyperlink{def_files_adef_defFilesAdef_maxMemoryBytes}{Max Memory Bytes} ~\newline
 \hyperlink{def_files_adef_defFilesAdef_maxMQueueBytes}{Max M\+Queue Bytes} ~\newline
 \hyperlink{def_files_sdef_defFilesSdef_maxPriority}{Max Priority} ~\newline
 \hyperlink{def_files_adef_defFilesAdef_maxQueuedSignals}{Max Queued Signals} ~\newline
 \hyperlink{def_files_adef_defFilesAdef_maxThreads}{Max Threads} ~\newline
 \hyperlink{def_files_adef_defFilesAdef_maxSecureStorageBytes}{Max Secure Storage Bytes} ~\newline
 \hyperlink{def_files_adef_defFilesAdef_pools}{Pools} ~\newline
 \hyperlink{def_files_adef_defFilesAdef_sandboxed}{Sandboxed} ~\newline
 \hyperlink{def_files_adef_defFilesAdef_start}{Start} ~\newline
 \hyperlink{def_files_adef_defFilesAdef_watchdogAction}{Watchdog Action} ~\newline
 \hyperlink{def_files_adef_defFilesAdef_watchdogTimeout}{Watchdog Timeout} ~\newline
\hypertarget{def_files_sdef_defFilesSdef_maxPriority}{}\subsubsection{Max Priority}\label{def_files_sdef_defFilesSdef_maxPriority}
Sets the maximum priority level for running the app.

Acts as a ceiling only. Lowers the priority level if an app would otherwise be allowed to use a higher priority. It won\textquotesingle{}t raise the priority level for any processes in the app.

Here\textquotesingle{}s a code sample where a process in the app\textquotesingle{}s .adef is configured to start at high priority, and the .sdef section for that app has max\+Priority set to {\ttfamily medium} so the process will start at medium priority.


\begin{DoxyCode}
apps:
\{
    foo
    \{
        maxPriority: high
    \}
\}
\end{DoxyCode}


Another process in the same .adef configured to start at low priority will still start at low priority.\hypertarget{def_files_sdef_defFilesSdef_bindings}{}\subsection{Bindings}\label{def_files_sdef_defFilesSdef_bindings}
Lists I\+P\+C {\ttfamily bindings} that connect apps’ external I\+P\+C interfaces. They\textquotesingle{}re listed in the \hyperlink{def_files_adef_defFilesAdef_extern}{extern section of their {\ttfamily }.adef files}. Each binding connects one client-\/side interface to one server-\/side interface.

Interfaces use the app name and the interface name, separated by a period (‘.\+’). The two bound-\/together interfaces are separated by an arrow (\char`\"{}-\/$>$\char`\"{}).

Here\textquotesingle{}s a code sample\+:


\begin{DoxyCode}
apps:
\{
    vavController
    thermostat
    airHandlerProxy
\}

bindings:
\{
    \textcolor{comment}{// Connect the VAV controller to the thermostat}
    vavController.temp -> thermostat.temp
    vavController.setpoint -> thermostat.setpoint

    \textcolor{comment}{// Connect the VAV controller to the supply air duct temperature sensor}
    vavController.ductTemp -> ductTemperatureSensor.temp

    \textcolor{comment}{// Hook up the VAV control outputs to the damper actuators.}
    vavController.supplyDamper -> supplyAirDamper.damper
    vavController.returnDamper -> returnAirDamper.damper

    \textcolor{comment}{// Use a network proxy to request duct temperature changes from the Air Handling Unit.}
    vavController.airHandler -> airHandlerProxy.airHandler
\}
\end{DoxyCode}


For security reasons, binding between apps is never performed unless explicitly specified in the {\ttfamily }.sdef or {\ttfamily }.adef files.

Beware that if an app\textquotesingle{}s client-\/side extern interface instance is left unbound, the process or processes that require that interface may not be able to run.





Copyright (C) Sierra Wireless Inc. Use of this work is subject to license. 