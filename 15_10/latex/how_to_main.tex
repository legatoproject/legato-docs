This section contains info about how to complete frequently needed tasks.

\hyperlink{howToCustomizeUART}{Customize U\+A\+R\+T} ~\newline
 \hyperlink{howToCustomizeUSB}{Customize Legato Yocto U\+S\+B} ~\newline
 \hyperlink{howToDebug}{Debug Apps} ~\newline
 \hyperlink{howToConfigTree}{Manage Config Tree} ~\newline
 \hyperlink{howToEcall}{Manage e\+Call} ~\newline
 \hyperlink{howToGNSS}{Manage G\+N\+S\+S} ~\newline
 \hyperlink{howToPowerMgmt}{Manage Device Power} ~\newline
 \hyperlink{howToNMEA}{N\+M\+E\+A Port Setup} ~\newline
 \hyperlink{howToPortLegacyGen}{Port Legacy Apps} ~\newline
 \hyperlink{howToSetUserTimebase}{Set User Timebase} ~\newline
 \hyperlink{howToAV}{Use Air\+Vantage} ~\newline
 \hyperlink{howToLogs}{Use Logs}





Copyright (C) Sierra Wireless Inc. Use of this work is subject to license. \hypertarget{howToCustomizeUART}{}\section{Customize U\+A\+R\+T}\label{howToCustomizeUART}
The topic describes how to customize the W\+P710x Dev Kit U\+A\+R\+T port so it can be configured as an application serial port.\hypertarget{how_to_customize_u_a_r_t_howToCustomizeUART_syntax}{}\subsection{Syntax}\label{how_to_customize_u_a_r_t_howToCustomizeUART_syntax}
The syntax is\+: 
\begin{DoxyCode}
AT!MAPUART=a,b
\end{DoxyCode}
 Where
\begin{DoxyItemize}
\item {\ttfamily a} is the service type
\item {\ttfamily b} is the U\+A\+R\+T number, either {\ttfamily 1} or {\ttfamily 2}.
\end{DoxyItemize}

Here\textquotesingle{}s a code sample setting {\ttfamily 17} for the service type and {\ttfamily 1} for the U\+A\+R\+T number\+: 
\begin{DoxyCode}
AT!MAPUART=17,1
\end{DoxyCode}


Reboot your target after running the command.

Run {\ttfamily A\+T!\+M\+A\+P\+U\+A\+R\+T}? to check the config settings.\hypertarget{how_to_customize_u_a_r_t_howToCustomizeUART_serviceType}{}\subsection{Service Type}\label{how_to_customize_u_a_r_t_howToCustomizeUART_serviceType}
The current A\+T command spec defines the service as\+:

\begin{DoxyVerb}Service to map to UART:
0:  UART disabled
1:  AT Command service
2:  Diagnostic Message service
3:  Reserved
4:  NMEA service
5-15:  Reserved
16: Linux Console
17: Customer Linux application
\end{DoxyVerb}






Copyright (C) Sierra Wireless Inc. Use of this work is subject to license. \hypertarget{howToCustomizeUSB}{}\section{Customize Legato Yocto U\+S\+B}\label{howToCustomizeUSB}
This topic provides details on how to customize the U\+S\+B behavior of the Legato Yocto distribution when working with the following Sierra Wireless devices\+:
\begin{DoxyItemize}
\item A\+R755x
\item A\+R8652
\item W\+P71xx
\end{DoxyItemize}

The information is based on the U\+S\+B Driver Developer Guide available from the \href{http://source.sierrawireless.com/}{\tt Source}.\hypertarget{how_to_customize_u_s_b_howToCustomizeUSB_enumeration}{}\subsection{Enumeration Sequence}\label{how_to_customize_u_s_b_howToCustomizeUSB_enumeration}
U\+S\+B endpoints are enumerated during the boot sequence as follows\+:

First, the U\+S\+B boot script ({\ttfamily S41usb}) starts {\ttfamily /usr/bin/usb/boot\+\_\+hsusb\+\_\+composition}.

This script executes the following\+: \begin{DoxyVerb}IF /etc/legato/usbmode exists
THEN execute /etc/legato/usbsetup
ELSE setup default composition
\end{DoxyVerb}


The actions taken are described in the rest of this topic.\hypertarget{how_to_customize_u_s_b_howToCustomizeUSB_defaultEndpoints}{}\subsection{Default Endpoints}\label{how_to_customize_u_s_b_howToCustomizeUSB_defaultEndpoints}
The table shows all the possible U\+S\+B interfaces available on the A\+R7 family.



These endpoints are grouped into compositions, and a fixed set of compositions is provided as shown in the next table. Switching between these compositions is done using the {\ttfamily A\+T!\+U\+S\+B\+C\+O\+M\+P} command. The default composition for the A\+R7 is composition 2.

\hypertarget{how_to_customize_u_s_b_howToCustomizeUSB_customizeEndpoints}{}\subsection{Customize Endpoints}\label{how_to_customize_u_s_b_howToCustomizeUSB_customizeEndpoints}
If none of the provided compositions meet your needs it\textquotesingle{}s possible to fine-\/tune the U\+S\+B configuration using a mechanism built into Legato. The first step is to create {\ttfamily /etc/legato/usbmode}. An example file is shipped in this directory in the default distribution, called {\ttfamily usbmode.\+ex}. {\ttfamily usbmode} contains a list, one per line, of the endpoints you wish to enumerate. For example, the default file contains

\begin{DoxyVerb}ecm
acm
nmea
\end{DoxyVerb}


The full list of supported modes is\+:

\begin{TabularC}{5}
\hline
\rowcolor{lightgray}{\bf Name }&{\bf Interface Type }&{\bf Description }&{\bf Target endpoint }&{\bf Sample host endpoint  }\\\cline{1-5}
ecm &C\+D\+C-\/\+E\+C\+M &Providing an I\+P connection over U\+S\+B &N\+I\+C usb0 &N\+I\+C (eg enp0s20u6i22) \\\cline{1-5}
eem &C\+D\+C-\/\+E\+E\+M &Providing an I\+P connection over U\+S\+B &N\+I\+C usb0 &N\+I\+C (eg enp0s20u6i22) \\\cline{1-5}
acm &C\+D\+C-\/\+A\+C\+M &Providing an A\+T command port &N/\+A &/dev/tty\+A\+C\+M0 \\\cline{1-5}
nmea &serial &N\+M\+E\+A strings when positioning is enabled &N/\+A &/dev/tty\+U\+S\+B0 \\\cline{1-5}
audio &U\+S\+B audio &Expose the target as a sound card &N/\+A &pcm\+C2\+D0c ~\newline
 pcm\+C2\+D0p \\\cline{1-5}
serial &serial &A serial interface to the embedded Linux &/dev/tty\+G\+S0 &/dev/tty\+U\+S\+B0 \\\cline{1-5}
usb\+\_\+mbim &C\+D\+C-\/\+M\+B\+I\+M &Mobile Broadband Interface Model ~\newline
 (an extension of C\+D\+C-\/\+N\+C\+M) &N/\+A &N\+I\+C (eg wwp0s20u6i12) ~\newline
 + cdc-\/wdm0 \\\cline{1-5}
\end{TabularC}




Copyright (C) Sierra Wireless Inc. Use of this work is subject to license. \hypertarget{howToDebug}{}\section{Debug Apps}\label{howToDebug}
This topic summarizes different ways to debug apps running on Legato.

There are various tools available to determine why your app is failing or blocking\+:


\begin{DoxyItemize}
\item \hyperlink{howToLogs}{Use Logs} to set filters, tracing, etc.
\item \hyperlink{how_to_debug_howToDebug_appStatus}{Check App Status} for current app info.
\item \hyperlink{how_to_debug_howToDebug_sdir}{Run sdir} to check if your is app connected.
\item \hyperlink{how_to_debug_howToDebug_execInApp}{Run exe\+In\+App} to rerun the app.
\item \hyperlink{how_to_debug_howToDebug_inspect}{Inspect App} to review memory pool details.
\item \hyperlink{how_to_debug_howToDebug_openTools}{Use Open Source Tools} can help, too.
\end{DoxyItemize}\hypertarget{how_to_debug_howToDebug_appStatus}{}\subsection{Check App Status}\label{how_to_debug_howToDebug_appStatus}
Run {\ttfamily app status} on the target to list all apps and their current status. See \hyperlink{toolsTarget_app}{app}\hypertarget{how_to_debug_howToDebug_sdir}{}\subsection{Run sdir}\label{how_to_debug_howToDebug_sdir}
Run {\ttfamily sdir list} on the target to determine if the app is failing because it\textquotesingle{}s unbound (I\+P\+C failing) or the server isn\textquotesingle{}t running (not connecting). This helps if your client is waiting for a service to appear or a binding to be created. {\ttfamily sdir} can also list apps or users offering services. See \hyperlink{toolsTarget_sdir}{sdir}\hypertarget{how_to_debug_howToDebug_execInApp}{}\subsection{Run exe\+In\+App}\label{how_to_debug_howToDebug_execInApp}
Run {\ttfamily  exec\+In\+App \mbox{[}proc\+Name=N\+A\+M\+E\mbox{]} exec\+Path \mbox{[}A\+R\+G\+S\mbox{]}} to execute a specific process in a running app\textquotesingle{}s sandbox. {\ttfamily exec\+In\+App} can also set process priority. See \hyperlink{toolsTarget_execInApp}{exec\+In\+App}\hypertarget{how_to_debug_howToDebug_inspect}{}\subsection{Inspect App}\label{how_to_debug_howToDebug_inspect}
Run {\ttfamily inspect \mbox{[}interval=S\+E\+C\+O\+N\+D\+S\mbox{]}} to set memory usage updates to help narrow any memory leaks causing failure. See \hyperlink{toolsTarget_inspect}{inspect}\hypertarget{how_to_debug_howToDebug_openTools}{}\subsection{Use Open Source Tools}\label{how_to_debug_howToDebug_openTools}
You can also try some standard, open source debugging tools like\+:
\begin{DoxyItemize}
\item {\ttfamily strace} -\/ a system call tracer good for low level code examination.
\item {\ttfamily G\+D\+B} -\/ the G\+N\+U project debugger good for setting breakpoints.
\end{DoxyItemize}

Run the -\/help option for documentation.





Copyright (C) Sierra Wireless Inc. Use of this work is subject to license. \hypertarget{howToConfigTree}{}\section{Manage Config Tree}\label{howToConfigTree}
These topics provide some how to details on managing the target config tree\+:

\hyperlink{howToConfigTreeImportExport}{Import/\+Export Config Trees} ~\newline
 \hyperlink{howToConfigTreeTransactions}{Config Tree Transactions} ~\newline
 \hyperlink{howToConfigTreeSchema}{Config Tree Schema}

There\textquotesingle{}s also some \hyperlink{basicConfig}{background} info.





Copyright (C) Sierra Wireless Inc. Use of this work is subject to license. \hypertarget{howToConfigTreeImportExport}{}\subsection{Import/\+Export Config Trees}\label{howToConfigTreeImportExport}
This topic summarizes how to create a script to automate importing and exporting config tree settings.

Legato currently overwrites private config trees during upgrading. The import/export script is an easy way to create a copy of your config trees so you can reinstate them after installing a new Legato version without rebuilding them.

This code sample shows how you can wrap {\ttfamily instapp} in a shell script that will automate the import/export functions\+:


\begin{DoxyCode}
\textcolor{preprocessor}{#!/bin/bash}

APP\_NAME=$1

\textcolor{keywordflow}{if} [ -z \textcolor{stringliteral}{"$APP\_NAME"} ]
then
    >&2 echo \textcolor{stringliteral}{"Application name not specified."}
    exit 1
fi


\textcolor{keywordflow}{if} [ -z \textcolor{stringliteral}{"$2"} ]
then
    \textcolor{keywordflow}{if} [ -z \textcolor{stringliteral}{"$DEST\_IP"} ]
    then
        >&2 echo \textcolor{stringliteral}{"Device IP address not specified."}
         exit 1
    fi
    DEVICE\_IP=$DEST\_IP
\textcolor{keywordflow}{else}
    DEVICE\_IP=$2
fi


\textcolor{keyword}{function} cfg
\{
    CMD=$1
\textcolor{preprocessor}{    #echo "Test: "}
    ssh root@$DEVICE\_IP \textcolor{stringliteral}{"/usr/local/bin/config $CMD $APP\_NAME:/ ~/$APP\_NAME.cfg"}
\}


cfg export
instapp $APP\_NAME.ar7 $DEVICE\_IP
cfg \textcolor{keyword}{import}
\end{DoxyCode}
 If you save this script as {\ttfamily inst.\+sh}, you can run like this\+: 
\begin{DoxyCode}
$ inst.sh <myApp> <myDeviceIP> 
\end{DoxyCode}


or


\begin{DoxyCode}
$ export DEST\_IP=<myDeviceIP>
$ inst.sh myApp
\end{DoxyCode}






Copyright (C) Sierra Wireless Inc. Use of this work is subject to license. \hypertarget{howToConfigTreeTransactions}{}\subsection{Config Tree Transactions}\label{howToConfigTreeTransactions}
This topic summarizes how to create read and write transactions to manage a target\textquotesingle{}s configuration database (config tree).

There are also pre-\/built \hyperlink{legatoServicesConfigTree}{Config Tree} A\+P\+Is and a target \hyperlink{toolsTarget_config}{config} tool.\hypertarget{how_to_config_tree_transactions_howToConfigTree_read}{}\subsubsection{Read from own Tree}\label{how_to_config_tree_transactions_howToConfigTree_read}
By default, every app gets read access to their own tree.

This code sample shows how an app can read a value from its own tree\+:


\begin{DoxyCode}
\hyperlink{le__cfg__interface_8h_a646829934bb23a878e19ade2c3e01eba}{le\_cfg\_IteratorRef\_t} iteratorRef = \hyperlink{le__cfg__interface_8h_aa766bff3a3ddbd2769b903fc56f6d9d2}{le\_cfg\_CreateReadTxn}(\textcolor{stringliteral}{"/test"});
\textcolor{keywordtype}{bool} myBoolVal = \hyperlink{le__cfg__interface_8h_aa3898fcb0d62b03c9a238d36b42d7a63}{le\_cfg\_GetBool}(iteratorRef, \textcolor{stringliteral}{"isBoolSet"}, \textcolor{keyword}{false});

\textcolor{keywordflow}{if} (myBoolVal)
\{
    \hyperlink{le__log_8h_a23e6d206faa64f612045d688cdde5808}{LE\_INFO}(\textcolor{stringliteral}{"The test value was set."});
\}
\textcolor{keywordflow}{else}
\{
    \hyperlink{le__log_8h_a23e6d206faa64f612045d688cdde5808}{LE\_INFO}(\textcolor{stringliteral}{"The test value was not set."});
\}
\end{DoxyCode}


This code tries to read a value from the tree named, {\itshape my\+App}. If the value {\itshape /test/is\+Bool\+Set} was previously written to the tree, that value will be returned. Otherwise, the default value {\itshape false} is returned.\hypertarget{how_to_config_tree_transactions_howToConfigTree_write}{}\subsubsection{Write to Tree}\label{how_to_config_tree_transactions_howToConfigTree_write}
Before you can write to a tree, you need to run as a user with the correct permissions. Permission settings are handled slightly differently, instead of writing under apps in the system tree, you write permission settings under the user\textquotesingle{}s branch.

Processes running under the root user or those with the same user I\+D as the config tree process aren\textquotesingle{}t restricted by permissions.

To create permissions, as the root user on the target console run {\ttfamily config set} using the {\ttfamily write} option\+:


\begin{DoxyCode}
$ config set /apps/myApp/configLimits/acl/myApp write
\end{DoxyCode}


The permissions are added to the system tree under {\ttfamily /apps/my\+App/config\+Limits/acl/my\+App}.

Once the permissions are set, you can add a write function like this\+:


\begin{DoxyCode}
\hyperlink{le__cfg__interface_8h_a646829934bb23a878e19ade2c3e01eba}{le\_cfg\_IteratorRef\_t} iteratorRef = \hyperlink{le__cfg__interface_8h_a9c817e5edf0df97034fdc432ce8d0f18}{le\_cfg\_CreateWriteTxn}(\textcolor{stringliteral}{"/test"});
\end{DoxyCode}
\hypertarget{how_to_config_tree_transactions_howToConfigTree_twoApps}{}\subsubsection{Two Apps One Tree}\label{how_to_config_tree_transactions_howToConfigTree_twoApps}
You may need to have one app read and another app read and write from a common tree.

To set permissions for two apps, as the root user on the target console run {\ttfamily config set} using the {\ttfamily common} option\+:


\begin{DoxyCode}
$ config set /apps/readerApp/configLimits/acl/common read
$ config set /apps/writerApp/configLimits/acl/common write
\end{DoxyCode}


Granting write permission on a tree also gives read permission.

Once the permissions are set, you need to create a read to the tree name that includes the iterator path {\ttfamily common\+:} like this\+:


\begin{DoxyCode}
\hyperlink{le__cfg__interface_8h_a646829934bb23a878e19ade2c3e01eba}{le\_cfg\_IteratorRef\_t} iteratorRef = \hyperlink{le__cfg__interface_8h_aa766bff3a3ddbd2769b903fc56f6d9d2}{le\_cfg\_CreateReadTxn}(\textcolor{stringliteral}{"
      common:/test"});
\textcolor{keywordtype}{bool} myBoolVal = \hyperlink{le__cfg__interface_8h_aa3898fcb0d62b03c9a238d36b42d7a63}{le\_cfg\_GetBool}(iteratorRef, \textcolor{stringliteral}{"isBoolSet"}, \textcolor{keyword}{false});

\textcolor{keywordflow}{if} (myBoolVal)
\{
    \hyperlink{le__log_8h_a23e6d206faa64f612045d688cdde5808}{LE\_INFO}(\textcolor{stringliteral}{"The test value was set."});
\}
\textcolor{keywordflow}{else}
\{
    \hyperlink{le__log_8h_a23e6d206faa64f612045d688cdde5808}{LE\_INFO}(\textcolor{stringliteral}{"The test value was not set."});
\}
\end{DoxyCode}


A write also needs the tree name to include the iterator path {\ttfamily common\+:} like this\+: 
\begin{DoxyCode}
\hyperlink{le__cfg__interface_8h_a646829934bb23a878e19ade2c3e01eba}{le\_cfg\_IteratorRef\_t} iteratorRef = \hyperlink{le__cfg__interface_8h_a9c817e5edf0df97034fdc432ce8d0f18}{le\_cfg\_CreateWriteTxn}(\textcolor{stringliteral}{"
      common:/test"});
\hyperlink{le__cfg__interface_8h_a1b760f2ef78e9c12dc207a5cbe333c99}{le\_cfg\_SetBool}(iteratorRef, \textcolor{stringliteral}{"isBoolSet"}, \textcolor{keyword}{true});
\end{DoxyCode}
\hypertarget{how_to_config_tree_transactions_howToConfigTree_readAnyTree}{}\subsubsection{Read Any Tree}\label{how_to_config_tree_transactions_howToConfigTree_readAnyTree}
You can create read access to {\itshape any} tree, but typically you won\textquotesingle{}t want it to run as a root process.

This code sample shows a supervisor\+App with read access to any tree\+:


\begin{DoxyCode}
$ config set /apps/supervisorApp/configLimits/allAccess read
\end{DoxyCode}


If you don\textquotesingle{}t specify a tree name, the read will be created on the user\textquotesingle{}s default tree\+: the tree with the same name as the app.\hypertarget{how_to_config_tree_transactions_howToConfigTree_process}{}\subsubsection{Write Process Any Tree}\label{how_to_config_tree_transactions_howToConfigTree_process}
You may need to create a write for a process to {\itshape any} tree for a process that isn\textquotesingle{}t part of app. This code sample uses the user name instead of the app name\+:


\begin{DoxyCode}
$ config set /users/testUser/configLimits/allAccess write
\end{DoxyCode}






Copyright (C) Sierra Wireless Inc. Use of this work is subject to license. \hypertarget{howToConfigTreeSchema}{}\subsection{Config Tree Schema}\label{howToConfigTreeSchema}
This is Legato\textquotesingle{}s config tree schema\+:

\hyperlink{howToConfigTreeSchema_bindingConfig}{I\+P\+C Binding Configuration Data}

\begin{DoxyVerb}system:
    apps
          appName1
              sandboxed (true, false)
              startManual (true, false)
              maxFileSystemBytes (integer)
              maxMQueueBytes (integer)
              maxThreads (integer)
              maxQueuedSignals (integer)
              maxSecureStorageBytes (integer)
              groups
                  groupName0
                  groupName1
                  ...
                  groupNameN
              files
                 0
                      src (string)
                      dest (string)
                  1
                      src (string)
                      dest (string)
                  ...
                  N
              procs
                  procName1
                      maxCoreDumpFileBytes (integer)
                      maxFileBytes (integer)
                      maxLockedMemoryBytes (integer)
                      maxFileDescriptors (integer)
                      priority (string)
                      faultAction (string)
                      args
                          0 (string) -> must contain the executable path relative to the sandbox root.
                          1 (string)
                          ...
                          N (string)
                      envVars
                          varName0
                              varValue0 (string)
                          varName1
                              varValue1 (string)
                          ...
                          varNameN
                              varValueN (string)
\end{DoxyVerb}
 ~\newline






Copyright (C) Sierra Wireless Inc. Use of this work is subject to license. \hypertarget{howToConfigTreeSchema_bindingConfig}{}\subsection{I\+P\+C Binding Configuration Data}\label{howToConfigTreeSchema_bindingConfig}
Binding lists are generated by the mk tools (i.\+e., mkapp and mksys) and are installed into the \char`\"{}system\char`\"{} configuration tree by the installer. They are in the \char`\"{}system\char`\"{} tree to keep them protected from sandboxed applications.

Bindings between clients and servers are stored under the {\bfseries client\textquotesingle{}s} configuration. If the client is an application, the bindings are stored under the application\textquotesingle{}s configuration in the \char`\"{}system\char`\"{} tree. If the client is a non-\/application user, the bindings are stored under the \char`\"{}/users/bindings\char`\"{} branch of the \char`\"{}system\char`\"{} tree.

\begin{DoxyVerb}system:/
    apps/
        <app name>/
            bindings/
                <client interface name>
                        app         <string>
                        user        <string>
                        interface   <string>
    users/
        <user name>/
            bindings/
                <client interface name>
                        app         <string>
                        user        <string>
                        interface   <string>\end{DoxyVerb}


Each {\ttfamily $<$client interface name$>$} corresponds to an I\+P\+C interface required by a component (listed in the {\ttfamily api\+:} subsection of a {\ttfamily requires\+:} section in the component\textquotesingle{}s {\ttfamily Component.\+cdef} file).

Either {\ttfamily app} or {\ttfamily user} must exist, but never both.

The {\ttfamily app} is the name of the application that will run the server on the target.

The {\ttfamily user} is the name of the user account that will run the server on the target.

The {\ttfamily interface} name corresponds to the server I\+P\+C interface name.

\begin{DoxyNote}{Note}
Changing the binding configuration in the configuration tree does not automatically result in changes to the actual I\+P\+C bindings that are in effect in the Service Directory. The \hyperlink{toolsTarget_sdir}{sdir} implements a {\ttfamily load} command that can be used to update the Service Directory with the latest binding configuration from the configuration tree.
\end{DoxyNote}




Copyright (C) Sierra Wireless Inc. Use of this work is subject to license. \hypertarget{howToEcall}{}\section{Manage e\+Call}\label{howToEcall}
This \href{http://source.sierrawireless.com/resources/legato/ecallinterface/}{\tt P\+D\+F doc} contains general info about Legato\textquotesingle{}s e\+Call functionality with consolidated details on how to setup the e\+Call Service A\+P\+I for different European regions.

There\textquotesingle{}s also \hyperlink{sampleApps_eCall}{e\+Call} sample app info that includes config setting details.





Copyright (C) Sierra Wireless Inc. Use of this work is subject to license. \hypertarget{howToGNSS}{}\section{Manage G\+N\+S\+S}\label{howToGNSS}
This topic summarizes how to manage a G\+N\+S\+S target device.

See \hyperlink{c_gnss}{G\+N\+S\+S} for A\+P\+I details.\hypertarget{how_to_g_n_s_s_howToGNSS_initialize}{}\subsection{Initialize}\label{how_to_g_n_s_s_howToGNSS_initialize}
This diagram shows how to initialize a G\+N\+S\+S device. The C\+O\+M\+P\+O\+N\+E\+N\+T\+\_\+\+I\+N\+I\+T initializes an event handler resulting in a ready state. Use \hyperlink{le__gnss__interface_8h_a8e1d96b1b64055b298a74cad1acfbbf8}{le\+\_\+gnss\+\_\+\+Enable()} and \hyperlink{le__gnss__interface_8h_add90639835a531c4b9d15554e4f3ba16}{le\+\_\+gnss\+\_\+\+Start()} to move to active state.



Also see \hyperlink{c_gnss_le_gnss_EnableDisable}{Enable/\+Disable G\+N\+S\+S device}





Copyright (C) Sierra Wireless Inc. Use of this work is subject to license. \hypertarget{howToPowerMgmt}{}\section{Manage Device Power}\label{howToPowerMgmt}
This topic summarizes how to manage device power. Information is provided on the Legato power management framework implementation with guidelines how to write power-\/efficient Legato apps.

See \hyperlink{basicPwrMgmt}{Power Management} for general info on managing device power.

Legato uses Linux power management techniques to minimize device power consumption\+:


\begin{DoxyItemize}
\item S\+U\+S\+P\+E\+N\+D\+: system saves its state in memory, places all peripherals in low power mode, and puts the app processor into its deepest idle state.
\item C\+P\+U\+I\+D\+L\+E\+: system estimates how long the app processor would be idle and puts it into appropriate low power state.
\item C\+P\+U\+F\+R\+E\+Q\+: system estimates how much the app processor is loaded and tunes the C\+P\+U frequency and voltage as necessary.
\end{DoxyItemize}

While C\+P\+U\+I\+D\+L\+E and C\+P\+U\+F\+R\+E\+Q seamlessly run in the O/\+S background, Legato apps have significant impact on the efficiency of system S\+U\+S\+P\+E\+N\+D functionality.\hypertarget{how_to_power_mgmt_howToPowerMgmt_wakeupSources}{}\subsection{Wakeup Sources}\label{how_to_power_mgmt_howToPowerMgmt_wakeupSources}
Legato uses Linux wakeup sources (called wake locks in Android) to control the system power state. If a component with appropriate privileges wants to prevent the system from entering S\+U\+S\+P\+E\+N\+D state, it acquires a wakeup source.

If a component with appropriate privileges has no interest in keeping the system powered, it releases a wakeup source. Linux auto-\/sleep component monitors the use of all system-\/wide wakeup sources and triggers entry into S\+U\+S\+P\+E\+N\+D state when none of the wakeup sources are held.

Wakeup sources can roughly be classified as\+:


\begin{DoxyItemize}
\item {\bfseries Kernel wakeup sources} are hard-\/coded in kernel components and can only be acquired and released by kernel, module and driver code.
\item {\bfseries User-\/space wakeup sources} are created on-\/demand, acquired and released by privileged app components. User apps write the wakeup source name to file {\ttfamily /sys/power/wake\+\_\+lock} to create and acquire wakeup source, and write the same name to file {\ttfamily /sys/power/wake\+\_\+unlock} to release the wakeup source.
\end{DoxyItemize}

The Legato Power Manager is the only Legato component with the privilege to write to {\ttfamily /sys/power/wake\+\_\+lock} and {\ttfamily /sys/power/wake\+\_\+unlock} files. Other components that need control over system power state must have the Linux C\+A\+P\+\_\+\+B\+L\+O\+C\+K\+\_\+\+S\+U\+S\+P\+E\+N\+D capability assigned, and perform the following Legato calls to request service from the Power Manager\+:

\hyperlink{le__pm__interface_8h_a85038248bcddc8963f7280ffa53acf62}{le\+\_\+pm\+\_\+\+New\+Wakeup\+Source()} create a wakeup source with a particular tag,

\hyperlink{le__pm__interface_8h_a6be9b6c443c506b0ce29da79e53e2534}{le\+\_\+pm\+\_\+\+Stay\+Awake()} acquire a wakeup source, and

\hyperlink{le__pm__interface_8h_a2ffb1fb0d165604226a9df35360080ea}{le\+\_\+pm\+\_\+\+Relax()} release a wakeup source.

When the Power Manager receives a request from another component via \hyperlink{le__pm__interface_8h_a85038248bcddc8963f7280ffa53acf62}{le\+\_\+pm\+\_\+\+New\+Wakeup\+Source()}, it\textquotesingle{}ll prefix the requested tag with string {\ttfamily legato\+\_\+} and post-\/fix it with the requestor’s process I\+D resulting in the wakeup source name format \begin{DoxyVerb}legato_<tag>_<process-id> \end{DoxyVerb}


This name format allows for easy traceability of wakeup sources. The process I\+D refers to a particular Legato app and the tag refers to a particular wakeup source within that app.

A wakeup source using this name is then created on behalf of the requesting component and a reference to the wakeup source is passed back to the requestor to use it in \hyperlink{le__pm__interface_8h_a6be9b6c443c506b0ce29da79e53e2534}{le\+\_\+pm\+\_\+\+Stay\+Awake()} and \hyperlink{le__pm__interface_8h_a2ffb1fb0d165604226a9df35360080ea}{le\+\_\+pm\+\_\+\+Relax()} calls.

Stale wakeup sources are automatically released by the Power Manager when their requestor exits and/or disconnects from the Power Manager.\hypertarget{how_to_power_mgmt_howToPowerMgmt_wakeupAndDeferrable}{}\subsection{Wakeup \& Deferrable Events}\label{how_to_power_mgmt_howToPowerMgmt_wakeupAndDeferrable}
Legato apps run in an event-\/driven model so the app remains idle until there\textquotesingle{}s an event to be processed. Once an event occurs, Legato detects it and dispatches it to one of its threads for processing.

From a power management aspect, there\textquotesingle{}s no reason for the system to stay powered while it\textquotesingle{}s idle. Also, events may not have the same processing importance, so Legato classifies them like this\+:


\begin{DoxyItemize}
\item {\bfseries Wakeup events}\+: require immediate attention so the system to must be powered.
\item {\bfseries Deferrable events}\+: can wait to be processed until the system is powered for another reason.
\end{DoxyItemize}

These event types are defined in the {\ttfamily fd\+Monitor} object. To classify the fd\+Monitor event as wakeup or deferrable, the app should call \hyperlink{le__fd_monitor_8h_a66a93ae01f1e6faf1d0c7645752d4442}{le\+\_\+fd\+Monitor\+\_\+\+Set\+Deferrable()} with the appropriate ‘is\+Deferrable’ flag. By default, if \hyperlink{le__fd_monitor_8h_a66a93ae01f1e6faf1d0c7645752d4442}{le\+\_\+fd\+Monitor\+\_\+\+Set\+Deferrable()} is not called; all events on that object will be assumed to be wakeup events.

The underlying Linux mechanism of waiting for wakeup events from {\ttfamily fd\+Monitor’s} file descriptor uses epoll\+\_\+wait() in conjunction with {\ttfamily E\+P\+O\+L\+L\+W\+A\+K\+E\+U\+P} flag. When an epoll\+\_\+wait() event occurs on a file descriptor that has this flag is set, epoll\+\_\+wait() will unblock the caller and an {\bfseries eventpoll} kernel wakeup source will be signaled. This causes the system to stay awake until one of these conditions are met\+:


\begin{DoxyItemize}
\item another epoll\+\_\+wait() call is made with the same file descriptor in the event list.
\end{DoxyItemize}

or


\begin{DoxyItemize}
\item the file descriptor is closed.
\end{DoxyItemize}

This way, callbacks for this {\ttfamily fd\+Monitor} event will be executed while {\bfseries eventpoll} wakeup source is held. This guarantees the system stays awake until all callbacks are executed. If a Legato component needs to keep the system powered beyond the scope of a Legato callback, it should acquire its own wakeup source.

This diagram shows a simple Legato app with process I\+D 25 that needs to keep the system powered between two Legato events using wakeup source tagged {\bfseries lock}. Wakeup sources are handed off to achieve this functionality. The system remains powered from the moment the first event occurs until the second event is fully processed.



App components that subscribe to wakeup events must have C\+A\+P\+\_\+\+B\+L\+O\+C\+K\+\_\+\+S\+U\+S\+P\+E\+N\+D Linux capability assigned. If not, all events will be assumed deferrable and calling \hyperlink{le__fd_monitor_8h_a66a93ae01f1e6faf1d0c7645752d4442}{le\+\_\+fd\+Monitor\+\_\+\+Set\+Deferrable()} will have no effect.\hypertarget{how_to_power_mgmt_howToPowerMgmt_clients}{}\subsection{Power Manager Clients}\label{how_to_power_mgmt_howToPowerMgmt_clients}
Using wakeup sources and E\+P\+O\+L\+L\+W\+A\+K\+E\+U\+P provides a simple programming model to write Power Manager clients. Here are some guidelines\+:


\begin{DoxyItemize}
\item Make sure that C\+A\+P\+\_\+\+B\+L\+O\+C\+K\+\_\+\+S\+U\+S\+P\+E\+N\+D capability is assigned to the client. Client components usually run sandboxed and unless this capability is not explicitly assigned, they will not be able to control system power state.
\item Don\textquotesingle{}t acquire wakeup sources if you just need to process an event callback. In this callback, the system will stay powered due to the {\bfseries eventpoll} wakeup source.
\item Acquire a wakeup source if system processing occurs outside a Legato callback. A typical use case is e.\+g. composing and sending a text message\+: a Legato callback won\textquotesingle{}t be invoked until the message is submitted and hence the message composition operation may be interrupted by system sleep if unprotected by a wakeup source.
\item Acquire a wakeup source if the system needs to stay awake monitoring a state transition. State transition events may typically be handled in a callback, but if the system needs to stay powered across multiple states, a wakeup source is required.
\end{DoxyItemize}\hypertarget{how_to_power_mgmt_howToPowerMgmt_troubleshooting}{}\subsection{Troubleshooting}\label{how_to_power_mgmt_howToPowerMgmt_troubleshooting}
\paragraph*{Problem }

{\bfseries System doesn\textquotesingle{}t suspend. }

{\itshape Probable} {\itshape cause} 

A wakeup source is being held ~\newline


{\itshape Test} 

Dump contents of /sys/kernel/debug/wakeup\+\_\+sources to find the fields that have a non-\/zero active\+\_\+since field. Also dump contents of {\ttfamily /sys/power/wake\+\_\+lock} to find active user-\/space wakeup sources. ~\newline


{\itshape Command} 

\begin{DoxyVerb} cat /sys/kernel/debug/wakeup_sources |sed -e s/"^       "/"unnamed"/ | awk '{print $6 "\t" $1}' | grep -v "^0" |sort –n
 cat /sys/power/wake_lock
\end{DoxyVerb}


\paragraph*{Problem }

{\bfseries System perpetually suspends and resumes. }

{\itshape Probable} {\itshape cause} 

Interrupt is constantly triggered.

{\itshape Test} 

Dump contents of {\ttfamily /proc/interrupts} to find the I\+R\+Q that\textquotesingle{}s constantly incrementing.

{\itshape Command} 

{\ttfamily cat} /proc/interrupts

\paragraph*{Problem }

{\bfseries System doesn\textquotesingle{}t resume. }

{\itshape Probable} {\itshape cause} 

Wakeup interrupts aren\textquotesingle{}t configured.

{\itshape Test} 

Dump contents of all\begin{DoxyVerb}/sys/devices/*/power/wakeup \end{DoxyVerb}
 files and check which devices have wakeup interrupts enabled.

{\itshape Command} 

\begin{DoxyVerb}find /sys/devices –name wakeup –exec cat “{}” “;” -print \end{DoxyVerb}






Copyright (C) Sierra Wireless Inc. Use of this work is subject to license. \hypertarget{howToNMEA}{}\section{N\+M\+E\+A Port Setup}\label{howToNMEA}
This topic summarizes how to setup an N\+M\+E\+A port on the A\+R755x platform.\hypertarget{how_to_n_m_e_a_howToNMEA_setup}{}\subsection{Setup Target}\label{how_to_n_m_e_a_howToNMEA_setup}
To setup an A\+R755x target, run the follow commands (only once)\+:


\begin{DoxyCode}
AT!ENTERCND=\textcolor{stringliteral}{"A710"}
AT!CUSTOM=\textcolor{stringliteral}{"GPSENABLE"},1
AT!CUSTOM=\textcolor{stringliteral}{"NMEAENABLE"},1
AT!CUSTOM=\textcolor{stringliteral}{"GPSSEL"},1
AT!GPSQMICONFIG=1
AT!GPSNMEA=1
AT!GPSNMEACONFIG=1,1
AT!GPSNMEASENTENCE=29FF
AT!GPSONLY=0
AT!RESET
\end{DoxyCode}
\hypertarget{how_to_n_m_e_a_howToNMEA_startNMEA}{}\subsection{Start N\+M\+E\+A}\label{how_to_n_m_e_a_howToNMEA_startNMEA}
Once, the target is setup, run this on the A\+R755x Linux console\+:


\begin{DoxyCode}
 start NMEA device:
echo \textcolor{stringliteral}{'$GPS\_START'} > /dev/nmea
\end{DoxyCode}


Then \char`\"{}/dev/nmea\char`\"{} file can be open in order to parse the N\+M\+E\+A frames.

\begin{DoxyWarning}{Warning}
If you need to use Legato positioning service in parallel, you can\textquotesingle{}t use this command; you have to use \hyperlink{le__pos_ctrl__interface_8h_ab0522cfb23a7b34863b7bd9475d38255}{le\+\_\+pos\+Ctrl\+\_\+\+Request()}.
\end{DoxyWarning}
\hypertarget{how_to_n_m_e_a_howToNMEA_stopNMEA}{}\subsection{Stop N\+M\+E\+A}\label{how_to_n_m_e_a_howToNMEA_stopNMEA}
To stop the N\+M\+E\+A, run this on the A\+R755x Linux console\+: 
\begin{DoxyCode}
echo \textcolor{stringliteral}{'$GPS\_STOP'} > /dev/nmea
\end{DoxyCode}


\begin{DoxyWarning}{Warning}
If you need to use Legato positioning service in parallel, you can\textquotesingle{}t use this command; you have to use \hyperlink{le__pos_ctrl__interface_8h_a5dfa743e5d134b265b883f7106846428}{le\+\_\+pos\+Ctrl\+\_\+\+Release()}.
\end{DoxyWarning}
\hypertarget{how_to_n_m_e_a_howToNMEA_getNMEA}{}\subsection{Get N\+M\+E\+A stream}\label{how_to_n_m_e_a_howToNMEA_getNMEA}
Here\textquotesingle{}s an example of a dump of G\+N\+S\+S N\+M\+E\+A frames (here only \$\+G\+Nxxx frames)\+: 
\begin{DoxyCode}
echo \textcolor{stringliteral}{'$GPS\_START'} > /dev/nmea
cat /dev/nmea | grep \textcolor{stringliteral}{'$GN'}

$GNGNS,094821.0,4849.931307,N,00216.053323,E,AA,14,0.6,161.5,48.0,,*6D
$GNGSA,A,2,05,13,18,21,27,29,31,,,,,,1.0,0.6,0.7*2C
$GNGSA,A,2,68,69,74,75,76,84,86,,,,,,1.0,0.6,0.7*2E
$GNGNS,094822.0,4849.931277,N,00216.053326,E,AA,15,0.6,161.5,48.0,,*6C
$GNGSA,A,2,05,13,16,18,21,27,29,31,,,,,0.9,0.6,0.7*23
$GNGSA,A,2,68,69,74,75,76,84,86,,,,,,0.9,0.6,0.7*26
$GNGNS,094823.0,4849.931249,N,00216.053330,E,AA,15,0.6,161.5,48.0,,*67
$GNGSA,A,2,05,13,16,18,21,27,29,31,,,,,0.9,0.6,0.7*23
$GNGSA,A,2,68,69,74,75,76,84,86,,,,,,0.9,0.6,0.7*26
$GNGNS,094824.0,4849.931225,N,00216.053333,E,AA,15,0.6,161.5,48.0,,*69
$GNGSA,A,2,05,13,16,18,21,27,29,31,,,,,0.9,0.6,0.7*23
$GNGSA,A,2,68,69,74,75,76,84,86,,,,,,0.9,0.6,0.7*26


Decoding...

$GNGNS: GNSS Fix data
http:\textcolor{comment}{//www.catb.org/gpsd/NMEA.html#\_gns\_fix\_data}

$GNGSA: GNSS Active Satellites + Dilution of precision
http:\textcolor{comment}{//www.catb.org/gpsd/NMEA.html#\_gsa\_gps\_dop\_and\_active\_satellites}

2D-Fix with:
- 8 GPS Satellites: 05,13,16,18,21,27,29,31
- 7 Glonass Satellites: 68,69,74,75,76,84,86

Time: 094824.0 --> 09:48:24 (UTC)
Latitude: 4849.931225,N --> 48 + (49.931225 / 60) = 48.83218708333333
Longitude: 00216.053333,E  --> 002 + (16.053333 / 60) = 2.26755555
https:\textcolor{comment}{//www.google.fr/maps/@48.8321871,2.2675556,17z​}
\end{DoxyCode}






Copyright (C) Sierra Wireless Inc. Use of this work is subject to license. \hypertarget{howToPortLegacyGen}{}\section{Port Legacy Apps}\label{howToPortLegacyGen}
This topic provides general info on porting legacy apps to Legato.

There\textquotesingle{}s also specific info available to \hyperlink{howToPortLegacyC}{Port Legacy C App}.\hypertarget{how_to_port_legacy_gen_howToPortLegacyGen_Sandboxed}{}\subsection{Sandboxed App}\label{how_to_port_legacy_gen_howToPortLegacyGen_Sandboxed}
High-\/level steps for {\bfseries porting your app to Legato}\+:


\begin{DoxyItemize}
\item Build the app’s executables and libraries using the legacy program’s build system, but using the appropriate cross-\/build tool chain.
\item Create an \hyperlink{defFilesAdef}{.adef file} for your app.
\item Leave the {\ttfamily executables\+:} section in {\ttfamily  .adef } empty (or omit that section entirely).
\item Include files to be installed as part of the app (executables, libraries, configuration files, devices, etc.) in the {\ttfamily files\+:} subsection of the \hyperlink{def_files_adef_defFilesAdef_bundles}{bundles\+: section of the .adef file}.
\item Use the \hyperlink{def_files_adef_defFilesAdef_requiresFile}{extern\+: section of the .adef file} to include files and directories that need to be imported into the sandbox from the target’s file system.
\item Add \hyperlink{def_files_adef_defFilesAdef_processRun}{run\+: lines in the {\ttfamily processes}\+: section in the .adef} to define processes that should run in the app.
\item Run \hyperlink{buildToolsmkapp}{mkapp} on the {\ttfamily  .adef } file to create the app bundle (e.\+g., .wp85 or .wp7 file) to be installed on the target.
\end{DoxyItemize}

An app bundle produced like this can be installed exactly the same as any other Legato app bundle.

Detailed info\+:

\hyperlink{defFiles}{Definition Files} ~\newline
 \hyperlink{basicAppsCreate}{Create Apps} ~\newline
 \hyperlink{howToPortLegacyC}{Port Legacy C App}

Here\textquotesingle{}s a {\ttfamily foo.\+adef} sample (for application foo) with executables bar1 and bar2 needing library lib1. The application name should be the same as the .adef file name\+:


\begin{DoxyCode}
bundles:
\{
    file:
    \{
        [x] bar1    /bin/
        [x] bar2    /bin/
        [r] lib1    /lib/
    \}
\}

processes:
\{
    run:
    \{
        (bar1)
        (bar2)
    \}
\}
\end{DoxyCode}


To bundle everything into an app, run\+: 
\begin{DoxyCode}
mkapp foo.adef -t [target] 
\end{DoxyCode}
 where \mbox{[}target\mbox{]} is something like \char`\"{}ar7\char`\"{} or \char`\"{}wp85\char`\"{}.

The app bundle file can be installed using the {\ttfamily instapp} tool.

\begin{DoxyNote}{Note}
Other app settings (e.\+g., {\ttfamily fault\+Action}) can also be set in the .adef files. See \hyperlink{defFilesAdef}{Application Definition .adef}.
\end{DoxyNote}
\hypertarget{how_to_port_legacy_gen_howToPortLegacyGen_portRootAccess}{}\subsection{Porting Apps with Root Access}\label{how_to_port_legacy_gen_howToPortLegacyGen_portRootAccess}
This method is necessary if your legacy apps require root privileges or access to system resources like {\ttfamily /proc}. It means your app will be ported as a {\bfseries non-\/sandboxed} Legato app.

In the .adef file, turn-\/off sandboxing\+:


\begin{DoxyCode}
sandboxed: \textcolor{keyword}{false}
\end{DoxyCode}


\begin{DoxyNote}{Note}
It can sometimes be easier to begin porting an application with sandboxing turned off and, when you have it working, turn sandboxing on and fix the resulting permissions issues by importing required files into your app using \hyperlink{def_files_adef_defFilesAdef_requires}{the .adef requires section}.
\end{DoxyNote}




Copyright (C) Sierra Wireless Inc. Use of this work is subject to license. \hypertarget{howToPortLegacyC}{}\subsection{Port Legacy C App}\label{howToPortLegacyC}
This topic describes how to get a P\+O\+S\+I\+X/\+Linux legacy app written in C running on a Legato device and using Legato A\+P\+Is to access services like S\+M\+S, S\+I\+M, voice calling, and data connections.

\begin{DoxyNote}{Note}
The examples in this topic all use the command-\/line tools, so you will need to have your shell configured correctly by running {\ttfamily  bin/legs } in the directory in which your framework is installed. 
\begin{DoxyCode}
$ path/to/LegatoFramework/bin/legs
\end{DoxyCode}


These examples assume you are using a Sierra Wireless W\+P85xx device. If you\textquotesingle{}re using another device, substitute that device wherever you see \char`\"{}\+W\+P85\char`\"{} or \char`\"{}wp85\char`\"{}.
\end{DoxyNote}
\hypertarget{how_to_port_legacy_c_howtoPortingLegacyC_CrossBuild}{}\subsubsection{Cross-\/\+Build}\label{how_to_port_legacy_c_howtoPortingLegacyC_CrossBuild}
The most basic way to get your legacy app running on a Legato target device is to recompile it using the provided cross-\/build tool chain and copy it onto the device using a tool like {\ttfamily scp}.

{\ttfamily 1}. Build a legacy app executable for your target device using the cross tool chain provided.


\begin{DoxyCode}
$ $WP85\_TOOLCHAIN\_DIR/arm-poky-linux-gnueabi-gcc -o legacyProgram main.c
\end{DoxyCode}


{\ttfamily 2}. Copy the legacy app executable onto the target using a tool like {\ttfamily scp\+:} 


\begin{DoxyCode}
$ scp legacyProgram 192.168.1.2:

legacyProgram                100% 9366     9.2KB/s   00:00
\end{DoxyCode}


{\ttfamily 3}. Run the legacy app from the target command-\/line\+:


\begin{DoxyCode}
root@swi-mdm9x15:~# ./legacyProgram
Hello world.
\end{DoxyCode}
\hypertarget{how_to_port_legacy_c_howtoPortingLegacyC_UseLegatoAppManagementTools}{}\subsubsection{Use Legato App Management Tools}\label{how_to_port_legacy_c_howtoPortingLegacyC_UseLegatoAppManagementTools}
By bundling your program as a Legato app, you gain access to a wealth of valuable features\+:
\begin{DoxyItemize}
\item Tools for installing and removing apps and checking app status on the target and on the development host.
\item Remote (over-\/the-\/air) installation, upgrade, removal, start, stop.
\item Autonomous fault recovery (automatic restart of process, whole app, or whole device) in the field.
\item Automatic mandatory access control (M\+A\+C) configuration.
\item Optional application sandboxing.
\item Optional application signing and/or encryption.
\end{DoxyItemize}

{\ttfamily 1}. Create a {\ttfamily  .adef } file (e.\+g., {\ttfamily legacy\+Program.\+adef}) that bundles the cross-\/compiled executable into an application\+:


\begin{DoxyCode}
\textcolor{comment}{// Disable the sandbox security to make things a little easier.}
sandboxed: \textcolor{keyword}{false}

\textcolor{comment}{// Put the cross-compiled legacy program in the app's bin directory.}
\textcolor{comment}{// [x] = make it executable.}
bundles:
\{
    file:
    \{
        [x] legacyProgram /bin/
    \}
\}

\textcolor{comment}{// Tell the Supervisor to start this program when the application is started.}
processes:
\{
    run:
    \{
        ( legacyProgram )
    \}
\}
\end{DoxyCode}


{\ttfamily 2}. Run {\ttfamily mkapp} to generate an application bundle for your target\+:


\begin{DoxyCode}
$ mkapp -t wp85 legacyProgram.adef
\end{DoxyCode}


{\ttfamily 3}. Install the app bundle on the target using {\ttfamily instapp\+:} 


\begin{DoxyCode}
$ instapp legacyProgram.wp85
Installing application \textcolor{stringliteral}{'legacyProgram'} from file \textcolor{stringliteral}{'legacyProgram.wp85'}.
Installing app \textcolor{stringliteral}{'legacyProgram'}...
Created user \textcolor{stringliteral}{'applegacyProgram'} (uid 1011, gid 1011).
DONE
\end{DoxyCode}


{\ttfamily 4}. From the target\textquotesingle{}s command line, use {\ttfamily app start} to run the program\+:


\begin{DoxyCode}
$ ssh root@192.168.1.2
Linux swi-mdm9x15 3.4.91-8fcd3d08ac\_7e84772e18 #1 PREEMPT Wed Jun 3 23:59:46 PDT 2015 armv7l GNU/Linux
root@swi-mdm9x15:~# app start legacyProgram
Starting app \textcolor{stringliteral}{'legacyProgram'}...
DONE
\end{DoxyCode}


{\ttfamily 5}. Look for the program output in the target device\textquotesingle{}s log using {\ttfamily logread}.

\begin{DoxyNote}{Note}
You can filter the log to show just your program\textquotesingle{}s output by piping the output from {\ttfamily logread} into {\ttfamily grep}.
\end{DoxyNote}

\begin{DoxyCode}
root@swi-mdm9x15:~# logread | grep legacyProgram
Jan 16 04:00:53 swi-mdm9x15 user.info Legato:  INFO | legacyProgram[27271] | Hello world.
\end{DoxyCode}
\hypertarget{how_to_port_legacy_c_howtoPortingLegacyC_useLegatoSvcs}{}\subsubsection{Use Legato Services}\label{how_to_port_legacy_c_howtoPortingLegacyC_useLegatoSvcs}
Many Legato services are provided through I\+P\+C-\/based A\+P\+Is. The {\ttfamily ifgen} tool can generate the I\+P\+C code for you, along with a header (.h) file that you can {\ttfamily \#include} to gain access to the service.

Here is how to use a Legato modem service A\+P\+I (e.\+g., le\+\_\+info). The source code for this example can be found in {\ttfamily  apps/sample/legacy/use\+Legato\+Api/ }.

{\ttfamily 1}. Run {\ttfamily ifgen} to generate the .c and .h files you need to access the interface.


\begin{DoxyItemize}
\item Use the {\ttfamily --gen-\/interface} option to generate the interface header ({\ttfamily \hyperlink{le__info__interface_8h}{le\+\_\+info\+\_\+interface.\+h}}).
\item Use the {\ttfamily --gen-\/client} option to generate the client-\/side I\+P\+C implementation ({\ttfamily le\+\_\+info\+\_\+client.\+c}).
\item Use the {\ttfamily --gen-\/local} option to generate definitions that are shared by both the client side and server side I\+P\+C code ({\ttfamily le\+\_\+info\+\_\+local.\+h}).
\end{DoxyItemize}

\begin{DoxyVerb}ifgen --gen-interface --gen-client --gen-local $LEGATO_ROOT/interfaces/modemServices/le_info.api
\end{DoxyVerb}


{\ttfamily 2}. Include {\ttfamily \hyperlink{legato_8h}{legato.\+h}} in your program.


\begin{DoxyCode}
\textcolor{preprocessor}{#include "\hyperlink{legato_8h}{legato.h}"}
\end{DoxyCode}


{\ttfamily 3}. Include the A\+P\+I\textquotesingle{}s generated \char`\"{}interface\char`\"{} header file.


\begin{DoxyCode}
\textcolor{preprocessor}{#include "\hyperlink{le__info__interface_8h}{le\_info\_interface.h}"}
\end{DoxyCode}


{\ttfamily 4}. Connect to the service by calling \hyperlink{le__info__interface_8h_ae1a4655ecfee5f91a69c772d204b569d}{le\+\_\+info\+\_\+\+Connect\+Service()} (using legacy main function).


\begin{DoxyCode}
\textcolor{keywordtype}{int} main(\textcolor{keywordtype}{int} argc, \textcolor{keywordtype}{char}** argv)
\{
    \hyperlink{le__info__interface_8h_ae1a4655ecfee5f91a69c772d204b569d}{le\_info\_ConnectService}();

    \textcolor{keywordflow}{return} EXIT\_SUCCESS;
\}
\end{DoxyCode}


\begin{DoxyNote}{Note}
At runtime, if the {\ttfamily le\+\_\+info} service isn\textquotesingle{}t available, this will block until it becomes available. In the meantime, you\textquotesingle{}ll see your app in the W\+A\+I\+T\+I\+N\+G C\+L\+I\+E\+N\+T\+S list if you run \hyperlink{toolsTarget_sdir}{sdir list}.
\end{DoxyNote}
{\ttfamily 5}. Add a call to one of the {\ttfamily le\+\_\+info} A\+P\+I functions (e.\+g., \hyperlink{le__info__interface_8h_a4cd7a99fddb014e25880e6354b1d02f8}{le\+\_\+info\+\_\+\+Get\+Device\+Model()} ).


\begin{DoxyCode}
\textcolor{keywordtype}{int} main(\textcolor{keywordtype}{int} argc, \textcolor{keywordtype}{char}** argv)
\{
    \hyperlink{le__info__interface_8h_ae1a4655ecfee5f91a69c772d204b569d}{le\_info\_ConnectService}();

    \textcolor{keywordtype}{char} deviceModelStr[256];

    \hyperlink{le__basics_8h_a1cca095ed6ebab24b57a636382a6c86c}{le\_result\_t} result = \hyperlink{le__info__interface_8h_a4cd7a99fddb014e25880e6354b1d02f8}{le\_info\_GetDeviceModel}(deviceModelStr, \textcolor{keyword}{sizeof}(
      deviceModelStr));

    \textcolor{keywordflow}{if} (result == \hyperlink{le__basics_8h_a1cca095ed6ebab24b57a636382a6c86ca5066a4bcec691c6b67843b8f79656422}{LE\_OK})
    \{
        printf(\textcolor{stringliteral}{"Hello world from %s.\(\backslash\)n"}, deviceModelStr);
    \}
    \textcolor{keywordflow}{else}
    \{
        printf(\textcolor{stringliteral}{"Failed to get device model. Error = '%s'.\(\backslash\)n"}, \hyperlink{le__log_8h_a99402d6a983f318e5b8bfcdf5dfe9024}{LE\_RESULT\_TXT}(result));
    \}

    \textcolor{keywordflow}{return} EXIT\_SUCCESS;
\}
\end{DoxyCode}


{\ttfamily 6}. Compile and link your executable with the code generated by {\ttfamily ifgen\+:} 
\begin{DoxyCode}
$ export CC=$WP85\_TOOLCHAIN\_DIR/arm-poky-linux-gnueabi-gcc
$ $CC -c main.c -I$LEGATO\_ROOT/framework/c/inc
$ $CC -c le\_info\_client.c -I$LEGATO\_ROOT/framework/c/inc
$ $CC -o legacyProgram main.o le\_info\_client.o -L$\{LEGATO\_ROOT\}/build/$\{TARGET\}/bin/lib -llegato -lpthread 
      -lrt
\end{DoxyCode}


{\ttfamily 7}. Specify which instance of the {\ttfamily le\+\_\+info} service your app should use by creating a binding in the {\ttfamily  .adef } file\+:


\begin{DoxyCode}
bindings:
\{
     .le\_info -> modemService.le\_info
\}
\end{DoxyCode}


\begin{DoxyNote}{Note}
Actually, there\textquotesingle{}s only one instance of {\ttfamily le\+\_\+info} today, but if there were multiple, this would specify which one to use; and even when there\textquotesingle{}s only one instance, we create a binding anyway to explicitly grant access permission so access is never unknowingly granted.
\end{DoxyNote}
{\ttfamily 9}. Re-\/generate your application bundle, install it, and run it on target\+:


\begin{DoxyCode}
$ mkapp -t wp85 legacyProgram.adef
$ instapp legacyProgram.wp85 192.168.1.2
Installing application \textcolor{stringliteral}{'legacyProgram'} from file \textcolor{stringliteral}{'legacyProgram.wp85'}.
Removing app \textcolor{stringliteral}{'legacyProgram'}...
Deleted user \textcolor{stringliteral}{'applegacyProgram'}.
Installing app \textcolor{stringliteral}{'legacyProgram'}...
Created user \textcolor{stringliteral}{'applegacyProgram'} (uid 1011, gid 1011).
DONE
$ startapp legacyProgram 192.168.1.2
\end{DoxyCode}
\hypertarget{how_to_port_legacy_c_howtoPortingLegacyC_handlers}{}\subsubsection{Callbacks from Legato A\+P\+Is}\label{how_to_port_legacy_c_howtoPortingLegacyC_handlers}
If you need asynchronous callbacks (i.\+e., handlers), you\textquotesingle{}ll need to service the Legato event loop for your thread. To do this, use \hyperlink{le__event_loop_8h_a12ce7f92f4bc6f5167d5a6ef86d7d0b1}{le\+\_\+event\+\_\+\+Get\+Fd()} and \hyperlink{le__event_loop_8h_a096222e98f6a0d92a79722018a752b58}{le\+\_\+event\+\_\+\+Service\+Loop()}. See \hyperlink{c_event_loop_c_event_integratingLegacyPosix}{Integrating with Legacy P\+O\+S\+I\+X Code} for more details.

The sample app for this is found in {\ttfamily apps/sample/legacy/use\+Legato\+Handler}.

Here\textquotesingle{}s some sample code\+:


\begin{DoxyCode}
\textcolor{keyword}{struct }pollfd pollControl;
pollControl.fd = \hyperlink{le__event_loop_8h_a12ce7f92f4bc6f5167d5a6ef86d7d0b1}{le\_event\_GetFd}();
pollControl.events = POLLIN;

\textcolor{keywordflow}{while} (\textcolor{keyword}{true})
\{
    \textcolor{keywordtype}{int} result = poll(&pollControl, 1, -1);

    \textcolor{keywordflow}{if} (result > 0)
    \{
        \textcolor{keywordflow}{while} (\hyperlink{le__event_loop_8h_a096222e98f6a0d92a79722018a752b58}{le\_event\_ServiceLoop}() == \hyperlink{le__basics_8h_a1cca095ed6ebab24b57a636382a6c86ca5066a4bcec691c6b67843b8f79656422}{LE\_OK})
        \{
            \textcolor{comment}{/* Work was done by le\_event\_ServiceLoop(), and it has more to do.  */}
        \}
    \}
    \textcolor{keywordflow}{else}
    \{
        \textcolor{comment}{// Poll failed.  You could check for zero if you're ultra paranoid,}
        \textcolor{comment}{// but poll should never return zero when timeout is -1.}
        \hyperlink{le__log_8h_a54b4b07f5396e19a8d9fca74238f4795}{LE\_FATAL}(\textcolor{stringliteral}{"poll() failed with errno %m."});
    \}
\}
\end{DoxyCode}
\hypertarget{how_to_port_legacy_c_howtoPortingLegacyC_sandboxing}{}\subsubsection{Sandboxing Your App}\label{how_to_port_legacy_c_howtoPortingLegacyC_sandboxing}
To tell the Supervisor to run your app inside a sandbox, remove the following line from your app\textquotesingle{}s {\ttfamily  .adef } file\+:


\begin{DoxyCode}
sandboxed: \textcolor{keyword}{false}
\end{DoxyCode}


Or, you can change {\ttfamily false} to {\ttfamily true\+:} 


\begin{DoxyCode}
sandboxed: \textcolor{keyword}{true}
\end{DoxyCode}


Then re-\/bundle your app using {\ttfamily mkapp}.

The most commonly-\/used system libraries, such as {\ttfamily libc} and {\ttfamily libpthread}, will be visible inside your app\textquotesingle{}s sandbox by default, but you may now find that your app won\textquotesingle{}t run because some other files are missing from its sandbox.

Use the \hyperlink{def_files_adef_defFilesAdef_requires}{requires\+: section in the app\textquotesingle{}s .adef file} to add things to the sandbox.\hypertarget{how_to_port_legacy_c_howtoPortingLegacyC_sampleApps}{}\subsubsection{Sample Legacy Apps}\label{how_to_port_legacy_c_howtoPortingLegacyC_sampleApps}
Sample \hyperlink{sampleApps_legacy}{Legacy C} apps are available in the {\ttfamily Legato/apps/sample/legacy} directory.





Copyright (C) Sierra Wireless Inc. Use of this work is subject to license. \hypertarget{howToSetUserTimebase}{}\section{Set User Timebase}\label{howToSetUserTimebase}
This topic provides details on setting the user timebase. Usually, the automated time daemon handles time synchronization, but that may not always be desirable.

See \hyperlink{c_rtc}{User Timebase} A\+P\+I





You never directly modify the R\+T\+C. Instead, a user timebase parses the time values needed to adjust the R\+T\+C (plus/minus) so it\textquotesingle{}s synchronized with the system clock.

This is how time synchronization works in Legato\+:

\hypertarget{how_to_set_user_timebase_howToSetUserTimebase_overview}{}\subsection{Overview}\label{how_to_set_user_timebase_howToSetUserTimebase_overview}
The modem manages a real time clock (battery backed clock V\+C\+O\+I\+N) that keeps time when the power is removed from the Legato module (if clock battery available). This clock is used by the modem to maintain several timebases derived from different domains (e.\+g., cellular or G\+P\+S). Apps shouldn\textquotesingle{}t directly change the value of the real time clock as it may have undesirable effects on modem function.

The user timebase is provided to synchronize with the R\+T\+C. An arbitrary stored time will increment on a millisecond basis and can be retrieved later assuming a battery (V\+C\+O\+I\+N) is available to keep the R\+T\+C running.\hypertarget{how_to_set_user_timebase_howToSetUserTimebase_getandset}{}\subsection{Get/\+Set User Timebase}\label{how_to_set_user_timebase_howToSetUserTimebase_getandset}
If you need to set the user timebase, you need to \hyperlink{c_rtc_c_rtc_disableDaemon}{Disable Time Daemon} and then \hyperlink{c_rtc_c_rtc_getSet}{Get/\+Set Time Value}.

See \hyperlink{c_rtc}{User Timebase} and \hyperlink{c_clock}{System Clock A\+P\+I}.\hypertarget{how_to_set_user_timebase_howToSetUserTimebase_background}{}\subsection{Background}\label{how_to_set_user_timebase_howToSetUserTimebase_background}
The R\+T\+C counts milliseconds since the G\+P\+S Epoch (January 6, 1980) and increments on each count so that each second in G\+P\+S time has a unique number. Unix time starts January 1, 1970, and has a fixed number of seconds per day. In Unix, days that have an extra second in them (i.\+e., leap second) use the same second twice. This means that over time an ever increasing number of leap seconds have to be accounted for when converting time between domains.

Leap seconds are introduced periodically because astronomically measured time and clock time would drift apart unless adjustments are made. It also means that Unix time, G\+P\+S time, and clock time are all drifting apart as well. The rate of drift is low (only 26 leap seconds have been added since leap seconds were introduced in 1972). It\textquotesingle{}s impossible to predict in which years leap seconds will be added so that calculation must be done retroactively, based on observation.

For practical purposes, it\textquotesingle{}s not necessary to convert the time you wish to maintain into real G\+P\+S time; it will suffice to use Unix epoch time. The amount of drift caused by leap seconds over a year is less than the inherent drift in the R\+T\+C. Under normal conditions, the time will be corrected from an external source like network time or G\+P\+S.



 \hypertarget{howToAV}{}\section{Use Air\+Vantage}\label{howToAV}
These topics describe how to use Air\+Vantage for remote administration of your Legato apps\+:

\hyperlink{howToAVConnect}{Connect to Air\+Vantage} ~\newline
 \hyperlink{howToAVInstallApp}{Install apps} ~\newline
 \hyperlink{howToAVData}{Manage Air\+Vantage Data}





Copyright (C) Sierra Wireless Inc. Use of this work is subject to license. \hypertarget{howToAVConnect}{}\subsection{Connect to Air\+Vantage}\label{howToAVConnect}
Also see \hyperlink{appInstallAirVantage}{Install apps}





This topic covers how to create an Air\+Vantage connection to your Legato device.

Detailed Air\+Vantage documentation\+: \href{https://doc.airvantage.net/display/USERGUIDE/Getting+Started}{\tt Air\+Vantage Getting Started}

\begin{DoxyNote}{Note}
You\textquotesingle{}ll need an account on Air\+Vantage Server, you can check with your Sierra Wireless contact.
\end{DoxyNote}
After you connect to your Air\+Vantage account, you can create system and app.\hypertarget{how_to_a_v_connect_connectingAirVantage_createSys}{}\subsubsection{Create System}\label{how_to_a_v_connect_connectingAirVantage_createSys}
Create a new System in Air\+Vantage Server\+:
\begin{DoxyItemize}
\item From the {\ttfamily Inventory menu}, choose {\ttfamily Systems}.
\item Click the Create button (+ symbol)\+: enter a Name (used for retrieval).
\item In the {\ttfamily Gateway} tab, click Create a New Gateway (+ symbol), a new windows will appear\+: enter your device\textquotesingle{}s I\+M\+E\+I and Serial Number.
\item Click {\ttfamily Create}.
\item After you return to the system creation window, click the Application field and select the ‘\+Legato\+\_\+\+L\+W\+M2\+M’ (firmware and application update over the air) application model. An application model is an X\+M\+L file defining the communication contract between Air\+Vantage and the embedded app.
\item Select \char`\"{}\+I want to activate my system after creating it\char`\"{}.
\item Click {\ttfamily Create}.
\end{DoxyItemize}\hypertarget{how_to_a_v_connect_connectingAirVantage_lwm2m}{}\subsubsection{L\+W\+M2\+M}\label{how_to_a_v_connect_connectingAirVantage_lwm2m}
Legato provides L\+W\+M2\+M support to connect to Air\+Vantage.

See \hyperlink{c_le_avc}{Air\+Vantage Connector}, \hyperlink{c_le_avdata}{Air\+Vantage Data} and \hyperlink{def_files_cdef_defFilesCdef_assets}{Assets} for details.\hypertarget{how_to_a_v_connect_connectingAirVantage_checkDeviceStatus}{}\subsubsection{Device Status}\label{how_to_a_v_connect_connectingAirVantage_checkDeviceStatus}
Check device status on the Air\+Vantage Server\+:


\begin{DoxyItemize}
\item From the {\ttfamily Monitor} menu, choose {\ttfamily Systems}
\item Browse to your device (by the name you created or the I\+M\+E\+I).
\item Click Details.
\end{DoxyItemize}

The {\ttfamily Last seen} field should be updated with a recent date. You can also see the initial data exchange in the Timeline screen.





Copyright (C) Sierra Wireless Inc. Use of this work is subject to license. \hypertarget{appInstallAirVantage}{}\subsection{Install apps}\label{appInstallAirVantage}
This page explains how to build your app, configure the Air\+Vantage platform, and then deploy your app remotely to your device.\hypertarget{how_to_a_v_install_app_appInstallAirVantage_buildapp}{}\paragraph{Build app}\label{how_to_a_v_install_app_appInstallAirVantage_buildapp}
Before you build your target app, ensure Legato is built and your shell environment is setup to use command-\/line tools (i.\+e., run {\ttfamily bin/legs}).

Then, run the following sample on your build host replacing {\ttfamily \$\{T\+A\+R\+G\+E\+T\+\_\+\+T\+Y\+P\+E\}} with your target device type (e.\+g., {\ttfamily wp85} or {\ttfamily ar7})\+:

\begin{DoxyVerb}Legato $ cd apps/sample/helloWorld
Legato $ mkapp -t ${TARGET_TYPE} helloWorld.adef
Legato $ update-pack ${TARGET_TYPE} –ai helloWorld.${TARGET_TYPE} –o helloWorld.update
Legato $ av-pack –f helloWorld.${TARGET_TYPE} <type name> _build_helloWorld/${TARGET_TYPE}
\end{DoxyVerb}


Where \char`\"{}$<$type name$>$\char`\"{} must be a globally-\/unique app type identifier, unique among all apps in all companies anywhere on Air\+Vantage. It should include company name, or user name if it\textquotesingle{}s a developer\textquotesingle{}s sample app. It should look like this\+:

\begin{DoxyVerb} av-pack -f helloWorld.wp85 abcCo.jsmith.helloWorld _build_helloWorld/wp85
\end{DoxyVerb}
\hypertarget{how_to_a_v_install_app_appInstallAirVantage_usingAv}{}\paragraph{Using Air\+Vantage Platform}\label{how_to_a_v_install_app_appInstallAirVantage_usingAv}
This part assumes you\textquotesingle{}ve already created your target/device on the platform using the section \hyperlink{connecting_air_vantage}{Connect to Air\+Vantage}

{\bfseries Upload app}

\begin{DoxyVerb}Upload `<uniqueAppName.targetType.zip>`
\end{DoxyVerb}


Zip file name will be something like {\ttfamily hello\+Worldabc\+Co.\+wp85.\+zip}.

See \href{https://doc.airvantage.net/display/USERGUIDE/Getting+Started}{\tt Air\+Vantage Getting Started} Develop+\+Activity \+: \char`\"{}\+My
\+Apps\char`\"{} Section -\/$>$ action Release ( select publish) for details.

{\bfseries Create app installation job}

You need to create the app install job on the platform\+: on the \char`\"{}\+Monitor\char`\"{} view for your system, use the \char`\"{}\+More\char`\"{} menu, and select \char`\"{}\+Install app\char`\"{} and then select the app you just uploaded,released and published in previous step.\hypertarget{how_to_a_v_install_app_appInstallAirVantage_rcvAppAgent}{}\paragraph{Receive app on Air\+Vantage agent}\label{how_to_a_v_install_app_appInstallAirVantage_rcvAppAgent}
{\bfseries Connect to server to receive app installation job}

This requires an {\ttfamily avc} control app using the L\+W\+M2\+M A\+V\+C A\+P\+I that accepts the download and installation. See \hyperlink{c_le_avc}{Air\+Vantage Connector} A\+P\+I for details.

{\bfseries Check Success Status}

If the installation was successful, you should find hello\+World in the installed apps and on the devices\textquotesingle{} Monitor view app list.\hypertarget{how_to_a_v_install_app_appInstallAirVantage_goingFurther}{}\paragraph{Going further}\label{how_to_a_v_install_app_appInstallAirVantage_goingFurther}
Start, stop, uninstall jobs can be created using corresponding U\+I buttons on the \char`\"{}\+Monitor\char`\"{} view for your system, in the \char`\"{}\+Applications\char`\"{} section, next to the app name you choose.





See \href{https://doc.airvantage.net/display/USERGUIDE/Getting+Started}{\tt Air\+Vantage Getting Started} more details.





Copyright (C) Sierra Wireless Inc. Use of this work is subject to license. \hypertarget{howToAVInstallApp}{}\subsection{Install apps}\label{howToAVInstallApp}
This page explains how to build your app, configure the Air\+Vantage platform, and then deploy your app remotely to your device.\hypertarget{how_to_a_v_install_app_appInstallAirVantage_buildapp}{}\subsubsection{Build app}\label{how_to_a_v_install_app_appInstallAirVantage_buildapp}
Before you build your target app, ensure Legato is built and your shell environment is setup to use command-\/line tools (i.\+e., run {\ttfamily bin/legs}).

Then, run the following sample on your build host replacing {\ttfamily \$\{T\+A\+R\+G\+E\+T\+\_\+\+T\+Y\+P\+E\}} with your target device type (e.\+g., {\ttfamily wp85} or {\ttfamily ar7})\+:

\begin{DoxyVerb}Legato $ cd apps/sample/helloWorld
Legato $ mkapp -t ${TARGET_TYPE} helloWorld.adef
Legato $ update-pack ${TARGET_TYPE} –ai helloWorld.${TARGET_TYPE} –o helloWorld.update
Legato $ av-pack –f helloWorld.update ${TARGET_TYPE} <type name> _build_helloWorld/${TARGET_TYPE}
\end{DoxyVerb}


Where \char`\"{}$<$type name$>$\char`\"{} must be a globally-\/unique app type identifier, unique among all apps in all companies anywhere on Air\+Vantage. It should include company name, or user name if it\textquotesingle{}s a developer\textquotesingle{}s sample app. It should look like this\+:

\begin{DoxyVerb} av-pack -f helloWorld.update abcCo.jsmith.helloWorld _build_helloWorld/wp85
\end{DoxyVerb}
\hypertarget{how_to_a_v_install_app_appInstallAirVantage_usingAv}{}\subsubsection{Using Air\+Vantage Platform}\label{how_to_a_v_install_app_appInstallAirVantage_usingAv}
This part assumes you\textquotesingle{}ve already created your target/device on the platform using the section \hyperlink{connecting_air_vantage}{Connect to Air\+Vantage}

{\bfseries Upload app}

\begin{DoxyVerb}Upload `<uniqueAppName.targetType.zip>`
\end{DoxyVerb}


Zip file name will be something like {\ttfamily hello\+Worldabc\+Co.\+wp85.\+zip}.

See \href{https://doc.airvantage.net/display/USERGUIDE/Getting+Started}{\tt Air\+Vantage Getting Started} Develop+\+Activity \+: \char`\"{}\+My
\+Apps\char`\"{} Section -\/$>$ action Release ( select publish) for details.

{\bfseries Create app installation job}

You need to create the app install job on the platform\+: on the \char`\"{}\+Monitor\char`\"{} view for your system, use the \char`\"{}\+More\char`\"{} menu, and select \char`\"{}\+Install app\char`\"{} and then select the app you just uploaded,released and published in previous step.\hypertarget{how_to_a_v_install_app_appInstallAirVantage_rcvAppAgent}{}\subsubsection{Receive app on Air\+Vantage agent}\label{how_to_a_v_install_app_appInstallAirVantage_rcvAppAgent}
{\bfseries Connect to server to receive app installation job}

This requires an {\ttfamily avc} control app using the L\+W\+M2\+M A\+V\+C A\+P\+I that accepts the download and installation. See \hyperlink{c_le_avc}{Air\+Vantage Connector} A\+P\+I for details.

{\bfseries Check Success Status}

If the installation was successful, you should find hello\+World in the installed apps and on the devices\textquotesingle{} Monitor view app list.\hypertarget{how_to_a_v_install_app_appInstallAirVantage_goingFurther}{}\subsubsection{Going further}\label{how_to_a_v_install_app_appInstallAirVantage_goingFurther}
Start, stop, uninstall jobs can be created using corresponding U\+I buttons on the \char`\"{}\+Monitor\char`\"{} view for your system, in the \char`\"{}\+Applications\char`\"{} section, next to the app name you choose.





See \href{https://doc.airvantage.net/display/USERGUIDE/Getting+Started}{\tt Air\+Vantage Getting Started} more details.





Copyright (C) Sierra Wireless Inc. Use of this work is subject to license. \hypertarget{howToAVData}{}\subsection{Manage Air\+Vantage Data}\label{howToAVData}
This topic describes how to manage data (read/write/execute) for cloud sevices like Air\+Vantage.

Legato provides easy L\+W\+M2\+M connectivity to Air\+Vantage through a .cdef file \hyperlink{def_files_cdef_defFilesCdef_assets}{Assets} section using pre-\/built A\+P\+Is.\hypertarget{how_to_a_v_data_howToAVData_APIs}{}\subsubsection{Air\+Vantage A\+P\+Is}\label{how_to_a_v_data_howToAVData_APIs}
Use the \hyperlink{c_le_avc}{Air\+Vantage Connector} A\+P\+I to configure your Air\+Vantage connectivity to handle app updates.

Use the \hyperlink{c_le_avdata}{Air\+Vantage Data} A\+P\+I to setup your Air\+Vantage data including read (variable), write (settings), and commands (executables).\hypertarget{how_to_a_v_data_howToAVData_assets}{}\subsubsection{.\+cdef Assets}\label{how_to_a_v_data_howToAVData_assets}
You add an \hyperlink{def_files_cdef_defFilesCdef_assets}{Assets}\textquotesingle{} section to your app\textquotesingle{}s component definition file ({\ttfamily }.cdef) customizing Air\+Vantage data including associating data types, setting default values, and executing commands using the pre-\/built \hyperlink{c_le_avdata}{Air\+Vantage Data} A\+P\+Is.





Copyright (C) Sierra Wireless Inc. Use of this work is subject to license. \hypertarget{howToLogs}{}\section{Use Logs}\label{howToLogs}
This topic summarizes how to use Legato logging.

There\textquotesingle{}s also some \hyperlink{basicLog}{background} info.





Legato creates log messages in \hyperlink{le__log_8h_a23e6d206faa64f612045d688cdde5808}{L\+E\+\_\+\+I\+N\+F\+O()} by default Only minimal info is reported, and only for the current app\+: essentially logs if the app is communicating. You need to modify default settings to enable monitoring for anything else.

There are two built-\/in features to control logging using the \hyperlink{how_to_logs_howToLogs_tool}{Logging Tool} or \hyperlink{how_to_logs_howToLogs_api}{Logging A\+P\+I}. There are also \hyperlink{c_logging_c_log_debugFiles}{App Crash Logs}.\hypertarget{how_to_logs_howToLogs_run}{}\subsection{Access Logs}\label{how_to_logs_howToLogs_run}
Run {\ttfamily logread} on the target to view the system log.

Run {\ttfamily logread -\/f } to start monitoring the logs and display messages as they are logged.

The installed app\textquotesingle{}s output \hyperlink{le__log_8h_a23e6d206faa64f612045d688cdde5808}{L\+E\+\_\+\+I\+N\+F\+O()} log message will appear in the target\textquotesingle{}s system log (syslog).\hypertarget{how_to_logs_howToLogs_tool}{}\subsection{Logging Tool}\label{how_to_logs_howToLogs_tool}
The target {\ttfamily log} tool is the easiest way to set logging controls. You can control what\textquotesingle{}s being logged, filter levels, trace keywords, and processes all through the command-\/line in a running system.

Run {\ttfamily log level I\+N\+F\+O \char`\"{}process\+Name/component\+Name\char`\"{}} to set the log level to I\+N\+F\+O for the specified component in a process.

Run {\ttfamily log trace \char`\"{}keyword\char`\"{} \char`\"{}process\+Name/component\+Name\char`\"{}} to use a keyword trace.

See \hyperlink{toolsTarget_log}{log} for details.\hypertarget{how_to_logs_howToLogs_syslogDefault}{}\subsection{Default syslog}\label{how_to_logs_howToLogs_syslogDefault}
By default, app processes will have their {\ttfamily stdout} and {\ttfamily stderr} redirected to the {\ttfamily syslog}. Each process’s stdout will be logged at I\+N\+F\+O severity level; it’s stderr will be logged at “\+E\+R\+R” severity level.

See \hyperlink{c_logging_c_log_basic_defaultSyslog}{Standard Out and Standard Error in Syslog} for more info.\hypertarget{how_to_logs_howToLogs_api}{}\subsection{Logging A\+P\+I}\label{how_to_logs_howToLogs_api}
The Logging A\+P\+I provides a toolkit to set error, warning, info, and debugging messages with macros and condition support including default environment variable controls that can be output to different devices and formats.

See \hyperlink{c_logging}{Logging A\+P\+I} for details.





Copyright (C) Sierra Wireless Inc. Use of this work is subject to license. 